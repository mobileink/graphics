\documentclass{article}

\usepackage{fontspec,xltxtra,xunicode}
\usepackage{fontspec}
\defaultfontfeatures{Scale=MatchLowercase}

%% \setmainfont[Mapping=tex-text]{Times New Roman}
%% \setromanfont[Mapping=tex-text]{Times New Roman}
%% \setsansfont[Mapping=tex-text]{Arial}

\setmainfont[Mapping=tex-text]{TeX Gyre Pagella}
%% \setromanfont[Mapping=tex-text]{TeX Gyre Pagella}
%% \setsansfont[Mapping=tex-text]{TeX Gyre Heros}

%% \setmainfont[Mapping=tex-text]{Minion Pro}
%% \setromanfont[Mapping=tex-text]{Minion Pro}
%% \setsansfont[Mapping=tex-text]{TeX Gyre Heros}

%% \setmonofont{Courier}

%% \usepackage{changepage}

%% \usepackage{graphicx}

%% \usepackage{tikz}
\usepackage{pgfplots}
\pgfplotsset{compat=1.9}

\usetikzlibrary{arrows,shapes,patterns,backgrounds,spy}

%% \usepgfplotslibrary{external}
%% \tikzexternalize
%% \tikzset{external/mode=graphics if exists}

%% \tikzset{external/system call={xelatex \tikzexternalcheckshellescape -shell-escape -halt-on-error -interaction=batchmode -jobname "\image" "\texsource" }}

%% %pdf setup
%% \tikzset{external/system call=%
%%     {xelatex \tikzexternalcheckshellescape -halt-on-error -interaction=batchmode -jobname "\image" "\texsource"}}

%% %optional png setup
%% \tikzset{
%%     png export/.style={
%%         external/system call=%
%%         {xelatex \tikzexternalcheckshellescape -halt-on-error -interaction=batchmode -jobname "\image" "\texsource";%
%%          convert -density 300 -transparent white "\image.pdf" "\image.png"; rm -f "\image.pdf"},
%%     }
%% }

%% \tikzset{%
%%     % Add size information to the .dpth file (png is in density not size)
%%     /pgf/images/external info,
%%     % Use the png export AND the import
%%     use png/.style={png export,png import},
%%     png import/.code={%
%%         \tikzset{%
%%             /pgf/images/include external/.code={%
%%                 % Here you can alter to whatever you want
%%                 % \pgfexternalwidth is only available if /pgf/images/external info
%%                 % is set
%%                 \includegraphics%
%%                 [width=\pgfexternalwidth,height=\pgfexternalheight]%
%%                 {{##1}.png}%
%%             }%
%%         }%
%%     }%
%% }

%% %%%<
%% \usepackage{verbatim}
%% \usepackage[active,tightpage]{preview}
%% \setlength\PreviewBorder{0pt}%
%% %%%>


%% \begin{comment}
%% :Title: Animated Normal Distrib

%% \end{comment}

\usepackage{animate}

%%%% macros %%%%


%%%%%%%%%%%%%%%%
\title{Animated Normal Curve}
%% \author{G. A. Reynolds}
%% \date{\today}

%%%%%%%%%%%%%%%%%%%%%%%%%%%%%%%%%%%%%%%%%%%%%%%%%%%%%%%%%%%%%%%%
\begin{document}

%% \maketitle

%% \pgfdeclarelayer{background layer}
%% \pgfdeclarelayer{foreground layer}
%% \pgfsetlayers{background layer,main,foreground layer}


%%%%%%%%%%%%%%%%

%% \def\mu{0}
%% \def\sigma{1}

\pgfplotsset{width=10cm,compat=1.9}
\pgfmathdeclarefunction{gaussian}{3}{%
  \pgfmathparse{1/(#3*sqrt(2*pi))*exp(-((#1-#2)^2)/(2*#3^2))}%
}
\pgfmathsetmacro\rMu{0}
\begin{animateinline}[poster=first,controls]{8}%
   \multiframe{65}{rSigma=2+-.03}{%
     \message{\rSigma}

  %% \tikzset{use png}

     \begin{tikzpicture}
       \useasboundingbox (-1,-1) rectangle (10,5);

%% \tikzset{
%%     %  Defines a custom style which generates BOTH, .pdf and .png export
%%     %  but prefers the .png on inclusion.
%%     %
%%     %  This style is not pre-defined, you may need to copy-paste and
%%     %  adjust it.
%%     png export/.style={
%%         external/system call/.add=
%%             {}
%%             {; convert -density 300 -transparent white "\image.pdf" "\image.png"},
%%         %
%%         /pgf/images/external info,
%%         /pgf/images/include external/.code={%
%%           \includegraphics
%%               [width=\pgfexternalwidth,height=\pgfexternalheight]
%%               {##1.png}%
%%         }
%%     }
%% }

%%        \tikzset{png export}

       %% \draw [help lines] (-1,-1) grid (10,5);
       %% \tikzset{
       %%   every pin/.style={fill=yellow!50!white,rectangle,rounded corners=3pt,font=\tiny},
       %%   small dot/.style={fill=black,circle,scale=0.3}
       %% }

       \pgfmathsetmacro\valueMu{gaussian(\rMu,\rMu,\rSigma)}
       \pgfmathsetmacro\valueSigma{gaussian(\rSigma,\rMu,\rSigma)}

       \pgfmathsetmacro\valueB{gaussian(2,\rMu,\rSigma)}
       \pgfmathsetmacro\valueC{gaussian(3,\rMu,\rSigma)}

       \begin{axis}[
           %% legend style={
           %%   %% cells={anchor=east},
           %%   %% legend pos=outer north east,
           %% },
           %% title={$\mu=3$, $\sigma=1$}],
           clip=false,
           no markers,
           domain=-4:4,
           samples=100,
           xmin=-4.2,
           xmax=4.2,
           xlabel={},
           xtick={-4,4},
           %% xticklabel=\empty,
           axis x line=bottom,
           x axis line style={<->},
           ymin=0,
           ymax=1.2,
           ylabel={},
           ytick={0,1},
           %% yticklabel=\empty,
           axis y line=left,
           %% every axis y label/.style={at=(current axis.above origin),anchor=south},
           %% every axis x label/.style={at=(current axis.below),anchor=south},
           height=5cm,
           width=10cm,
           %% xtick=\empty,
           %% ytick=\empty,
           %% enlargelimits=false,
           %% clip=false,
           %% axis on top,
           %% grid = major
           %% hide y axis
         ]
         %% \addlegendentry[align=left]{$\sigma=\rSigma$};
         \addplot [gray] {gaussian(x,0,1)};
         \addplot [very thick,cyan!50!black] {gaussian(x, \rMu, \rSigma)};

         %% \node (25.4,.5) {$\mu=3$} ;
         %% \node[small dot,pin=-30:{$(1,1)$}] at (axis description cs:1,1) {};
         \draw [gray] (axis cs:-\rSigma,0) -- (axis cs:-\rSigma,\valueSigma)
         (axis cs:\rSigma,0) -- (axis cs:\rSigma,\valueSigma);

         %% \node[below] at (axis cs:\rMu, 0)  {$\mu=\rMu$};
         %% \node[pin=west:{$1$}] at (axis cs:-3.2, 1) {};
         \draw (axis cs:-0.5,-.1) node[below] {$\mu=0$} -- (axis cs:0,0) ;
         \draw (axis cs:1,-.1) node[below right] {$\sigma=\rSigma$} -- (axis cs:\rSigma,0) ;

         %% \draw [gray] (axis cs:-2\rSigma,0) -- (axis cs:-2\rSigma,1)
         %% (axis cs:2\rSigma,0) -- (axis cs:2\rSigma,1);
         \draw [black] (axis cs:0,0) -- (axis cs:0,\valueMu) ;
         %% \draw [yshift=2.6cm, latex-latex](axis cs:-1, 0) -- node [fill=white] {$0.683$} (axis cs:1, 0);
         %% \draw [yshift=2.0cm, latex-latex](axis cs:-2, 0) -- node [fill=white] {$0.954$} (axis cs:2, 0);
         %% \node[above] at (axis cs:1, 1)  {$\mu - 2\sigma$};
         %% \node[above] at (axis cs:2, 1)  {$\mu - \sigma$};
         %% \node[above] at (axis cs:3, 1)  {$\mu$};
       \end{axis}
     \end{tikzpicture}
   }
\end{animateinline}

%%%%%%%%%%%%%%%%
% define the bounding box
%% % Create n frames with \xb as parameter
%% \foreach \xb in {0,0.1,...,2.1}{
%%     \def\setA{(0,0) circle (1)}

%% \begin{animateinline}[poster=first,controls]{8}%
%%    \multiframe{21}{rxb=0+.1,icount=1+1,rvo=3+0}{%

%%     \begin{tikzpicture}
%%      \draw (-2,2) rectangle (4,-2) ; % bbox

%%     \def\setB{(\rxb,0) circle (1)}

%%         % intersection
%%         \begin{scope}
%%             \clip \setA;
%%             \fill[black!20] \setB;
%%         \end{scope}
%%         \begin{scope}[even odd rule]% first circle without the second
%%             \clip \setB \boundb;
%%             \fill[red!20] \setA;
%%         \end{scope}
%%         \begin{scope}[even odd rule]% first circle without the second
%%             \clip \setA \boundb;
%%             \fill[blue!20] \setB;
%%         \end{scope}
%%         \draw \setA;
%%         \draw \setB;
%%         \node at (-1,0) [left] {$A$};
%%         \node at (\rxb+1,0) [right] {$B$};
%%         \node at (4,2) [below left] {$A\cap B$};
%%     \end{tikzpicture}
%% }% multiframe
%% \end{animateinline}

\end{document}
