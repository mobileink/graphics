% Author: Gregg Reynolds
\documentclass{article}
\usepackage{etoolbox}

\usepackage[latin1]{inputenc}
\usepackage{amsmath}
\usepackage{rotating}

\usepackage{tikz}
\usetikzlibrary{%
  calc,%
  patterns,%
  arrows,%
  decorations,%
  decorations.markings%
}

\usepackage{animate}

\usepackage{verbatim}

\begin{comment}
  :Title: Logarithm
  :Tags: Plots, GNUPLOT, External file

\end{comment}

\title{e}
\author{Gregg Reynolds}
%\date{}                                           % Activate to display a given date or no date

%%%% macros

\pgfmathdeclarefunction{easeInQuadratic}{4}{%
  \pgfmathparse{#3*(#1/#4)^2+#2} %% c(t/d)^2 + b
}

%%%%%%%%%%%%%%%%%%%%%%%%%%%%%%%%%%%%%%%%%%%%%%%%%%%%%%%%%%%%%%%%
\begin{document}
\pagestyle{empty}

%% constants
%%  counters: use i prefix

%% tikz bounding boxes
%% e graphs
\pgfmathsetmacro\bbminx{-2}
\pgfmathsetmacro\bbminy{-2}
\pgfmathsetmacro\bbmaxx{4}
\pgfmathsetmacro\bbmaxy{3}

%% tkz bounding box
\pgfmathsetmacro\tkzminx{-2}
\pgfmathsetmacro\tkzminy{-2}
\pgfmathsetmacro\tkzmaxx{12}
\pgfmathsetmacro\tkzmaxy{6}

\newcommand{\bbdraw}{draw}

\newcommand{\tikzscale}{2}

%% \def\iFrameRate{20}	%%
%% \def\iFrames{30}	%% d : duration
\newcounter{iFrameRate}
\setcounter{iFrameRate}{30}

\newcounter{iFrames}
\setcounter{iFrames}{30}

%% \def\rFromX{1}	%% b : begin point
%% \def\rToX{e}	%%   : end point
%% \def\rFromY{14}	%% b : begin point
%% \def\rToY{0}	%%   : end point
%% \pgfmathsetmacro\rDeltaX{\rToX-\rFromX}  %% c :  end-begin, total change
%% \pgfmathsetmacro\rDeltaY{\rToY-\rFromY}  %% c :  end-begin, total change

%% \def\r{8pt} %% circle radius

\pgfmathsetmacro\rFromXa{1}	%% b : begin point
\pgfmathsetmacro\rToXa{e}	%%   : end point
\pgfmathsetmacro\rFromXb{e}	%% b : begin point
\pgfmathsetmacro\rToXb{e^2}	%% b : begin point
%% \pgfmathparse{e^2}
%% \pgfmathsetmacro\rToXb{\pgfmathresult} %%   : end point
\def\rFromY{14}	%% b : begin point
\def\rToY{0}	%%   : end point
\pgfmathsetmacro\rDeltaXa{\rToXa-\rFromXa}  %% c :  end-begin, total change
\pgfmathsetmacro\rDeltaXb{\rToXb-\rFromXb}  %% c :  end-begin, total change
\pgfmathsetmacro\rDeltaY{\rToY-\rFromY}  %% c :  end-begin, total change

\begin{figure}[ht]
  \center
  \caption{Hyperbolic Rotation 1}
  \hspace*{-3cm}
  %%%%%%%%%%%%%%%%%%%%%%%%%%%%%%%%%%%%%%%%%%%%%%%%%%%%%%%%%%%%%%%%
%% Hyperbolic Rotation 1
%% simple circular and hyperbolic rotation - sweep of radius
%%%%%%%%%%%%%%%%%%%%%%%%%%%%%%%%%%%%%%%%%%%%%%%%%%%%%%%%%%%%%%%%
\pgfmathparse{\value{iFrameRate}}
\begin{animateinline}[%
    poster=first%
    %% ,draft
    %% ,loop
    ,controls%
  ]{\pgfmathresult}%

  %%%%  INITIAL FRAME REQUIRED in order to get n+1 frames (good starts and ends)

  %% \begin{tikzpicture}
  %%   \useasboundingbox[draw] {(-1,-2) rectangle (10,16)};

  %% \end{tikzpicture}
  %% \newframe

  %%%% MULTIFRAMES START FROM 1 NOT 0
  \multiframe{\pgfmathresult}{iFrame=1+1}{%
    %% \message{\iFrame}
    %% \message{\arabic{iFrames}}
    %% \message{\rFromXa}
    %% \message{\rToXa}
    %% \message{\rFromXb}
    %% \message{\rToXb}

    \pgfmathsetmacro\xa{easeInQuadratic(\iFrame,\rFromXa,\rDeltaXa,\value{iFrames})};
    \pgfmathsetmacro\xb{easeInQuadratic(\iFrame,\rFromXb,\rDeltaXb,\value{iFrames})};
    %% \pgfmathsetmacro\newx{easeInQuadratic(\iFrame,\rFromXa,\rDeltaXa,\value{iFrames})};

    \begin{tikzpicture}[%
        scale=\tikzscale
        %% ,domain=-.3:1.3
      ]
      %% \if usebb
      \useasboundingbox[\bbdraw] {(\bbminx,\bbminy) rectangle (\bbmaxx,\bbmaxy)};
      %% \fi


      \draw[very thin,color=gray] (-1,0) grid (3.2,1.2);

      %% axis of symmetry:  f(x) = x
      \draw[<->,color=gray] (-1,-1) -- (1.2,1.2) node[right] {$f(x) =x$};
      \draw[black] (0,0) -- (1,1);

      %% e diameter
      \draw[black] (0,0) -- (e,1/e);

      %% %% coordinate rectangle
      %% \begin{scope}[domain=1:e, even odd rule]
      %%   \draw[ultra thick,black,domain=1:e] (0,0) -- (0,{1/\xa}) -- (\xa,{1/\xa}) -- (\xa,0) -- cycle;
      %%   \clip[domain=1:e] (0,0) -- (0,{1/\xa}) -- (\xa,{1/\xa}) -- (\xa,0) -- cycle;
      %%   \draw[pattern=horizontal lines light blue] (0,0) rectangle (e,1);
      %% \end{scope}

      %%  unit circle
      \begin{scope}
        %% \clip (0,0) rectangle (1.1,1.1);
        \draw (0,0) circle (1);
      \end{scope}

      %% ref segments
      \draw[gray] (e,0) -- (e,1/e);

      \draw[gray] (0,1/e) node[left] {$e^{-1}$} -- (e,1/e);

      \draw[gray] (0,.5) node[left] {$0.5$} -- (e,.5);

      %% dot origin
      \fill[black] (0,0) node[below left] {$O$} circle (.4pt);

      %% dot (0,1/e)
      \fill[black] (0,1/e) circle (.4pt);

      %% dot (0,1)
      \fill[black] (0,1) node[above left] {$1$} {} circle (.4pt);
      %% dot (1,0)
      \fill[black] (1,0) node[below right] {$1$} circle (.4pt);

      %% dot (e,0)
      \fill[black] (e,0) node[below] {$e$} circle (.4pt) -- (e,1.2);

      %% e area
      \begin{scope}[domain=1:e, even odd rule]
        \clip[domain=\xa:\xb] (1,0) --  plot (\xa,{(1/\xa)}) -- (\xa,0) -- cycle;
        \draw[pattern=vertical lines,pattern color=blue] (0,0) rectangle (e,1);
      \end{scope}

      %% hyperbolic sector
      \begin{scope}[domain=1:e, even odd rule]
        \clip[domain=1:\xa] (0,0) --  plot (\xa,{(1/\xa)}) -- cycle;
        \fill[color=green!50!blue,opacity=.15] (0,0) rectangle (e,1);
      \end{scope}

      %% f(x) = 1/x
      \draw[<->,domain=.7:3.2,color=green!50!black] plot (\xa,{(1/\xa)});
      \draw[color=green!50!black] (.8,1.2) node[left] {$f(x) = \frac{1}{x}$};

      %% hyperbolic arc
      \draw[thick,red,domain=1:\xa] plot (\x,{(1/\x)});
      %% hyperbolic radii
      \draw[very thick] (1,0) -- (1,1);
      \draw[semithick] (0,0) -- (\xa,1/\xa);

      %%%%%%%%%%%%%%%%

      %% sweeping vector
      %%  \draw[-latex,black,postaction={decorate,
      %%                     decoration={markings,
      %%                       mark=at position 0.74 with {\arrow{latex}},
      %%                       mark=at position 0.82 with {\arrow{latex}};}
      %%                    }] (0,0) -- (1.8, 1);
      %% \draw [->] (1.8,.975) -- (1.85,.975);

      %%%%%%%%%%%%%%%%
    \end{tikzpicture}
  }% multiframe
\end{animateinline}

%%% Local Variables: 
%%% mode: latex
%%% TeX-master: t
%%% End: 

\end{figure}

\end{document}
