% Author: Gregg Reynolds
\documentclass{article}

\usepackage[latin1]{inputenc}
\usepackage{amsmath}
\usepackage{rotating}

\usepackage{tikz}
\usetikzlibrary{%
  calc,%
  patterns,%
  arrows,%
  decorations,%
  decorations.markings%
}

\usepackage{animate}

\usepackage{verbatim}

\begin{comment}
  :Title: Logarithm
  :Tags: Plots, GNUPLOT, External file

\end{comment}

\title{e}
\author{Gregg Reynolds}
%\date{}                                           % Activate to display a given date or no date

%%%% macros

\pgfmathdeclarefunction{easeInQuadratic}{4}{%
  \pgfmathparse{#3*(#1/#4)^2+#2} %% c(t/d)^2 + b
}

%%%%%%%%%%%%%%%%%%%%%%%%%%%%%%%%%%%%%%%%%%%%%%%%%%%%%%%%%%%%%%%%
\begin{document}
\pagestyle{empty}

%% constants
%%  counters: use i prefix

%% tikz bounding boxes
%% e graphs
\pgfmathsetmacro\bbminx{-2}
\pgfmathsetmacro\bbminy{-2}
\pgfmathsetmacro\bbmaxx{e^2+1}
\pgfmathsetmacro\bbmaxy{3}

%% tkz bounding box
\pgfmathsetmacro\tkzminx{-2}
\pgfmathsetmacro\tkzminy{-2}
\pgfmathsetmacro\tkzmaxx{12}
\pgfmathsetmacro\tkzmaxy{6}

\newcommand{\bbdraw}{draw}

\newcommand{\tikzscale}{1.5}

%% \def\iFrameRate{20}	%%
%% \def\iFrames{30}	%% d : duration
\newcounter{iFrameRate}
\setcounter{iFrameRate}{30}

\newcounter{iFrames}
\setcounter{iFrames}{30}

%% \def\rFromX{1}	%% b : begin point
%% \def\rToX{e}	%%   : end point
%% \def\rFromY{14}	%% b : begin point
%% \def\rToY{0}	%%   : end point
%% \pgfmathsetmacro\rDeltaX{\rToX-\rFromX}  %% c :  end-begin, total change
%% \pgfmathsetmacro\rDeltaY{\rToY-\rFromY}  %% c :  end-begin, total change

%% \def\r{8pt} %% circle radius

\pgfmathsetmacro\rFromXa{1}	%% b : begin point
\pgfmathsetmacro\rToXa{e}	%%   : end point
\pgfmathsetmacro\rFromXb{e}	%% b : begin point
\pgfmathsetmacro\rToXb{e^2}	%% b : begin point
%% \pgfmathparse{e^2}
%% \pgfmathsetmacro\rToXb{\pgfmathresult} %%   : end point
\def\rFromY{14}	%% b : begin point
\def\rToY{0}	%%   : end point
\pgfmathsetmacro\rDeltaXa{\rToXa-\rFromXa}  %% c :  end-begin, total change
\pgfmathsetmacro\rDeltaXb{\rToXb-\rFromXb}  %% c :  end-begin, total change
\pgfmathsetmacro\rDeltaY{\rToY-\rFromY}  %% c :  end-begin, total change

\begin{figure}[ht]
  \center
  \caption{Hyperbolic Rotation 2}
  \hspace*{-3cm}
  \include{hyperbolic.rotation.2}
\end{figure}

\end{document}
