%%%%%%%%%%%%%%%%%%%%%%%%%%%%%%%%%%%%%%%%%%%%%%%%%%%%%%%%%%%%%%%%
%%  Equilateral Hyperbola - 4th Quadrant
%%%%%%%%%%%%%%%%%%%%%%%%%%%%%%%%%%%%%%%%%%%%%%%%%%%%%%%%%%%%%%%%
\begin{tikzpicture}[scale=3,domain=-.3:1.3]

 \draw[very thin,color=gray] (-3.2,-3.2) grid (3.2,3.2);
%  \draw[->] (-1.2,0) -- (3.2,0) node[right] {$x$};
%  \draw[->] (0,-1.2) -- (0,2.2) node[above] {$f(x)$};

%  axes (h, v)
 \draw[<->,black] (-3.2,0) -- (3.2,0);
 \draw[<->,black] (0,-3.2) -- (0,3.2);

 %% focii:  (sqrt 2, sqrt 2)
 \filldraw[red] (\sqrttwo,\sqrttwo)
 node[above,black] {$\text{F}$}
 node[right,black] {$\scriptstyle(\sqrt{2},\sqrt{2})$}  circle (.5pt);
 \fill[red] (-\sqrttwo,-\sqrttwo)
 node[below,black] {$\text{F}'$}
 node[left,black] {$\scriptstyle(-\sqrt{2},-\sqrt{2})$}  circle (.5pt);


 %% transverse axis of hyperbola:  f(x) = x
  \draw[color=gray] (-3.2,-3.2) -- (3.2,3.2) node[right] {$f(x) =x$};

 %% conjugate axis of hyperbola:  f(x) = -x
  \draw[color=gray] (-3.2,3.2) -- (3.2,-3.2) node[right] {$f(x) =-x$};

  %% transverse semi-axis length sqrt(2)
  \draw[black] (0,0) -- (1,1) node[above right,black] {A};
  %% conjugate semi-axis length 2
  \draw[black,dashed] (0,2) -- (-1,1) node[above left] {B} -- (-2,0) -- (0,-2) -- (2,0) -- (1,1);
%  \draw[gray] (0,2) -- (2,0);
  \draw[black] (0,2) -- (1,1);

  %% latera recta
 \begin{scope}[domain=1/e:e]
   \clip (e,e) -- (1/e,e) -- plot (\x,{(1/\x)}) -- (e,1/e) -- cycle;
  \draw ({sqrt(8)},0) -- (0,{sqrt(8)});
  \end{scope}

  %% some ref lines:  sqrt 2
  \draw[gray] (1/\sqrttwo,0) node[color=black, below] {$\frac{1}{\sqrt{2}}$}
  -- (1/\sqrttwo,1/\sqrttwo);
  \draw[gray] (0,1/\sqrttwo)  node[color=black,left] {$\frac{1}{\sqrt{2}}$}
 -- (1/\sqrttwo,1/\sqrttwo);

  \draw[gray] (\sqrttwo,0) node[color=black, below] {$\scriptstyle\sqrt{2}$}
  -- (\sqrttwo,\sqrttwo);
  \draw[gray] (0,\sqrttwo)  node[color=black,left] {$\scriptstyle\sqrt{2}$}
 -- (\sqrttwo,\sqrttwo);

 %% directrix
  \draw[gray] (\sqrttwo,0) -- ($(\sqrttwo,0)!2!(0,\sqrttwo)$);
  \draw[gray] (\sqrttwo,0) -- ($(\sqrttwo,0)!-1!(0,\sqrttwo)$);

  \draw[gray] (-\sqrttwo,0) node[color=black, above] {$-\scriptstyle\sqrt{2}$}
  -- (-\sqrttwo,-\sqrttwo);
  \draw[gray] (0,-\sqrttwo)  node[color=black, right] {$-\scriptstyle\sqrt{2}$}
 -- (-\sqrttwo,-\sqrttwo);

  %%%%%%%%%%%%%%%%
  % shade hyperbolic and circle sectors
  \begin{scope}[domain=1:e, even odd rule]
%    \clip (0,0) -- (e,1/e) -- (1,1) -- cycle;
%    \fill[color=blue,opacity=.15] (0,0) circle(1);

%    \clip (0,0) circle(1) (0,0)--(1,1) -- plot (\x,{(1/\x)}) -- (e,1/e) -- cycle;
    \fill[opacity=.15] (0,0)--(1,1) -- plot (\x,{(1/\x)}) -- (e,1/e) -- cycle;
 \end{scope}

 % shade "area under curve" in 3rd quadrant
 \begin{scope}[domain=1:e, even odd rule]
%    \clip (0,0) -- (e,1/e) -- (1,1) -- cycle;
%    \fill[color=blue,opacity=.15] (0,0) circle(1);
%   \clip (0,0) circle(1) (0,0)--(1,1) -- plot (\x,{(1/\x)}) -- (e,1/e) -- cycle;
    \fill[opacity=.15] (-1,0)--(-1,-1) -- plot (-\x,{(-1/\x)}) -- (-e,-1/e) -- (-e,0) -- cycle;
 \end{scope}

 %% e diameter
  \draw[black] (-e,-1/e) -- (e,1/e);

%%%%%%%%%%%%%%%%
%% circle stuff
%%  unit circle
  \draw[thick,black] (0,0)  circle (1);
  \draw[thick,black] (-1,-1) rectangle (1,1);
%  \draw[thick,black] (-\sqrttwo,-\sqrttwo) rectangle (\sqrttwo,\sqrttwo);
   % %% 1 radian
   % \draw[thick,red] (1,0) arc [start angle=0,radius=1,end angle=deg(1)]
   % node[black,above right] {$1$} -- (0,0);
   % \draw[thick,red] (0,0) -- (.6420,1); %% tan(pi/2 - 1)

   % %% 2 radians
   % \draw[thick,red] (1,0) arc [start angle=0,radius=1,end angle=deg(2)]
   % node[black,above right] {$2$} -- (0,0);
   % \draw[thick,red] (0,0) -- (-.4576,1); %% tan(2-pi/2)

   % %% e radians
   % \draw[thick,red] (1,0) arc [start angle=0,radius=1,end angle=deg(e)]
   % node[black,right=4pt,above=1pt] {\begin{turn}{45}$\theta=e\ rads$\end{turn}} -- (0,0);
   % \draw[thick,red] (0,0) -- (-1,.4505); %% tan(pi-e)


%   \draw[thick,red] (0,0) -- (1,e); %% (-.210641,1); %% tan(pi/4+1-pi/2)

   %% pi/4 + 1 radians
%   \draw[thick,red] (1,0) arc [start angle=0,radius=1,end angle=deg((pi/4)+1)]
%   node[black,above left] {$\scriptstyle\frac{\pi}{4}+1$} -- (0,0);
  \draw[thick,red] (0,0) -- ($(0,0)!3.3!{(pi/4+1) r}:(1,0)$);

   %% pi/4 + 2 radians
  \draw[thick,black] (1,0) arc [start angle=0,radius=1,end angle=deg(pi/4+2)] coordinate (t) {};
  \draw[thick,red] (t) -- (1,1);
  \draw[thick,red] (t) -- (-1,-1);
  \draw[thick,red] (t) -- ($(0,0)!1!pi/4 r:({cos(2 r)},0)$) -- (0,0);
%   node[black,above left] {$\scriptstyle\frac{\pi}{4}+2$} -- (0,0);
%   \draw[thick,red] (0,0) -- (-1,.3720); %% tan(pi/4+2)
%   \draw[thick,red] (0,0) -- (-e,1); %% tan(pi/4+2)
  \draw[thick,red] (0,0) -- ($(0,0)!3.3!{(pi/4+2) r}:(1,0)$);

   %% pi/4 + 3 radians
%   \draw[thick,red] (1,0) arc [start angle=0,radius=1,end angle=deg(pi/4+3)]
%   node[black,below left] {$\scriptstyle\frac{\pi}{4}+3$} -- (0,0);
%   \draw[thick,red] (0,0) -- (-1,.3720); %% tan(pi/4+2)
%  \draw[thick,red] (0,0) -- ($(0,0)!3.5!{(pi/4+3) r}:(1,0)$);

   %% pi/4 + 4 radians
%   \draw[thick,red] (1,0) arc [start angle=0,radius=1,end angle=deg(pi/4+4)]
%   node[black,below right] {$\scriptstyle\frac{\pi}{4}+4$} -- (0,0);
  \draw[thick,red] (0,0) -- ($(0,0)!3.2!{(pi/4+4) r}:(1,0)$);

   %% pi/4 + 5 radians
%   \draw[thick,red] (1,0) arc [start angle=0,radius=1,end angle=deg(pi/4+5)]
%   node[black,below right] {$\scriptstyle\frac{\pi}{4}+5$} -- (0,0);

   %% pi/4 + 6 radians
%   \draw[thick,red] (1,0) arc [start angle=0,radius=1,end angle=deg(pi/4+6)]
%   node[black,below right] {$\scriptstyle\frac{\pi}{4}+6$} -- (0,0);

   %% pi/4 + e radians
%  \draw[thick,red] (1,0) arc [start angle=0,radius=1,end angle=deg(pi/4+e)]
%  node[black,above] (e) {} -- (0,0);
%  \draw[->,black] (-1.5,-.5) node {$\theta=e$} -- (e);
  \draw[thick,red] (0,0) -- (-e,-1) node[below right] {$\theta=e$};
%   node[black,above left] {$\scriptstyle\frac{\pi}{4}+e$};
%   \draw[thick,red] (0,0) -- (-e,-e); %% tan(pi/4+2)
%  \draw[thick,red] (0,0) -- ($(0,0)!3.4!{(pi/4+pi) r}:(1,0)$);


   %% pi/4 + pi radians
%   \draw[thick,red] (1,0) arc [start angle=0,radius=1,end angle=deg(pi/4+pi)];
%   node[black,below right] {$\scriptstyle\frac{\pi}{4}+\pi$};
%   \draw[thick,red] (0,0) -- (-e,-e); %% tan(pi/4+2)

%   \draw (e,0) -- (tangent cs:node=unitCircle, point={(e,0)}, solution=1);


%   \draw[thick,red] (1,0) arc [start angle=0,radius=1,end angle=deg(pi/4-1)]
%   node[black,above right] {$1$} -- (0,0);

   %% root 2 circle
    \draw[gray] (0,0) circle (\sqrttwo);

    \draw[gray] (1/\sqrttwo,1/\sqrttwo) -- (\sqrttwo,0);

%    \draw[gray] (0,.25) node[left] {$.25$} -- (2,.25);

%   \draw[thick,green,dashed] (1,0) arc [start angle=0,radius=1,end angle=deg(1)] -- (0,0);
%   \draw[thick,green,dashed] (1,0) arc [start angle=0,radius=1,end angle=deg(2)] -- (0,0);

%   \draw[thick,black] (-2,3) -- (-1,3) arc [start angle=0,radius=1,end angle=deg(1)] -- cycle;

 % 2.  circular rectilinear rotation
 %% dot endpts of arc
%\draw[-latex,red] (1/\sqrttwo,1/\sqrttwo) -- (1/\sqrttwo,\esqinv);
% \draw[-latex,red] (1/\sqrttwo,\esqinv) -- (1,\esqinv);

 % 3.  circular arc travel
 \begin{scope}
 \clip (0,0) -- (1,1) -- (e,1/e) -- cycle;
 \draw[thick,red] circle (1);
\end{scope}

%\draw[gray] (1/\sqrttwo,1/\sqrttwo) -- (1,\esqinv); % NB: wrong x for
                                % pt 2

% 4. dots
% \fill[red] (1/\sqrttwo,1/\sqrttwo) node[above] {$A$} circle (.5pt);
% \fill[red] (1/\sqrttwo,\esqinv) node[left] {$B$} circle (.5pt);
% \fill[red] (1,\esqinv)  node[right] {$C$} circle (.5pt);


 %%%%%%%%%%%%%%%%
 % 1.  hyperbolic rectilinear rotation
%  \draw[-latex,red] (1,1) -- (e,1);
%  \draw[-latex,red] (e,1) -- (e,1/e);

%% f(x) = 1/x
  \draw[domain=.3:3.2,color=black,thick] plot (\x,{(1/\x)});
  \draw[color=black] (.35,3.1) node[above right] {$f(x) = \frac{1}{x}$};

  \draw[domain=-.3:-3.2,color=black,thick] plot (\x,{(1/\x)});
%  \draw[color=green!50!black] (.35,3.1) node[above right] {$f(x) = \frac{1}{x}$};

 %% chord
%  \draw[black] (1,1) -- (e,1/e);

  %% plotted arc
 \draw[thick,red,domain=1:e] plot (\x,{(1/\x)});

 %%%%%%%%%%%%%%%%

%% ref segments
  \draw[gray] (e,0) -- (e,1/e);
  \draw[gray] (-e,0) -- (-e,-1/e);

  \draw[gray] (-e,1/e) -- (e,1/e);
  \draw[gray] (0,-1/e) -- (-e,-1/e);

 \draw[gray] (-1/e,1) -- (-1/e,-1);
 
%  \draw[gray] (-1/\esq,0) -- (-1/\esq,1);
%  \draw[gray] (0,-1/e) -- (-e,-1/e);

  \draw[gray] (-e,-e) rectangle (e,e);
%  \draw[gray] (-e,e) -- (-e,-e);

 \draw[gray] (-{(e-2)},e) -- (-{(e-2)},-e);

%% dot origin
%  \fill[black] (0,0) node[below left] {$O$} circle (.4pt);

 %% dot (0,1)
  \fill[black] (0,1) node[left] {$1$} node[above right] {} circle (.4pt);
  %% dot (1,0)
  \fill[black] (1,0) node[below] {$1$} circle (.4pt);

 %% dot (1,1)
 % \fill[red] (1,1) node[above right] { $A'$}  circle (.5pt);

%% dot (0,1/e)
 \fill[black] (0,1/e) circle (.4pt) node[left] {$\frac{1}{e}$} circle (.4pt);
  %% dot (1/e,0)
 \fill[black] (1/e,0) circle (.4pt) node[below] {$\frac{1}{e}$} circle (.4pt);
 %% dot (1/e,1/e)
  \fill[black] (1/e,1/e) circle (.4pt);

 %% dot (0,-1/e)
 \fill[black] (0,-1/e) circle (.4pt) node[right] {$-\frac{1}{e}$} circle (.4pt);
  %% dot (-1/e,0)
 \fill[black] (-1/e,0) circle (.4pt) node[above] {$-\frac{1}{e}$} circle (.4pt);

 %% dot (e,0)
  \fill[black] (e,0) node[below] {$e$} circle (.4pt);
 %% dot (0,e)
  \fill[black] (0,e) node[left] {$e$} circle (.4pt);
 %% dot (-e,0)
  \fill[black] (-e,0) node[above] {$-e$} circle (.4pt);
 %% dot (-e,1)
  \fill[black] (-e,1) node[above left] {$(-e,1)$} circle (.4pt);
 %% dot (-e,-1)
  \fill[black] (-e,-1) node[below left] {$(-e,-1)$} circle (.4pt);

 %% dot (0,e)
  \fill[black] (0,-e) node[right] {$-e$} circle (.4pt);


%%  dot (1,e)
  \fill[black] (1,e) node[anchor=south west] {$(1,e)$} circle (.4pt);
%%  dot (-1,e)
  \fill[black] (-1,e) node[anchor=south west] {$(-1,e)$} circle (.4pt);
  
 %% dot (e,1)
  \fill[black] (e,1) node[above right,color=black]  {$(e,1)$} circle (.4pt); 

%% dot (e,1/e)
  \fill[black] (e, 1/e) circle (.4pt)
  node[above=2pt,right=8pt] {$\scriptstyle\theta$}
  node[above=5pt,left=6pt] {$\scriptstyle\theta$} ;
 %% dot (1/e,e)
  \fill[black] (1/e,e) node[above right] {$(1/e,e)$} circle (.4pt);
%% dot (-e,-1/e)
  \fill[black] (-e,-1/e) node[below left] {$(-e,-\frac{1}{e})$} circle (.4pt);
 %% dot (1/e,e)
  \fill[black] (-1/e,-e) node[below left] {$(-\frac{1}{e},-e)$} circle (.4pt);

 %% dot (e,e)
  \fill[black] (e, e) circle (.4pt) node[below right]  {$(e,e)$};
 %% dot (-e,e)
  \fill[black] (-e, e) circle (.4pt) node[below right]  {$(-e,e)$};
 %% dot (-e,-e)
  \fill[black] (-e,-e) circle (.4pt) node[below left]  {$(-e,-e)$};

  %%%%%%%%%%%%%%%%
  %% chords
%  \draw (e,1/e) -- (1/e,e);
  \draw (e,1/e) -- (\sqrttwo,\sqrttwo);
%  \draw (e,1/e) -- (-\sqrttwo,-\sqrttwo);
  \draw (-\sqrttwo,-\sqrttwo) -- ($(-\sqrttwo,-\sqrttwo)!1.1!(e,1/e)$);
  % \begin{scope}[domain=e:3] 
  %   \clip (e,1) -- (e,1/e) -- plot (\x,{(1/\x)}) cycle;
  %   \clip (e,1/e) (-\sqrttwo,-\sqrttwo) -- ($(-\sqrttwo,-\sqrttwo)!1.1!(e,1/e)$) -- cycle;
  %   \draw (e,1/e) circle (.1);
  % \end{scope}

% \draw (\e,0) arc (0:90:\e);
% \fill (canvas polar cs:angle=45,radius=2.71cm) circle (.4pt)
% node[right] {$P$};
% \fill (canvas polar cs:angle=30,radius=2.71cm)
% node[right]{$\frac{\pi\,e}{4} = PE$};

%  \fill (0,\e) circle (.4pt) node[anchor=east] {($0,e$)};

  %% dot (0, 1/e)
%  \fill[black] (1, 1/e) circle (.4pt);
 %% dot (e, 1/e)

 %%%% legend
  \draw (-2,1.5) node[fill=gray!50, text width=3cm]
  {$1/e=.367879$\\
    $1-1/e=0.632120$\\
    $e-2=0.718281$};

  \draw (1,-2.5) node[fill=gray!50, text width=6cm, right]
  {\begin{center} Unit Equilateral Hyperbola\\\end{center}
    Formula: $xy=1$ \\\vspace{2pt}
    Transverse semiaxis: O-A; length: $\sqrt{2}$\\
    Conjugate semiaxis: O-B; length: $\sqrt{2}$\\
    Focii: $\text{F,F}'; c = 2$\\\vspace{2pt}
    Eccentricity = $\frac{2\text{B}^2}{\text{A}} = \sqrt{2}$\\\vspace{4pt}
    Area of shaded regions: $1+1=2$
  };

\end{tikzpicture}

%%% Local Variables: 
%%% mode: latex
%%% TeX-master: t
%%% End: 
