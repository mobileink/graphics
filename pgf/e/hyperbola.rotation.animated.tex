
\def\iFrameRate{20}	%%
\def\iFrames{30}	%% d : duration
\def\rFromX{1}	%% b : begin point
\def\rToX{e}	%%   : end point
\def\rFromY{14}	%% b : begin point
\def\rToY{0}	%%   : end point
\pgfmathsetmacro\rDeltaX{\rToX-\rFromX}  %% c :  end-begin, total change
\pgfmathsetmacro\rDeltaY{\rToY-\rFromY}  %% c :  end-begin, total change

\def\r{8pt} %% circle radius

%%%%%%%%%%%%%%%%%%%%%%%%%%%%%%%%%%%%%%%%%%%%%%%%%%%%%%%%%%%%%%%%
%%  SWEEP e
%%%%%%%%%%%%%%%%%%%%%%%%%%%%%%%%%%%%%%%%%%%%%%%%%%%%%%%%%%%%%%%%
\begin{figure}[ht]
  \center
  \caption{sweep e}
  \hspace*{-3cm}

\begin{animateinline}[poster=first%
    %% ,loop
    ,controls%
  ]{\iFrameRate}%

  %%%%  INITIAL FRAME REQUIRED in order to get n+1 frames (good starts and ends)

  %% \begin{tikzpicture}
  %%   \useasboundingbox[draw] {(-1,-2) rectangle (10,16)};

  %% \end{tikzpicture}
  %% \newframe

  %%%% MULTIFRAMES START FROM 1 NOT 0
  \multiframe{\iFrames}{iFrame=1+1}{%
    \message{\iFrame}

    \begin{tikzpicture}[scale=3,domain=-.3:1.3]
      %% \useasboundingbox[draw] {(-1,-2) rectangle (10,16)};

%%%%%%%%%%%%%%%%%%%%%%%%%%%%%%%%%%%%%%%%%%%%%%%%%%%%%%%%%%%%%%%%


  %%%%%%%%%%%%%%%%
  %% % 1.  hyperbolic rectilinear rotation
  %% \draw[-latex,red] (1,1) -- (e,1);
  %% \draw[-latex,red] (e,1) -- (e,1/e);

  %% f(x) = 1/x
  \draw[<->,domain=.7:3.2,color=green!50!black] plot (\x,{(1/\x)});
  \draw[color=green!50!black] (.8,1.2) node[left] {$f(x) = \frac{1}{x}$};

  %% sweep animation
  %% \draw[-latex,thick,red,domain=1:e] plot (\x,{(1/\x)});
  \pgfmathsetmacro\x{easeInQuadratic(\iFrame,\rFromX,\rDeltaX,\iFrames)};
  \draw[thick,red,domain=1:\x] plot (\x,{(1/\x)}) coordinate (endpt) {};
  %% \draw (0,{1/\x}) -- (endpt);

  %% unit square
  \begin{scope}[domain=1:\e, even odd rule]
    \draw[ultra thick,black,domain=1:e] (0,0) -- (0,{1/\x}) -- (endpt) -- (\x,0) -- cycle;
    \clip[domain=1:\x] (0,0) -- (0,{1/\x}) -- (endpt) -- (\x,0) -- cycle;
    \draw[pattern=horizontal lines light blue] (0,0) rectangle (e,1);
  \end{scope}

  %% e area
  \begin{scope}[domain=1:\e, even odd rule]
    \clip[domain=1:\x] (1,0) --  plot (\x,{(1/\x)}) -- (\x,0) -- cycle;
    \draw[pattern=vertical lines,pattern color=blue] (0,0) rectangle (e,1);
  \end{scope}

  %% hyperbolic axis sweep
  \begin{scope}[domain=1:\e, even odd rule]
    \clip[domain=1:\x] (0,0) --  plot (\x,{(1/\x)}) -- cycle;
    \fill[color=green!50!blue,opacity=.15] (0,0) rectangle (e,1);
  \end{scope}

  \draw[very thick] (1,0) -- (1,1);
  \draw[semithick] (0,0) -- (endpt);

 %%%%%%%%%%%%%%%%

 %% sweeping vector
%%  \draw[-latex,black,postaction={decorate,
%%                     decoration={markings,
%%                       mark=at position 0.74 with {\arrow{latex}},
%%                       mark=at position 0.82 with {\arrow{latex}};}
%%                    }] (0,0) -- (1.8, 1);
%% \draw [->] (1.8,.975) -- (1.85,.975);

%%%%%%%%%%%%%%%%

 \draw[very thin,color=gray] (-1,0) grid (3.2,1.2);

  %% some ref lines
  \draw[gray] (1/\sqrttwo,0) node[color=black, below] {$\frac{1}{\sqrt{2}}$}
  -- (1/\sqrttwo,1/\sqrttwo);
  \draw[gray] (0,1/\sqrttwo) -- (1/\sqrttwo,1/\sqrttwo);

  %%%%%%%%%%%%%%%%
  % shade hyperbolic and circle sectors
 %%  \begin{scope}[domain=1:\e, even odd rule]
 %%    \clip (0,0) -- (e,1/e) -- (1,1) -- cycle;
 %%    \fill[color=blue,opacity=.15] (0,0) circle(1);

 %%    \clip (0,0) circle(1) (0,0)--(1,1) -- plot (\x,{(1/\x)}) -- (\e,\inve) -- cycle;
 %%    \fill[opacity=.15] (0,0)--(1,1) -- plot (\x,{(1/\x)}) -- (\e,\inve) -- cycle;
 %% \end{scope}

 %% axis:  f(x) = x
  \draw[<->,color=gray] (-1,-1) -- (1.2,1.2) node[right] {$f(x) =x$};
  \draw[black] (0,0) -- (1,1);

  %% e diameter
  \draw[black] (0,0) -- (e,1/e);

%%%%%%%%%%%%%%%%
%% circle stuff
%%  unit circle
  \begin{scope}
    %% \clip (0,0) rectangle (1.1,1.1);
    \draw (0,0) circle (1);
   \end{scope}

 %% ref segments
  \draw[gray] (e,0) -- (e,1/e);

  \draw[gray] (0,\inve) node[left] {$e^{-1}$} -- (\e,\inve);

  \draw[gray] (0,.5) node[left] {$0.5$} -- (\e,.5);

  %% dot origin
  \fill[black] (0,0) node[below left] {$O$} circle (.4pt);

  %% dot (0,1/e)
 \fill[black] (0,\inve) circle (.4pt);

  %% dot (0,1)
  \fill[black] (0,1) node[above left] {$1$} {} circle (.4pt);
  %% dot (1,0)
  \fill[black] (1,0) node[below right] {$1$} circle (.4pt);

  %% dot (1,1)
%  \fill[red] (1,1) node[above right] { $A'$}  circle (.5pt);

 %% label (1,e)
%  \fill[black] (1,\e) node[anchor=south west] {$(1,e)$} circle (.7pt);

 %% dot (e,0)
  \fill[black] (\e,0) node[below] {$e$} circle (.4pt) -- (\e,1.2);

  %% %% dot (e,1)
  %% \fill[red] (\e,1) node[above right,color=red]  {$B'$} circle (.4pt); 

  %% %% dot (e,1/e)
  %% \fill[red] (\e, \inve) node[right] {$C'$} circle (.4pt);

    \end{tikzpicture}
  }% multiframe
\end{animateinline}

\end{figure}


%%% Local Variables: 
%%% mode: latex
%%% TeX-master: t
%%% End: 
