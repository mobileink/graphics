%% tikzies - tikz easing library
\def\iFrameRate{20}	%%
\def\iFrames{30}	%% d : duration
\def\rFromXa{1}	%% b : begin point
\def\rToXa{e}	%%   : end point
\def\rFromXb{e}	%% b : begin point
\pgfmathsetmacro\rToXb{e^2} %%   : end point
\def\rFromY{14}	%% b : begin point
\def\rToY{0}	%%   : end point
\pgfmathsetmacro\rDeltaXa{\rToXa-\rFromXa}  %% c :  end-begin, total change
\pgfmathsetmacro\rDeltaXb{\rToXb-\rFromXb}  %% c :  end-begin, total change
\pgfmathsetmacro\rDeltaY{\rToY-\rFromY}  %% c :  end-begin, total change

\def\r{8pt} %% circle radius

%%%%%%%%%%%%%%%%%%%%%%%%%%%%%%%%%%%%%%%%%%%%%%%%%%%%%%%%%%%%%%%%
%%  SWEEP e
%%%%%%%%%%%%%%%%%%%%%%%%%%%%%%%%%%%%%%%%%%%%%%%%%%%%%%%%%%%%%%%%
\begin{figure}[ht]
  \centering
  \caption{Hyperbolic Rotation 3}
  \hspace*{-3cm}
  \begin{animateinline}[poster=first%
      %% ,loop
      ,controls%
    ]{\iFrameRate}%
    %%%%  INITIAL FRAME REQUIRED in order to get n+1 frames (good starts and ends)
    %% \begin{tikzpicture}
    %%   \useasboundingbox[draw] {(-1,-2) rectangle (10,16)};
    %% \end{tikzpicture}
    %% \newframe
    %%%% MULTIFRAMES START FROM 1 NOT 0
    \multiframe{\iFrames}{iFrame=1+1}{%
      \message{\iFrame}

      \begin{tikzpicture}[scale=2,domain=-.3:7.4]
        \useasboundingbox {(-1,-1) rectangle (e^2+.2,2.2)};

        %%%%%%%%%%%%%%%%%%%%%%%%%%%%%%%%%%%%%%%%%%%%%%%%%%%%%%%%%%%%%%%%

        %% grid
        \draw[very thin,gray!50] (0,0) grid (e^2+.2,2.2);
        %% axes
        \draw[<->,black] (-.2,0) -- (e^2+.2,0);
        \draw[<->,black] (0,-.2) -- (0,2.2);

        %% ref segments
        \draw[gray] (0,.5) node[left] {} -- (.1,.5);
        %% dot (0,0.5)
        %% \fill[gray] (0,.5) node[left] {$0.5$} circle (.4pt);

        %% dot origin
        %% \fill[black] (0,0) node[below left] {$0$} circle (.4pt);

        %% dot (0,1/e)
        %% \fill[black] (0,\inve) circle (.4pt);

        %% log scale
        \fill[black] (0,0) node[below,right] {$logarithmic:$};
        \fill[black] (1,0) node[below] {$0$};
        \fill[black] (e,0) node[below] {$1$};
        \draw[black] (e,0) -- (e,2.2);
        \fill[black] (e^2,0) node[below] {$2$};
        \draw[black] (e^2,0) -- (e^2,2.2);

        \fill[gray] (-.6,-.4) node {$arithmetic:$};
        \fill[gray] (0,-.4) node {$0$};
        \fill[gray] (1,-.4) node {$1$};
        \fill[gray] (2,-.4) node {$2$};
        \fill[gray] (e,-.4) node {$e^1$};
        \fill[gray] (3,-.4) node {$3$};
        \fill[gray] (4,-.4) node {$4$};
        \fill[gray] (5,-.4) node {$5$};
        \fill[gray] (6,-.4) node {$6$};
        \fill[gray] (7,-.4) node {$7$};
        \fill[gray] (e^2,-.4) node {$e^2$};


        %%%%%%%%%%%%%%%%
        %% % 1.  hyperbolic rectilinear rotation
        %% \draw[-latex,red] (1,1) -- (e,1);
        %% \draw[-latex,red] (e,1) -- (e,1/e);

        %% hyperbolic rotation animation
        \begin{scope}[even odd rule]
          \pgfmathsetmacro\xa{easeInQuadratic(\iFrame,\rFromXa,\rDeltaXa,\iFrames)};
          \pgfmathsetmacro\xb{easeInQuadratic(\iFrame,\rFromXb,\rDeltaXb,\iFrames)};
          %% shade unit square
          \fill[gray,opacity=.2] (0,0) rectangle (1,1);
          %% shade unit rectangle
          \draw[pattern=horizontal lines light blue] (0,0) rectangle (\xb,1/\xb);
          \draw (0,1/\xb) node[left] {$e^n$} -- (\xb,1/\xb);
          %% shade rotated region - source
          \begin{scope}
            \clip[samples=100,domain=\xa:\xb] (\xa,0) -- plot (\x,{(1/\x)}) -- (\xb,0) -- cycle ;
            \fill[green,opacity=.25] (0,0) rectangle (e^2,1);
          \end{scope}
          %% shade rotated region - destination
          \begin{scope}
            \pgfmathsetmacro\samps{1-\iFrame/\iFrames}
            \clip[samples=100*\samps,domain=\xb:\rToXb*.9]
            	(\xb,0) -- plot (\x,{(1/\x)}) -- (e^2,1/\e^2) -- (e^2,0) -- cycle;
            	%% (\xb,0) -- (\xb,{(1/\xb)}) -- (\rToXb,1/\rToXb) -- (\rToXb,0) -- cycle;
            \draw[opacity=.5,pattern=crosshatch dots light steel blue] (0,0) rectangle (e^2,1);
          \end{scope}
          %% f(x) = 1/x
          \draw[>->,samples=100,domain=.4:e^2+.2,color=green!50!black] plot (\x,{(1/\x)}) coordinate (esq);
          \draw[color=green!50!black] (esq) node[above left] {$f(x) = \frac{1}{x}$};
          %% border rotated region
          \draw[green!50!black,very thick,samples=100,domain=\xa:\xb]
          	(\xa,0) -- plot (\x,{(1/\x)}) -- (\xb,0) -- cycle ;
          %% area labels
          \draw (3.5,.7) node[above] (area) {area = $1$};
          %% label of rotated region - source
          \draw[->] (area)
          	.. controls +(down:12pt) and +(up:12pt) .. ({\xa+(\xb-\xa)*.2},{(1/\xa)*.5});
          %% label of unit square
          \draw[->] (area) .. controls +(left:36pt) and +(up:24pt) .. (.5,.5);
          %% label of unit rectangle
          \draw[->] (area) .. controls +(left:36pt) and +(up:24pt) .. (\xa/2,{(1/\xa)*.2});
          %% label of rotated region - destination
          \draw[->,opacity=(1-\iFrame/\iFrames)] (area)
          	.. controls +(right:12pt) and +(up:12pt) .. (4.5,.1);
        \end{scope}


        %% hyperbolic sweep animation
        %% \draw[-latex,thick,red,domain=1:e] plot (\x,{(1/\x)});
        %% \pgfmathsetmacro\x{easeInQuadratic(\iFrame,\rFromX,\rDeltaX,\iFrames)};
        %% \draw[thick,red,domain=1:\x] plot (\x,{(1/\x)}) coordinate (endpt) {};
        %% \draw (0,{1/\x}) -- (endpt);

        %% unit square
        %% \begin{scope}[domain=1:\e, even odd rule]
        %%   \draw[ultra thick,black,domain=1:e] (0,0) -- (0,{1/\x}) -- (endpt) -- (\x,0) -- cycle;
        %%   \clip[domain=1:\x] (0,0) -- (0,{1/\x}) -- (endpt) -- (\x,0) -- cycle;
        %%   \draw[pattern=horizontal lines light blue] (0,0) rectangle (e,1);
        %% \end{scope}

        %% e area shading
        %% \begin{scope}[domain=1:\e, even odd rule]
        %%   \clip[domain=1:\x] (1,0) --  plot (\x,{(1/\x)}) -- (\x,0) -- cycle;
        %%   \draw[pattern=vertical lines,pattern color=blue] (0,0) rectangle (e,1);
        %% \end{scope}

        %% hyperbolic axis sweep shading
        %% \begin{scope}[domain=1:\e, even odd rule]
        %%   \clip[domain=1:\x] (0,0) --  plot (\x,{(1/\x)}) -- cycle;
        %%   \fill[color=green!50!blue,opacity=.15] (0,0) rectangle (e,1);
        %% \end{scope}

        %% \draw[semithick] (0,0) -- (endpt);

        %%%%%%%%%%%%%%%%

        %% sweeping vector
        %%  \draw[-latex,black,postaction={decorate,
        %%                     decoration={markings,
        %%                       mark=at position 0.74 with {\arrow{latex}},
        %%                       mark=at position 0.82 with {\arrow{latex}};}
        %%                    }] (0,0) -- (1.8, 1);
        %% \draw [->] (1.8,.975) -- (1.85,.975);

        %%%%%%%%%%%%%%%%


        %% dot (1,1)
        %  \fill[red] (1,1) node[above right] { $A'$}  circle (.5pt);

        %% label (1,e)
        %  \fill[black] (1,\e) node[anchor=south west] {$(1,e)$} circle (.7pt);

        %% %% dot (e,1)
        %% \fill[red] (\e,1) node[above right,color=red]  {$B'$} circle (.4pt); 

        %% %% dot (e,1/e)
        %% \fill[red] (\e, \inve) node[right] {$C'$} circle (.4pt);

      \end{tikzpicture}
    }% multiframe
  \end{animateinline}

\end{figure}


%%% Local Variables: 
%%% mode: latex
%%% TeX-master: t
%%% End: 
