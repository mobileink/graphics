%%%%%%%%%%%%%%%%%%%%%%%%%%%%%%%%%%%%%%%%%%%%%%%%%%%%%%%%%%%%%%%%
%% Hyperbolic Rotation - oops. 
%% three-move ``rotation'' - ignore!
%%%%%%%%%%%%%%%%%%%%%%%%%%%%%%%%%%%%%%%%%%%%%%%%%%%%%%%%%%%%%%%%
\begin{animateinline}[%
    poster=first%
    %% ,draft
    %% ,loop
    ,controls%
  ]{\the\value{iFrameRate}}%

  %%%%  INITIAL FRAME REQUIRED in order to get n+1 frames (good starts and ends)

  %% \begin{tikzpicture}
  %%   \useasboundingbox[draw] {(-1,-2) rectangle (10,16)};

  %% \end{tikzpicture}
  %% \newframe

  %%%% MULTIFRAMES START FROM 1 NOT 0
  \multiframe{\value{iFrames}}{iFrame=1+1}{%
    %% \message{\arabic{iFrames}}
    %% \message{\iFrame}
    %% \message{\rFromXa}
    %% \message{\rToXa}
    %% \message{\rFromXb}
    %% \message{\rToXb}

    \begin{tikzpicture}[%
        scale=\tikzscale
        ,domain=-.3:7.4%
      ]
      %% \if usebb
      \useasboundingbox[\bbdraw]{(\bbminx,\bbminy) rectangle (\bbmaxx,\bbmaxy)};
      %% \fi

      \pgfmathsetmacro\xa{easeInQuadratic(\iFrame,\rFromXa,\rDeltaXa,\value{iFrames})};
      \pgfmathsetmacro\xb{easeInQuadratic(\iFrame,\rFromXb,\rDeltaXb,\value{iFrames})};

      %% grid
      \draw[very thin,gray!50] (0,0) grid (e^2+.2,7);
      %% axes
      \draw[<->,black] (-.2,0) -- (e^2+.2,0);
      \draw[<->,black] (0,-.2) -- (0,2.2);

      %% ref segments
      \draw[gray] (0,.5) node[left] {} -- (.1,.5);
      %% dot (0,0.5)
      %% \fill[gray] (0,.5) node[left] {$0.5$} circle (.4pt);

      %% dot origin
      %% \fill[black] (0,0) node[below left] {$0$} circle (.4pt);

      %% dot (0,1/e)
      %% \fill[black] (0,\inve) circle (.4pt);

      %% log scale
      \fill[black] (-1,0) node[below] {$logarithmic:$};
      \fill[black] (1,0) node[below] {$0$};
      \fill[black] (e,0) node[below] {$1$};
      \draw[black] (e,0) -- (e,2.2);
      \fill[black] (e^2,0) node[below] {$2$};
      \draw[black] (e^2,0) -- (e^2,2.2);

      \fill[gray] (-1,-.5) node {$arithmetic:$};
      \fill[gray] (0,-.5) node {$0$};
      \fill[gray] (1,-.5) node {$1$};
      \fill[gray] (2,-.5) node {$2$};
      \fill[gray] (e,-.5) node {$e^1$};
      \fill[gray] (3,-.5) node {$3$};
      \fill[gray] (4,-.5) node {$4$};
      \fill[gray] (5,-.5) node {$5$};
      \fill[gray] (6,-.5) node {$6$};
      \fill[gray] (7,-.5) node {$7$};
      \fill[gray] (e^2,-.5) node {$e^2$};

      %% v axis labels
      \fill[->,gray] (0,e) node {$e^1$} -- (e^2+.2,e);

      %%%%%%%%%%%%%%%%
      %% % 1.  hyperbolic rectilinear rotation
      %% \draw[-latex,red] (1,1) -- (e,1);
      %% \draw[-latex,red] (e,1) -- (e,1/e);

      %% hyperbolic rotation animation
      \begin{scope}[even odd rule]
        %% %% shade unit square
        %% \fill[gray,opacity=.2] (0,0) rectangle (1,1);
        %% %% coordinate rectangle
        %% \draw[pattern=horizontal lines light blue] (0,0) rectangle (\xb,1/\xb);
        %% \draw (0,1/\xb) node[left] {$e^n$} -- (\xb,1/\xb);

        %% shade rotated region - source
        %% \begin{scope}
        %%   \clip[samples=100,domain=1:e] (1,0) -- plot (\x,1/\x+e*(1/\x)) -- (\xb,0) -- cycle ;
        %%   \fill[green,opacity=.25] (0,0) rectangle (e^2,1);
        %% \end{scope}

        %% \draw[red,samples=100,domain=1:e] plot (\x,{1/\x+(e-1/\x)*\xa}); % (e-1/\x)*pct

        \pgfmathsetmacro\pct{\iFrame/\value{iFrames}}

        %% \draw[red,dashed,samples=100,domain=1:e] plot (\x,1/\x);

        %% \draw[red,dashed,samples=100,domain=1:e] plot (\x,{1+1/\x+((e-1)/\x)});

        %%%%%%%%%%%%%%%%
        \draw[<->,green!10!black,samples=100,domain=.5:e^2+.2]
        plot (\x,{1/\x+\pct*(1+(e-1)/\x)});

        %% lower left to upper left

        \draw[green!50!black,ultra thick,samples=100,domain=1:e]
        (1,\pct) -- plot (\x,{1/\x+\pct*(1+(e-1)/\x)})
        -- (e,\pct) -- cycle;

        %% %% shade rotated region - destination
        %% \begin{scope}
        %%   \pgfmathsetmacro\samps{1-\iFrame/\value{iFrames}}
        %%   \clip[samples=100*\samps,domain=\xb:\rToXb*.9]
        %%   (\xb,0) -- plot (\x,{(1/\x)}) -- (e^2,1/e^2) -- (e^2,0) -- cycle;
        %%   %% (\xb,0) -- (\xb,{(1/\xb)}) -- (\rToXb,1/\rToXb) -- (\rToXb,0) -- cycle;
        %%   \draw[opacity=.5,pattern=crosshatch dots light steel blue] (0,0) rectangle (e^2,1);
        %% \end{scope}
        %% %% f(x) = 1/x
        %% \draw[>->,samples=100,domain=.4:e^2+.2,color=green!50!black] plot (\x,{(1/\x)});
        %% \draw[color=green!50!black] ({((e^2)-.5)},{1/((e^2)-.5)}) node[above left] {$f(x) = \frac{1}{x}$};
        %% %% border rotated region
        %% \draw[green!50!black,very thick,samples=100,domain=\xa:\xb]
        %% (\xa,0) -- plot (\x,{(1/\x)}) -- (\xb,0) -- cycle ;
        %% %% area labels
        %% \draw (3.5,.7) node[above] (area) {area = $1$};
        %% %% label of rotated region - source
        %% \draw[->] (area)
        %% .. controls +(down:12pt) and +(up:12pt) .. ({\xa+(\xb-\xa)*.2},{(1/\xa)*.5});
        %% %% label of unit square
        %% \draw[->] (area) .. controls +(left:36pt) and +(up:24pt) .. (.5,.5);
        %% %% label of unit rectangle
        %% \draw[->] (area) .. controls +(left:36pt) and +(up:24pt) .. (\xa/2,{(1/\xa)*.2});
        %% %% label of rotated region - destination
        %% \draw[->,opacity=(1-\iFrame/\value{iFrames})] (area)
        %% .. controls +(right:12pt) and +(up:12pt) .. (4.5,.1);
      \end{scope}
    \end{tikzpicture}
  }% multiframe

  \newframe*
  %%%%%%%%%%%%%%%%%%%%%%%%%%%%%%%%%%%%%%%%%%%%%%%%%%%%%%%%%%%%%%%%
  %% STEP 2:  upper left to upper right
  \multiframe{\value{iFrames}}{iFrame=1+1}{%
    %% \message{\arabic{iFrames}}
    %% \message{\iFrame}
    %% \message{\rFromXa}
    %% \message{\rToXa}
    %% \message{\rFromXb}
    %% \message{\rToXb}

    \begin{tikzpicture}[%
        scale=\tikzscale
        ,domain=-.3:7.4%
      ]
      %% \if usebb
      \useasboundingbox[\bbdraw]{(\bbminx,\bbminy) rectangle (\bbmaxx,\bbmaxy)};
      %% \fi

      \pgfmathsetmacro\xa{easeInQuadratic(\iFrame,\rFromXa,\rDeltaXa,\value{iFrames})};
      \pgfmathsetmacro\xb{easeInQuadratic(\iFrame,\rFromXb,\rDeltaXb,\value{iFrames})};

      %% grid
      \draw[very thin,gray!50] (0,0) grid (e^2+.2,7);
      %% axes
      \draw[<->,black] (-.2,0) -- (e^2+.2,0);
      \draw[<->,black] (0,-.2) -- (0,2.2);

      %% ref segments
      \draw[gray] (0,.5) node[left] {} -- (.1,.5);
      %% dot (0,0.5)
      %% \fill[gray] (0,.5) node[left] {$0.5$} circle (.4pt);

      %% dot origin
      %% \fill[black] (0,0) node[below left] {$0$} circle (.4pt);

      %% dot (0,1/e)
      %% \fill[black] (0,\inve) circle (.4pt);

      %% log scale
      \fill[black] (-1,0) node[below] {$logarithmic:$};
      \fill[black] (1,0) node[below] {$0$};
      \fill[black] (e,0) node[below] {$1$};
      \draw[black] (e,0) -- (e,2.2);
      \fill[black] (e^2,0) node[below] {$2$};
      \draw[black] (e^2,0) -- (e^2,2.2);

      \fill[gray] (-1,-.5) node {$arithmetic:$};
      \fill[gray] (0,-.5) node {$0$};
      \fill[gray] (1,-.5) node {$1$};
      \fill[gray] (2,-.5) node {$2$};
      \fill[gray] (e,-.5) node {$e^1$};
      \fill[gray] (3,-.5) node {$3$};
      \fill[gray] (4,-.5) node {$4$};
      \fill[gray] (5,-.5) node {$5$};
      \fill[gray] (6,-.5) node {$6$};
      \fill[gray] (7,-.5) node {$7$};
      \fill[gray] (e^2,-.5) node {$e^2$};

      %% v axis labels
      \fill[->,gray] (0,e) node {$e^1$} -- (e^2+.2,e);

      %%%%%%%%%%%%%%%%
      %% % 1.  hyperbolic rectilinear rotation
      %% \draw[-latex,red] (1,1) -- (e,1);
      %% \draw[-latex,red] (e,1) -- (e,1/e);

      %% hyperbolic rotation animation
      \begin{scope}[even odd rule]
        %% %% shade unit square
        %% \fill[gray,opacity=.2] (0,0) rectangle (1,1);
        %% %% coordinate rectangle
        %% \draw[pattern=horizontal lines light blue] (0,0) rectangle (\xb,1/\xb);
        %% \draw (0,1/\xb) node[left] {$e^n$} -- (\xb,1/\xb);

        %% shade rotated region - source
        %% \begin{scope}
        %%   \clip[samples=100,domain=1:e] (1,0) -- plot (\x,1/\x+e*(1/\x)) -- (\xb,0) -- cycle ;
        %%   \fill[green,opacity=.25] (0,0) rectangle (e^2,1);
        %% \end{scope}

        %% \draw[red,samples=100,domain=1:e] plot (\x,{1/\x+(e-1/\x)*\xa}); % (e-1/\x)*pct

        \pgfmathsetmacro\pct{\iFrame/\value{iFrames}}

        %% \draw[red,dashed,samples=100,domain=1:e] plot (\x,1/\x);

        %% \draw[red,dashed,samples=100,domain=1:e] plot (\x,{1+1/\x+((e-1)/\x)});

        \draw[<->,green!10!black,samples=100,domain=.5:e^2+.2]
        	plot (\x,{1/\x+(1+(e-1)/\x)});

                %% upper left to upper right
        \draw[green!50!black,ultra thick,samples=100,domain=\xa:\xb]
        	(\xa,1) -- plot (\x,{1/\x+(1+(e-1)/\x)})
        	-- (\xb,1) -- cycle;

        %% %% shade rotated region - destination
        %% \begin{scope}
        %%   \pgfmathsetmacro\samps{1-\iFrame/\value{iFrames}}
        %%   \clip[samples=100*\samps,domain=\xb:\rToXb*.9]
        %%   (\xb,0) -- plot (\x,{(1/\x)}) -- (e^2,1/e^2) -- (e^2,0) -- cycle;
        %%   %% (\xb,0) -- (\xb,{(1/\xb)}) -- (\rToXb,1/\rToXb) -- (\rToXb,0) -- cycle;
        %%   \draw[opacity=.5,pattern=crosshatch dots light steel blue] (0,0) rectangle (e^2,1);
        %% \end{scope}
        %% %% f(x) = 1/x
        %% \draw[>->,samples=100,domain=.4:e^2+.2,color=green!50!black] plot (\x,{(1/\x)});
        %% \draw[color=green!50!black] ({((e^2)-.5)},{1/((e^2)-.5)}) node[above left] {$f(x) = \frac{1}{x}$};
        %% %% border rotated region
        %% \draw[green!50!black,very thick,samples=100,domain=\xa:\xb]
        %% (\xa,0) -- plot (\x,{(1/\x)}) -- (\xb,0) -- cycle ;
        %% %% area labels
        %% \draw (3.5,.7) node[above] (area) {area = $1$};
        %% %% label of rotated region - source
        %% \draw[->] (area)
        %% .. controls +(down:12pt) and +(up:12pt) .. ({\xa+(\xb-\xa)*.2},{(1/\xa)*.5});
        %% %% label of unit square
        %% \draw[->] (area) .. controls +(left:36pt) and +(up:24pt) .. (.5,.5);
        %% %% label of unit rectangle
        %% \draw[->] (area) .. controls +(left:36pt) and +(up:24pt) .. (\xa/2,{(1/\xa)*.2});
        %% %% label of rotated region - destination
        %% \draw[->,opacity=(1-\iFrame/\value{iFrames})] (area)
        %% .. controls +(right:12pt) and +(up:12pt) .. (4.5,.1);
      \end{scope}
    \end{tikzpicture}
  }% multiframe

  \newframe*
  %%%%%%%%%%%%%%%%%%%%%%%%%%%%%%%%%%%%%%%%%%%%%%%%%%%%%%%%%%%%%%%%
  %% STEP 3:  upper right to lower right
  %%%%%%%%%%%%%%%%%%%%%%%%%%%%%%%%%%%%%%%%%%%%%%%%%%%%%%%%%%%%%%%%
  \multiframe{\value{iFrames}}{iFrame=1+1}{%
    %% \message{\arabic{iFrames}}
    %% \message{\iFrame}
    %% \message{\rFromXa}
    %% \message{\rToXa}
    %% \message{\rFromXb}
    %% \message{\rToXb}

    \begin{tikzpicture}[%
        scale=\tikzscale
        ,domain=-.3:7.4%
      ]
      %% \if usebb
      \useasboundingbox[\bbdraw]{(\bbminx,\bbminy) rectangle (\bbmaxx,\bbmaxy)};
      %% \fi

      \pgfmathsetmacro\xa{easeInQuadratic(\iFrame,\rFromXa,\rDeltaXa,\value{iFrames})};
      \pgfmathsetmacro\xb{easeInQuadratic(\iFrame,\rFromXb,\rDeltaXb,\value{iFrames})};

      %% grid
      \draw[very thin,gray!50] (0,0) grid (e^2+.2,7);
      %% axes
      \draw[<->,black] (-.2,0) -- (e^2+.2,0);
      \draw[<->,black] (0,-.2) -- (0,2.2);

      %% ref segments
      \draw[gray] (0,.5) node[left] {} -- (.1,.5);
      %% dot (0,0.5)
      %% \fill[gray] (0,.5) node[left] {$0.5$} circle (.4pt);

      %% dot origin
      %% \fill[black] (0,0) node[below left] {$0$} circle (.4pt);

      %% dot (0,1/e)
      %% \fill[black] (0,\inve) circle (.4pt);

      %% log scale
      \fill[black] (-1,0) node[below] {$logarithmic:$};
      \fill[black] (1,0) node[below] {$0$};
      \fill[black] (e,0) node[below] {$1$};
      \draw[black] (e,0) -- (e,2.2);
      \fill[black] (e^2,0) node[below] {$2$};
      \draw[black] (e^2,0) -- (e^2,2.2);

      \fill[gray] (-1,-.5) node {$arithmetic:$};
      \fill[gray] (0,-.5) node {$0$};
      \fill[gray] (1,-.5) node {$1$};
      \fill[gray] (2,-.5) node {$2$};
      \fill[gray] (e,-.5) node {$e^1$};
      \fill[gray] (3,-.5) node {$3$};
      \fill[gray] (4,-.5) node {$4$};
      \fill[gray] (5,-.5) node {$5$};
      \fill[gray] (6,-.5) node {$6$};
      \fill[gray] (7,-.5) node {$7$};
      \fill[gray] (e^2,-.5) node {$e^2$};

      %% v axis labels
      \fill[->,gray] (0,e) node {$e^1$} -- (e^2+.2,e);

      %%%%%%%%%%%%%%%%
      %% % 1.  hyperbolic rectilinear rotation
      %% \draw[-latex,red] (1,1) -- (e,1);
      %% \draw[-latex,red] (e,1) -- (e,1/e);

      %% hyperbolic rotation animation
      \begin{scope}[even odd rule]
        %% %% shade unit square
        %% \fill[gray,opacity=.2] (0,0) rectangle (1,1);
        %% %% coordinate rectangle
        %% \draw[pattern=horizontal lines light blue] (0,0) rectangle (\xb,1/\xb);
        %% \draw (0,1/\xb) node[left] {$e^n$} -- (\xb,1/\xb);

        %% shade rotated region - source
        %% \begin{scope}
        %%   \clip[samples=100,domain=1:e] (1,0) -- plot (\x,1/\x+e*(1/\x)) -- (\xb,0) -- cycle ;
        %%   \fill[green,opacity=.25] (0,0) rectangle (e^2,1);
        %% \end{scope}

        %% \draw[red,samples=100,domain=1:e] plot (\x,{1/\x+(e-1/\x)*\xa}); % (e-1/\x)*pct

        \pgfmathsetmacro\pct{\iFrame/\value{iFrames}}

        %% \draw[red,dashed,samples=100,domain=1:e] plot (\x,1/\x);

        %% \draw[red,dashed,samples=100,domain=1:e] plot (\x,{1+1/\x+((e-1)/\x)});

        \draw[<->,green!10!black,samples=100,domain=.5:e^2+.2]
        	plot (\x,{1/\x+(1-\pct+(e-1)/\x)});

%% upper right to lower left
        \draw[green!50!black,ultra thick,samples=100,domain=e:e^2]
        	(e,1-\pct) -- plot (\x,{1/\x+(1-\pct+(e-1)/\x)})
        	-- (e^2,1-\pct) -- cycle;

        %% %% shade rotated region - destination
        %% \begin{scope}
        %%   \pgfmathsetmacro\samps{1-\iFrame/\value{iFrames}}
        %%   \clip[samples=100*\samps,domain=\xb:\rToXb*.9]
        %%   (\xb,0) -- plot (\x,{(1/\x)}) -- (e^2,1/e^2) -- (e^2,0) -- cycle;
        %%   %% (\xb,0) -- (\xb,{(1/\xb)}) -- (\rToXb,1/\rToXb) -- (\rToXb,0) -- cycle;
        %%   \draw[opacity=.5,pattern=crosshatch dots light steel blue] (0,0) rectangle (e^2,1);
        %% \end{scope}
        %% %% f(x) = 1/x
        %% \draw[>->,samples=100,domain=.4:e^2+.2,color=green!50!black] plot (\x,{(1/\x)});
        %% \draw[color=green!50!black] ({((e^2)-.5)},{1/((e^2)-.5)}) node[above left] {$f(x) = \frac{1}{x}$};
        %% %% border rotated region
        %% \draw[green!50!black,very thick,samples=100,domain=\xa:\xb]
        %% (\xa,0) -- plot (\x,{(1/\x)}) -- (\xb,0) -- cycle ;
        %% %% area labels
        %% \draw (3.5,.7) node[above] (area) {area = $1$};
        %% %% label of rotated region - source
        %% \draw[->] (area)
        %% .. controls +(down:12pt) and +(up:12pt) .. ({\xa+(\xb-\xa)*.2},{(1/\xa)*.5});
        %% %% label of unit square
        %% \draw[->] (area) .. controls +(left:36pt) and +(up:24pt) .. (.5,.5);
        %% %% label of unit rectangle
        %% \draw[->] (area) .. controls +(left:36pt) and +(up:24pt) .. (\xa/2,{(1/\xa)*.2});
        %% %% label of rotated region - destination
        %% \draw[->,opacity=(1-\iFrame/\value{iFrames})] (area)
        %% .. controls +(right:12pt) and +(up:12pt) .. (4.5,.1);
      \end{scope}
    \end{tikzpicture}
  }% multiframe

  \newframe*
  %%%%%%%%%%%%%%%%%%%%%%%%%%%%%%%%%%%%%%%%%%%%%%%%%%%%%%%%%%%%%%%%
  %% STEP 4:  complete the cycle: lower right back to lower left
  %%%%%%%%%%%%%%%%%%%%%%%%%%%%%%%%%%%%%%%%%%%%%%%%%%%%%%%%%%%%%%%%
  \multiframe{\value{iFrames}}{iFrame=1+1}{%
    %% \message{\arabic{iFrames}}
    %% \message{\iFrame}
    %% \message{\rFromXa}
    %% \message{\rToXa}
    %% \message{\rFromXb}
    %% \message{\rToXb}

    \begin{tikzpicture}[%
        scale=\tikzscale
        ,domain=-.3:7.4%
      ]
      %% \if usebb
      \useasboundingbox[\bbdraw]{(\bbminx,\bbminy) rectangle (\bbmaxx,\bbmaxy)};
      %% \fi

      \pgfmathsetmacro\xa{easeInQuadratic(\iFrame,\rToXa,-\rDeltaXa,\value{iFrames})};
\message{\xa}
      \pgfmathsetmacro\xb{easeInQuadratic(\iFrame,\rToXb,-\rDeltaXb,\value{iFrames})};
\message{\xb}
      %% \pgfmathsetmacro\xa{easeInQuadratic(\iFrame,\rFromXa,\rDeltaXa,\value{iFrames})};
      %% \pgfmathsetmacro\xb{easeInQuadratic(\iFrame,\rFromXb,\rDeltaXb,\value{iFrames})};

      %% grid
      \draw[very thin,gray!50] (0,0) grid (e^2+.2,7);
      %% axes
      \draw[<->,black] (-.2,0) -- (e^2+.2,0);
      \draw[<->,black] (0,-.2) -- (0,2.2);

      %% ref segments
      \draw[gray] (0,.5) node[left] {} -- (.1,.5);
      %% dot (0,0.5)
      %% \fill[gray] (0,.5) node[left] {$0.5$} circle (.4pt);

      %% dot origin
      %% \fill[black] (0,0) node[below left] {$0$} circle (.4pt);

      %% dot (0,1/e)
      %% \fill[black] (0,\inve) circle (.4pt);

      %% log scale
      \fill[black] (-1,0) node[below] {$logarithmic:$};
      \fill[black] (1,0) node[below] {$0$};
      \fill[black] (e,0) node[below] {$1$};
      \draw[black] (e,0) -- (e,2.2);
      \fill[black] (e^2,0) node[below] {$2$};
      \draw[black] (e^2,0) -- (e^2,2.2);

      \fill[gray] (-1,-.5) node {$arithmetic:$};
      \fill[gray] (0,-.5) node {$0$};
      \fill[gray] (1,-.5) node {$1$};
      \fill[gray] (2,-.5) node {$2$};
      \fill[gray] (e,-.5) node {$e^1$};
      \fill[gray] (3,-.5) node {$3$};
      \fill[gray] (4,-.5) node {$4$};
      \fill[gray] (5,-.5) node {$5$};
      \fill[gray] (6,-.5) node {$6$};
      \fill[gray] (7,-.5) node {$7$};
      \fill[gray] (e^2,-.5) node {$e^2$};

      %% v axis labels
      \fill[->,gray] (0,e) node {$e^1$} -- (e^2+.2,e);

      %%%%%%%%%%%%%%%%
      %% % 1.  hyperbolic rectilinear rotation
      %% \draw[-latex,red] (1,1) -- (e,1);
      %% \draw[-latex,red] (e,1) -- (e,1/e);

      %% hyperbolic rotation animation
      \begin{scope}[even odd rule]
        %% %% shade unit square
        %% \fill[gray,opacity=.2] (0,0) rectangle (1,1);
        %% %% coordinate rectangle
        %% \draw[pattern=horizontal lines light blue] (0,0) rectangle (\xb,1/\xb);
        %% \draw (0,1/\xb) node[left] {$e^n$} -- (\xb,1/\xb);

        %% shade rotated region - source
        %% \begin{scope}
        %%   \clip[samples=100,domain=1:e] (1,0) -- plot (\x,1/\x+e*(1/\x)) -- (\xb,0) -- cycle ;
        %%   \fill[green,opacity=.25] (0,0) rectangle (e^2,1);
        %% \end{scope}

        %% \draw[red,samples=100,domain=1:e] plot (\x,{1/\x+(e-1/\x)*\xa}); % (e-1/\x)*pct

        \pgfmathsetmacro\pct{\iFrame/\value{iFrames}}

        %% \draw[red,dashed,samples=100,domain=1:e] plot (\x,1/\x);

        %% \draw[red,dashed,samples=100,domain=1:e] plot (\x,{1+1/\x+((e-1)/\x)});

        \draw[<->,green!10!black,samples=100,domain=.5:e^2+.2]
        plot (\x,{1/\x+(e-1)/\x});

        %% lower right to lower left
        \draw[green!50!black,ultra thick,samples=100,domain=\xa:\xb]
        (\xa,0) -- plot (\x,1/\x) -- (\xb,0) -- cycle;

        %% (e,0) -- plot (\x,{1/\x+\pct*(1+(e-1)/\x)})

        %% %% lower left to upper left

        %% \draw[green!50!black,ultra thick,samples=100,domain=1:e]
        %% (1,\pct) -- plot (\x,{1/\x+\pct*(1+(e-1)/\x)})
        %% -- (e,\pct) -- cycle;

        %% %% shade rotated region - destination
        %% \begin{scope}
        %%   \pgfmathsetmacro\samps{1-\iFrame/\value{iFrames}}
        %%   \clip[samples=100*\samps,domain=\xb:\rToXb*.9]
        %%   (\xb,0) -- plot (\x,{(1/\x)}) -- (e^2,1/e^2) -- (e^2,0) -- cycle;
        %%   %% (\xb,0) -- (\xb,{(1/\xb)}) -- (\rToXb,1/\rToXb) -- (\rToXb,0) -- cycle;
        %%   \draw[opacity=.5,pattern=crosshatch dots light steel blue] (0,0) rectangle (e^2,1);
        %% \end{scope}
        %% %% f(x) = 1/x
        %% \draw[>->,samples=100,domain=.4:e^2+.2,color=green!50!black] plot (\x,{(1/\x)});
        %% \draw[color=green!50!black] ({((e^2)-.5)},{1/((e^2)-.5)}) node[above left] {$f(x) = \frac{1}{x}$};
        %% %% border rotated region
        %% \draw[green!50!black,very thick,samples=100,domain=\xa:\xb]
        %% (\xa,0) -- plot (\x,{(1/\x)}) -- (\xb,0) -- cycle ;
        %% %% area labels
        %% \draw (3.5,.7) node[above] (area) {area = $1$};
        %% %% label of rotated region - source
        %% \draw[->] (area)
        %% .. controls +(down:12pt) and +(up:12pt) .. ({\xa+(\xb-\xa)*.2},{(1/\xa)*.5});
        %% %% label of unit square
        %% \draw[->] (area) .. controls +(left:36pt) and +(up:24pt) .. (.5,.5);
        %% %% label of unit rectangle
        %% \draw[->] (area) .. controls +(left:36pt) and +(up:24pt) .. (\xa/2,{(1/\xa)*.2});
        %% %% label of rotated region - destination
        %% \draw[->,opacity=(1-\iFrame/\value{iFrames})] (area)
        %% .. controls +(right:12pt) and +(up:12pt) .. (4.5,.1);
      \end{scope}
    \end{tikzpicture}
  }% multiframe

\end{animateinline}

%%% Local Variables: 
%%% mode: latex
%%% TeX-master: t
%%% End: 
