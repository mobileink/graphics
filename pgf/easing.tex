\documentclass{article}

\usepackage{fontspec,xltxtra,xunicode}
\usepackage{fontspec}
\defaultfontfeatures{Scale=MatchLowercase}

%% \setmainfont[Mapping=tex-text]{Times New Roman}
%% \setromanfont[Mapping=tex-text]{Times New Roman}
%% \setsansfont[Mapping=tex-text]{Arial}

\setmainfont[Mapping=tex-text]{TeX Gyre Pagella}
%% \setromanfont[Mapping=tex-text]{TeX Gyre Pagella}
%% \setsansfont[Mapping=tex-text]{TeX Gyre Heros}

%% \usepackage{tikz}
\usepackage{pgfplots}
\pgfplotsset{compat=1.9}

\usepgflibrary{fpu}

%% \usetikzlibrary{arrows,shapes,patterns,backgrounds,spy}

%% \usepackage{verbatim}
%% \usepackage[active,tightpage]{preview}
%% \setlength\PreviewBorder{0pt}%
%% %%%>


%% \begin{comment}
%% :Title: Animated Ball

%% \end{comment}

\usepackage{animate}

\usepackage{xifthen}

%%%% macros %%%%

%% see http://gizma.com/easing/
%% see https://github.com/danro/jquery-easing/blob/master/jquery.easing.js

%% args for all easing funcs:  t b c d
%% t : ``time'' elapsed (clock time or frames) since start of anim
%% b : start val
%% c : change in val; b+c = end val
%% d : duration (total ``time'' of change, same units as t)

%% latex conventions:
%% t : iFrame
%% b : rFrom
%% c : rDelta
%% d : iFrames
%% ex: \pgfmathsetmacro\y{easeInQuadratic(\iFrame,\rFrom,\rDelta,\iFrames)};

%% THE ONLY ARG that changes is t.

%% Common computations:
%% 	t/d = percent of the way between b and b+c

%% easeInQuadratic t b c d
\pgfmathdeclarefunction{easeInQuadratic}{4}{%
  \pgfmathparse{#3*(#1/#4)^2+#2} %% c(t/d)^2 + b
}

%% easeInCubic t b c d
\pgfmathdeclarefunction{easeInCubic}{4}{%
  \pgfmathparse{#3*(#1/#4)^3+#2}%
}

%% easeInQuartic t b c d
\pgfmathdeclarefunction{easeInQuartic}{4}{%
  \pgfmathparse{#3*(#1/#4)^4+#2}%
}

%% easeInQuintic t b c d
\pgfmathdeclarefunction{easeInQuintic}{4}{%
  \pgfmathparse{#3*(#1/#4)^5+#2}%
}


\pgfmathsetmacro\halfpi{pi/2}

%% easeInSin t b c d
%% js: return -c * Math.cos(t/d * (Math.PI/2)) + c + b;
\pgfmathdeclarefunction{easeInSin}{4}{%
  \pgfmathparse{ -#3 * cos(#1/#4 * \halfpi r) + #3 + #2}% radians
}

%% easeInExp t b c d
%% js: return (t==0) ? b : c * Math.pow(2, 10 * (t/d - 1)) + b;
%% \pgfkeys{/pgf/fpu,/pgf/fpu/output format=float}
\pgfmathdeclarefunction{easeInExp}{4}{%
  \pgfmathparse{pow(2, 10 * (#1/#4 -1))}
  \pgfmathparse{ #3 * \pgfmathresult + #2}%
}
%% \pgfkeys{/pgf/fpu=false}


%% easeInCirc t b c d
%% js: -c * (Math.sqrt(1 - (t/=d)*t) - 1) + b;
\pgfmathdeclarefunction{easeInCirc}{4}{%
  \pgfmathparse{ -#3 * (sqrt(1 - (#1/#4)^2) - 1) + #2}
}

%% easeInElastic: function (t, b, c, d) {
%% 		var s=1.70158;var p=0;var a=c;
%% 		if (t==0) return b;  if ((t/=d)==1) return b+c;  if (!p) p=d*.3;
%% 		if (a < Math.abs(c)) { a=c; var s=p/4; }
%% 		else var s = p/(2*Math.PI) * Math.asin (c/a);
%% 		return -(a*Math.pow(2,10*(t-=1)) * Math.sin( (t*d-s)*(2*Math.PI)/p )) + b;
%% }
%% easeInElastic t b c d
%% js: -c * (Math.sqrt(1 - (t/=d)*t) - 1) + b;

\newdimen\a
\newdimen\s
\newdimen\p
\newdimen\t
\newdimen\pct
\pgfmathsetlength{\s}{1.70158}
\pgfmathsetlength{\p}{0}

%% \pgfmathdeclarefunction{easeInElastic}{4}{%
%% \pgfmathsetlength{\a}{#3}
%% \pgfmathsetlength{\t}{#1}
%% \pgfmathsetlength{\pct}{#1/#4}
%% \ifdim\t=0pt
%%     \message{b eq 0}
%%     #2 %% b
%% \fi
%% \ifdim\pct=1pt
%% 	\message{t/d eq 1}
%%       \pgfmathparse{#2+#3}\pgfmathresult %% b+c
%% \fi
%% \ifdim\p<0pt %% latex lacks a not op!!
%% 	\message{p lt 0}
%%       \pgfmathsetlength{p}{\p*\d*.3}
%% \else
%%       \ifdim\p>0pt %% latex lacks a not op!!
%% 		\message{p gt 0}
%%               \pgfmathsetlength{p}{\p*\d*.3}
%%       \else
%%         	\message{p eq 0}
%%       \fi
%% \fi

%% }

%% easeInBack: function (t, b, c, d, s) {
%% 		if (s == undefined) s = 1.70158;
%% 		return c*(t/=d)*t*((s+1)*t - s) + b;
\pgfmathdeclarefunction{easeInBack}{4}{%
  \pgfmathsetmacro\s{1.70158}
  \pgfmathsetmacro\dt{#1/#4}
  \pgfmathparse{ #3 * (\dt)^2 * ((\s+1)*\dt - \s) + #2}
}

%% easeOutBack: function (t, b, c, d, s) {
%% if (s == undefined) s = 1.70158;
%% 	return c*((t=t/d-1)*t*((s+1)*t + s) + 1) + b;
\pgfmathdeclarefunction{ezoutback}{4}{%
  \pgfmathsetmacro\s{1.70158}
  \pgfmathsetmacro\dt{#1/#4-1}
  \pgfmathparse{ #3*((\dt)^2*((\s+1)*\dt+\s)+1) + #2}
}

%% easeInOutBack: function (t, b, c, d, s) {
%% 		if (s == undefined) s = 1.70158; 
%% 		if ((t/=d/2) < 1) return c/2*(t*t*(((s*=(1.525))+1)*t - s)) + b;
%% 		return c/2*((t-=2)*t*(((s*=(1.525))+1)*t + s) + 2) + b;
\newdimen\dtp
\pgfmathdeclarefunction{easeinoutback}{4}{%
  \pgfmathsetmacro{\s}{1.70158}
  \pgfmathsetlength{\dtp}{#1/(#4/2)}
%% \message{\the\dtp}
  \pgfmathsetmacro{\dt}{#1/(#4/2)}
  \pgfmathsetmacro{\dtt}{\dt-2)}
%% \message{\dt}
  \ifdim\dtp<1pt
  	\message{dtp < 1}
  	\pgfmathparse{#3/2*(\dt*\dt*(((\s*1.525))+1)*\dt - (\s*1.525)) + #2}
  \else
  	\message{dtp ge 1}
  	\pgfmathparse{#3/2*((\dtt)*\dtt*(((\s*1.525))+1)*\dtt + (\s*1.525) + 2) + #2}
  %% \pgfmathparse{#3*(#1/#4)^5+#2}%
  \fi
}

%% Math.easeOutQuint = function (t, b, c, d) {
%% 	t /= d;
%% 	t--;
%% 	return c*(t*t*t*t*t + 1) + b;
%% };

\pgfmathdeclarefunction{easeoutquint}{4}{%
  \pgfmathparse{#1/#4)-1}%
  \pgfmathsetmacro\t{\pgfmathresult}%
  \pgfmathparse{#3*(\t^5+3)+#2}%
}

%% easeOutBounce: function (x, t, b, c, d) {
%%   if ((t/=d) < (1/2.75)) {
%%     return c*(7.5625*t*t) + b;
%%   } else if (t < (2/2.75)) {
%%     return c*(7.5625*(t-=(1.5/2.75))*t + .75) + b;
%%   } else if (t < (2.5/2.75)) {
%%     return c*(7.5625*(t-=(2.25/2.75))*t + .9375) + b;
%%   } else {
%%     return c*(7.5625*(t-=(2.625/2.75))*t + .984375) + b;
%%   }
%% }

%% to handle floats in ifthenelse:
\newdimen\tdimen
\newdimen\testa
\newdimen\tf

\pgfmathdeclarefunction{easeoutbounce}{4}{%
  \message{">bounce"}
  \pgfmathparse{#1/#4}%
  \pgfmathsetmacro\t{\pgfmathresult}%
  \tdimen = \t pt
  \pgfmathparse{1/2.75}
  \pgfmathsetmacro\ta{\pgfmathresult}%
  \testa = \ta pt
  \message{\ta}
  \pgfmathparse{\t<\ta}
  \pgfmathsetmacro\tb{\pgfmathresult}%
  \tf = \tb pt
  \message{\tb}
  %% \ifthenelse{\tb}{1}{1}
}
%%   \pgfmathparse{1/2.75}
%% \message{\pgfmathresult}
%% %% \message{\t < 1/2.75}

%% {\pgfmathparse{#3*7.5625\t^2+#2}\pgfmathresult}
%%     \ifthenelse{\pgfmathless{#1}{2/2.75}}
%%       {\pgfmathparse{#1-(1.5/2.75)}
%%       \pgfmathsetmacro\t\pgfmathresult
%%       \pgfmathparse{#3*(7.5625*(\t^2 + .75)) + #2}\pgfmathresult}
%%     {%else
%%       \ifthenelse{\pgfmathless{#1}{2.5/2.75}}
%%         {\pgfmathparse{#1-(2.25/2.75)}
%%         \pgfmathsetmacro\t\pgfmathresult
%%         \pgfmathparse{#3*(7.5625*(\t^2 + .9375)) + #2}\pgfmathresult}
%%       {%else
%%         \pgfmathparse{#1-(2.625/2.75)}
%%         \pgfmathsetmacro\t\pgfmathresult
%%         \pgfmathparse{#3*(7.5625*(\t^2 + .984375)) + #2}\pgfmathresul}
%%     }
%%   }
%% }


%% http://sol.gfxile.net/interpolation/
%% for (i = 0; i < N; i++)
%% {
%%   v = i / N;		-  
%%   v = 0.5 - cos(-v * Pi) * 0.5;
%%   X = (A * v) + (B * (1 - v));
%%% here A - beg pt B - end pt
%% }
\pgfmathdeclarefunction{easeOutSin}{4}{%
  % #1 = time elapsed (secs or frame nbr)
  % #2 = beginning val (B)
  % #3 = change (end pt - beg pt)
  % #4 = duration in secs or frames
  \pgfmathparse{#1/#4)}%
  \pgfmathsetmacro\v{\pgfmathresult}%
  \pgfmathparse{0.5-cos(-\v*pi)*0.5}%
  \pgfmathsetmacro\vv{\pgfmathresult}%
  \pgfmathparse{#2*\vv+(#2+#4)*(1-\vv)}%
}

%% v = ((v * (N - 1)) + w) / N; 



%%%%%%%%%%%%%%%%%%%%%%%%%%%%%%%%%%%%%%%%%%%%%%%%%%%%%%%%%%%%%%%%
\begin{document}

\def\x{+0}
%% \def\y{+4}

%% easeInQuadratic t b c d
%% easeInCubic t b c d
%% easeInQuartic t b c d
%% easeInQuintic t b c d
%% easeInSin t b c d

%% easeOutQuint t b c d
%% easeOutSin t b c d
%% easeOutBounce t b c d

%% animate package api conventions:
%% t : iFrame  (``time'' elapsed, in frames)
%% b : rFrom   (beginning value to be eased)
%% c : rDelta  (total change in value; b+c = end value
%% d : iFrames (``duration'' of animation, in frames
%% ex: \pgfmathsetmacro\y{easeInQuad(\iFrame,\rFrom,\rDelta,\iFrames)};

\def\iFrameRate{30}	%%
\def\iFrames{60}	%% d : duration
\def\rFromY{14}	%% b : begin point
\def\rToY{0}	%%   : end point
\pgfmathsetmacro\rDeltaY{\rToY-\rFromY}  %% c :  end-begin, total change

\def\r{8pt} %% circle radius

\begin{animateinline}[poster=first%
    %% ,loop
    ,controls%
  ]{\iFrameRate}%

  %%%%  INITIAL FRAME REQUIRED in order to get n+1 frames (good starts and ends)

  \begin{tikzpicture}
    \useasboundingbox[draw] {(-1,-2) rectangle (10,16)};

    %% \draw (0,\rFromY) -- (5,\rFromY);
    %% \draw (0,\rToY) -- (5,\rToY);

    %% \node[draw] (0,15) {Quadratic};
    %% \pgfmathsetmacro\y{easeInQuadratic(\iFrame,\rFromY,\rDeltaY,\iFrames)};
    \filldraw (0,\rFromY) circle (\r) ;
    \draw (0,15) node[rotate=45] {In$^2$};

    %% \pgfmathsetmacro\y{easeInCubic(\iFrame,\rFromY,\rDeltaY,\iFrames)};
    \filldraw (1,\rFromY) circle (\r) ;
    \draw (1,15) node[rotate=45] {In$^3$};

    %% \pgfmathsetmacro\y{easeInQuartic(\iFrame,\rFromY,\rDeltaY,\iFrames)};
    \filldraw (2,\rFromY) circle (\r) ;
    \draw (2,15) node[rotate=45] {In$^4$};

    %% \pgfmathsetmacro\y{easeInQuintic(\iFrame,\rFromY,\rDeltaY,\iFrames)};
    \filldraw (3,\rFromY) circle (\r) ;
    \draw (3,15) node[rotate=45] {In$^5$};

    %% \pgfmathsetmacro\y{easeInSin(\iFrame,\rFromY,\rDeltaY,\iFrames)};
    \filldraw (4,\rFromY) circle (\r) ;
    \draw (4,15) node[rotate=45] {InSin};

    %% Exp broken - fails with ``Dimension too large''
    %% \pgfmathsetmacro\y{easeInExp(\iFrame,\rFromY,\rDeltaY,\iFrames)};
    %% \filldraw (5,\rFromY) circle (\r) ;
    \draw (5,15) node[rotate=45] {InExp};

    %% \pgfmathsetmacro\y{easeInCirc(\iFrame,\rFromY,\rDeltaY,\iFrames)};
    \filldraw (6,\rFromY) circle (\r) ;
    \draw (6,15) node[rotate=45] {InCirc};

    %% \pgfmathsetmacro\y{easeInElastic(\iFrame,\rFromY,\rDeltaY,\iFrames)};
    %% \message{\y}
    %% %% \filldraw (6,\rFromY) circle (\r) ;
    \draw (7,15) node[rotate=45] {InElastic};

    %% \pgfmathsetmacro\y{easeInBack(\iFrame,\rFromY,\rDeltaY,\iFrames)};
    \filldraw[fill=green] (8,\rFromY) circle (\r) ;
    \draw (8,15) node[rotate=45] {InBack};

    %% \pgfmathsetmacro\y{ezoutback(\iFrame,\rFromY,\rDeltaY,\iFrames)};
    \filldraw[fill=red] (9,\rFromY) circle (\r) ;
    \draw (9,15) node[rotate=45] {OutBack};

  \end{tikzpicture}
  \newframe

  %%%% MULTIFRAMES START FROM 1 NOT 0
  \multiframe{\iFrames}{iFrame=1+1}{%
    \message{\iFrame}

    \begin{tikzpicture}
      \useasboundingbox[draw] {(-1,-2) rectangle (10,16)};

      %% \draw (0,\rFromY) -- (5,\rFromY);
      %% \draw (0,\rToY) -- (5,\rToY);

      %% \node[draw] (0,15) {Quadratic};
      \pgfmathsetmacro\y{easeInQuadratic(\iFrame,\rFromY,\rDeltaY,\iFrames)};
      \filldraw (0,\y) circle (\r) ;
      \draw (0,15) node[rotate=45] {In$^2$};

      \pgfmathsetmacro\y{easeInCubic(\iFrame,\rFromY,\rDeltaY,\iFrames)};
      \filldraw (1,\y) circle (\r) ;
      \draw (1,15) node[rotate=45] {In$^3$};

      \pgfmathsetmacro\y{easeInQuartic(\iFrame,\rFromY,\rDeltaY,\iFrames)};
      \filldraw (2,\y) circle (\r) ;
      \draw (2,15) node[rotate=45] {In$^4$};

      \pgfmathsetmacro\y{easeInQuintic(\iFrame,\rFromY,\rDeltaY,\iFrames)};
      \filldraw (3,\y) circle (\r) ;
      \draw (3,15) node[rotate=45] {In$^5$};

      \pgfmathsetmacro\y{easeInSin(\iFrame,\rFromY,\rDeltaY,\iFrames)};
      \filldraw (4,\y) circle (\r) ;
      \draw (4,15) node[rotate=45] {InSin};

      %% Exp broken - fails with ``Dimension too large''
      %% \pgfmathsetmacro\y{easeInExp(\iFrame,\rFromY,\rDeltaY,\iFrames)};
      %% \filldraw (5,\y) circle (\r) ;
      \draw (5,15) node[rotate=45] {InExp};

      \pgfmathsetmacro\y{easeInCirc(\iFrame,\rFromY,\rDeltaY,\iFrames)};
      \filldraw[fill=blue] (6,\y) circle (\r) ;
      \draw (6,15) node[rotate=45] {InCirc};

      %% \pgfmathsetmacro\y{easeInElastic(\iFrame,\rFromY,\rDeltaY,\iFrames)};
      %% \message{\y}

      \pgfmathsetmacro\y{easeInBack(\iFrame,\rFromY,\rDeltaY,\iFrames)};
      \filldraw (7,\y) circle (\r) ;
      \draw (7,15) node[rotate=45] {InBack};

      \pgfmathsetmacro\y{ezoutback(\iFrame,\rFromY,\rDeltaY,\iFrames)};
      \filldraw[fill=green] (8,\y) circle (\r) ;
      \draw (8,15) node[rotate=45] {OutBack};

      \pgfmathsetmacro\y{easeinoutback(\iFrame,\rFromY,\rDeltaY,\iFrames)};
      \filldraw[fill=red] (9,\y) circle (\r) ;
      \draw (9,15) node[rotate=45] {InOutBack};

    \end{tikzpicture}
  }
\end{animateinline}

\end{document}
