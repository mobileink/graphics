\documentclass{standalone}

%% \usepackage{fontspec,xltxtra,xunicode}
%% \usepackage{fontspec}
%% \defaultfontfeatures{Scale=MatchLowercase}

%% \setmainfont[Mapping=tex-text]{Times New Roman}
%% \setromanfont[Mapping=tex-text]{Times New Roman}
%% \setsansfont[Mapping=tex-text]{Arial}

%% \setmainfont[Mapping=tex-text]{TeX Gyre Pagella}
%% \setromanfont[Mapping=tex-text]{TeX Gyre Pagella}
%% \setsansfont[Mapping=tex-text]{TeX Gyre Heros}

\usepackage{tikz}

\usetikzlibrary{external}

\usepackage{pgffor}

\usepackage{animate}

\usepgflibrary{fpu}

%% \tikzexternalize

%% \setlength\parindent{0pt}

\pgfmathsetmacro\halfpi{pi/2}
\pgfmathdeclarefunction{easeInSin}{4}{%
  \pgfmathparse{ -#3 * cos(#1/#4 * \halfpi r) + #3 + #2}% radians
}

\pgfmathsetmacro\s{1.70158}

\pgfmathdeclarefunction{ezoutback}{4}{%
  \pgfmathsetmacro\dt{#1/#4-1}
  \pgfmathparse{ #3*((\dt)^2*((\s+1)*\dt+\s)+1) + #2}
}

%% \pgfkeys{/ymax/.initial=0}
%% \pgfkeys{/tikz/.cd,
%%   ymax/.store in=\ymax,
%%   ymax=0,
%%    }

\pgfkeys{/ymax/.initial=0}
%% \pgfkeyssetvalue{/ymax}{0}


%%%%%%%%%%%%%%%%%%%%%%%%%%%%%%%%%%%%%%%%%%%%%%%%%%%%%%%%%%%%%%%%
\begin{document}


\def\iFrameCount{30}
\def\iFrameRate{30}
\def\ydist{4}
\def\xdist{4}
\def\rFromY{0}	%% b : begin point
\def\rToY{4}	%%   : end point
\pgfmathsetmacro\rDeltaY{\rToY-\rFromY}  %% c :  end-begin, total change

\def\pts{(0,1),(1,2),(2,3)}
\expandafter\def\expandafter\coords{\foreach \pt in \pts {\pt}}

\begin{animateinline}[poster=first,%
    controls,
%% autoplay,loop,
    begin={\begin{tikzpicture}[scale=1]%
        \useasboundingbox (-5,-3) rectangle (6,3);
    },%
    end={\end{tikzpicture}}%
  ]%
  {\iFrameRate}%
  \multiframe{\iFrameCount}{iCurrFrame=0+1}{%
    \message{\iCurrFrame}
    \pgfmathsetmacro{\pct}{(\iCurrFrame)/(\iFrameCount - 1)};
    \pgfmathsetmacro{\xdelta}{(\pct > .5)? (1-\pct)*\xdist : (\pct)*\xdist}
    \pgfmathsetmacro{\ydelta}{\pct*\ydist}
    %% \pgfmathsetmacro{\yda}{(\pct > .5)? \ydist : (2*\pct)*\ydist}
    %% \pgfmathsetmacro{\ydb}{(\pct > .5)? (\pct - .5)*2*\ydist : 0}

      %% \draw[help lines] (-4,-2) grid (4,2);

      \draw (-4,-2) node (a0) {};
      \draw (-4,2)  node (b0) {};
      \draw[gray] (a0.center) .. node[pos=\pct] (mid){} controls (-3,0) .. (b0.center);

      %% \draw (-2,-2+\y)  node (b1) {};
      %% \draw[thick] (a0.center) .. controls (-3,0) .. (-4,-2+\ydelta);
      \draw[thick,pos=.5] (a0.center) .. controls (mid.center) .. (mid.center);
      \fill (a0) circle[radius=1pt] node[left] {$a$};
      \fill (b0) circle[radius=1pt] node[right] {$b$};

      \pgfmathsetmacro\y{easeInSin(\iCurrFrame,\rFromY,\rDeltaY,(\iFrameCount-1))};
      \draw (-2,-2) node (a1) {};
      \draw (-2,2)  node (b1) {};
      \draw[gray] (a1.center) -- (b1.center);
      %% \draw (-2,-2+\y)  node (b1) {};
      \draw[thick] (a1.center) -- (-2,-2+\y);
      \fill (a1) circle[radius=1pt] node[left] {$a$};
      \fill (b1) circle[radius=1pt] node[right] {$b$};

      %%%%%%%%%%%%%%%%%%%%%%%%%%%%%%%%%%%%%%%%%%%%%%%%%%%%%%%%%%%%%%%%
      %% linear
      \draw (0,-2) node (a2) {};
      \draw (0,2)  node (b2) {};
      \draw[gray] (a2.center) -- (b2.center);
      %% \draw (0,-2+\ydelta)  node (b2) {};
      \draw[thick] (a2.center) -- (0,-2+\ydelta);
      \fill (a2.center) circle[radius=1pt] node[left] {$a$};
      \fill (b2.center) circle[radius=1pt] node[right] {$b$};

      %% for some reason this cause x to offset right
      \pgfmathsetmacro\yc{ezoutback(\iCurrFrame,\rFromY,\rDeltaY,\iFrameCount)};

      %% trying to store ymax in global, but can't get it to work!
      \pgfkeysgetvalue{/ymax}{\ymax};
      \message{\ymax};
      %% \pgfmathparse{ (\yc > \pgfkeysvalueof{/ymax}) ? \yc : \pgfkeysvalueof{/ymax}};
      \pgfmathparse{ (\yc > \ymax) ? \yc : \ymax};
      \pgfkeys{/ymax/.expanded=\pgfmathresult};

      %% \message{\yc};
      %% \message{\ymax};

      \draw (2,-2) node (a3) {};
      \draw (2,2)  node (b3) {};
      \draw[gray] (a3.center) -- (b3.center);
      %% \draw (2,-2+\yc)  node (b3) {};
      %% \draw[red] (a3.center) -- (ymax.center);
      \draw[thick] (a3.center) -- (2,-2+\yc);
      \fill (a3.center) circle[radius=1pt] node[left] {$a$};
      \fill (b3.center) circle[radius=1pt] node[right] {$b$};

      \draw (4,-2) node (a4) {};
      \draw (4,2)  node (b4) {};
      \draw[gray] (a4.center) -- (4,0) arc[start angle=0,end angle=360,radius=.5] -- (b4.center);
           %% .. controls (b4.center) .. (5,0);
           %% .. controls (a4.center) .. (3,0) .. controls (b4.center) .. (b4.center);
      %% \draw (2,-2+\yc)  node (b4) {};
      %% \draw[red] (a4.center) -- (ymax.center);
      \draw[red] (4,0) arc[start angle=0,end angle=(\pct*360),radius=.5];
      \draw[thick] (a4.center) -- (4,-2+\ydelta);
      \fill (a4.center) circle[radius=1pt] node[left] {$a$};
      \fill (b4.center) circle[radius=1pt] node[right] {$b$};

  }
\end{animateinline}

\end{document}
