\documentclass{standalone}

\usepackage{fontspec,xltxtra,xunicode}
\usepackage{fontspec}
\defaultfontfeatures{Scale=MatchLowercase}

%% \setmainfont[Mapping=tex-text]{Times New Roman}
%% \setromanfont[Mapping=tex-text]{Times New Roman}
%% \setsansfont[Mapping=tex-text]{Arial}

\setmainfont[Mapping=tex-text]{TeX Gyre Pagella}
%% \setromanfont[Mapping=tex-text]{TeX Gyre Pagella}
%% \setsansfont[Mapping=tex-text]{TeX Gyre Heros}

\usepackage{tikz}
\usepackage{pgfplots}
\usepackage{pgfplotstable}
%% Preamble:
\pgfplotsset{width=7cm,compat=1.9}
%% \pgfplotsset{xticklabel={\tick},scaled x ticks=false}
\pgfplotsset{plot coordinates/math parser=false}

\usetikzlibrary{arrows,shapes,patterns,backgrounds,spy}
\usepackage{pgffor}

\usepackage{animate}

\usepackage{arrayjobx}
\usepackage{multido}

\usepackage{layouts}

\usepackage{etoolbox}

\newcounter{mylistcounter}

\def\saveitem#1{%
\stepcounter{mylistcounter}%
\expandafter\def\csname mylist\themylistcounter\endcsname{#1}}

\forcsvlist{\saveitem}{%
  1,
  2,
  3
}%

\def\getnthelement#1{\csname mylist#1\endcsname}


%%%%%%%%%%%%%%%%%%%%%%%%%%%%%%%%%%%%%%%%%%%%%%%%%%%%%%%%%%%%%%%%
\begin{document}


\def\FrameCount{30}
\def\FrameRate{30}
\def\dist{2}
\def\xdist{4}

\def\pts{(0,1),(1,2),(2,3)}
\expandafter\def\expandafter\coords{\foreach \pt in \pts {\pt}}

%% \begin{figure}[t]
%% \centering
%% \begin{tikzpicture}
%%   \draw[help lines] (-4,-4) grid (4,4);

%%   \fill (-2,0) node[circle,fill] (a) {};
%%   \fill (2,0) node[circle,fill] (b) {};
%%   \draw (a) .. controls (0,1) .. (b);
%%   \draw (a) .. controls (0,-1) .. (b);

%% \end{tikzpicture}

\begin{animateinline}[poster=first,%
    controls,
    begin={\begin{tikzpicture}[scale=1]%
        %% \useasboundingbox[draw=black] (0,0) rectangle (\dist,\dist);
    },%
    end={\end{tikzpicture}}%
  ]%
  {\FrameRate}%
  %% \filldraw (0,0) circle (.4pt);
  %% \newframe %% step 0
  %% \filldraw (0,0) circle (.4pt);

  \multiframe{\FrameCount}{iCurrFrame=0+1}{%
    \message{\iCurrFrame}
    \pgfmathsetmacro{\pct}{(\iCurrFrame)/(\FrameCount - 1)};
    \pgfmathsetmacro{\xdelta}{(\pct > .5)? (1-\pct)*\xdist : (\pct)*\xdist}
    \pgfmathsetmacro{\ydelta}{\pct*\dist}
    \pgfmathsetmacro{\yda}{(\pct > .5)? \dist : (2*\pct)*\dist}
    \pgfmathsetmacro{\ydb}{(\pct > .5)? (\pct - .5)*2*\dist : 0}

      %% \draw[help lines] (-4,-2) grid (4,2);

      \draw (-4,0) node (a) {};
      \draw (0,0) node (b) {};
      \draw (4,0) node (c) {};

      \fill[gray] (a.center) .. controls (-2,-1) .. (b.center) .. controls (-2,-1+\ydelta) .. cycle;
      \draw (a.center) .. controls (-2,-1+\ydelta) .. (b.center);
      \draw (a.center) .. controls (-2,1) .. (b.center);
      \draw (a.center) .. controls (-2,-1) .. (b.center);

      %% \fill[gray] (b.center) .. controls (2,-1) .. (c.center)
      %%     .. controls (2+\xdelta,-1+\ydelta) .. cycle;
      %% \draw (b.center) .. controls (2+\xdelta,-1+\ydelta) .. (c.center);
      \draw (b.center) .. controls (2,1) .. (c.center);
      \draw (b.center) .. controls (2,-1) .. (c.center);

      \fill[gray] (b.center) .. controls (2,-1+\yda) and (2,-1+\ydb) .. (c.center)
          .. controls (2,-1) .. cycle; %
      \draw (b.center) .. controls (2,-1+\yda) and (2,-1+\ydb) .. (c.center);


      %% \draw[gray] (b.center) .. controls (2,1) and (2,-1) .. (c.center);
          %% .. controls (2,-1) .. cycle; %


      %% \fill[red] (2,-1+\yda) circle[radius=1pt];
      %% \fill[blue] (2,-1+\ydb) circle[radius=1pt];
      %% \fill[red] (2+\xdelta,-1+\ydelta) circle[radius=1pt];

      \fill (-4,0) circle[radius=1pt];
      \fill (0,0) circle[radius=1pt];
      \fill (4,0) circle[radius=1pt];




      %% \end{axis}
  }
\end{animateinline}

\end{document}
