\documentclass{standalone}

\usepackage{fontspec,xltxtra,xunicode}
\usepackage{fontspec}
\defaultfontfeatures{Scale=MatchLowercase}

%% \setmainfont[Mapping=tex-text]{Times New Roman}
%% \setromanfont[Mapping=tex-text]{Times New Roman}
%% \setsansfont[Mapping=tex-text]{Arial}

\setmainfont[Mapping=tex-text]{TeX Gyre Pagella}
%% \setromanfont[Mapping=tex-text]{TeX Gyre Pagella}
%% \setsansfont[Mapping=tex-text]{TeX Gyre Heros}

\usepackage{tikz}
%% \usepackage{pgfplots}
%% \usepackage{pgfplotstable}
%% %% Preamble:
%% \pgfplotsset{width=7cm,compat=1.9}
%% %% \pgfplotsset{xticklabel={\tick},scaled x ticks=false}
%% \pgfplotsset{plot coordinates/math parser=false}

%% \usetikzlibrary{arrows,shapes,patterns,backgrounds,spy}
\usepackage{pgffor}
\usepgflibrary{fpu}

\usepackage{animate}

%% \usepackage{arrayjobx}
%% \usepackage{multido}

%% \usepackage{layouts}

%% \usepackage{etoolbox}

%% \newcounter{mylistcounter}

%% \def\saveitem#1{%
%% \stepcounter{mylistcounter}%
%% \expandafter\def\csname mylist\themylistcounter\endcsname{#1}}

%% \forcsvlist{\saveitem}{%
%%   1,
%%   2,
%%   3
%% }%

%% \def\getnthelement#1{\csname mylist#1\endcsname}
\pgfmathdeclarefunction{easeInQuadratic}{4}{%
  \pgfmathparse{#3*(#1/#4)^2+#2} %% c(t/d)^2 + b
}

\pgfmathdeclarefunction{easeInQuintic}{4}{%
  \pgfmathparse{#3*(#1/#4)^5+#2}%
}

\pgfmathsetmacro\s{1.70158}
\pgfmathdeclarefunction{easeInBack}{4}{%
  \pgfmathsetmacro\dt{#1/#4}
  \pgfmathparse{ #3 * (\dt)^2 * ((\s+1)*\dt - \s) + #2}
}

%%%%%%%%%%%%%%%%%%%%%%%%%%%%%%%%%%%%%%%%%%%%%%%%%%%%%%%%%%%%%%%%
\begin{document}


\def\iFrameCount{30}
\def\iFrameRate{30}
\def\rFromY{0}	%% b : begin point
\def\rToY{2}	%%   : end point
\pgfmathsetmacro\rDeltaY{\rToY-\rFromY}  %% c :  end-begin, total change

\def\dist{2}
\def\xdist{4}

\def\pts{(0,1),(1,2),(2,3)}
\expandafter\def\expandafter\coords{\foreach \pt in \pts {\pt}}

%% \begin{figure}[t]
%% \centering
%% \begin{tikzpicture}
%%   \draw[help lines] (-4,-4) grid (4,4);

%%   \fill (-2,0) node[circle,fill] (a) {};
%%   \fill (2,0) node[circle,fill] (b) {};
%%   \draw (a) .. controls (0,1) .. (b);
%%   \draw (a) .. controls (0,-1) .. (b);

%% \end{tikzpicture}

\begin{animateinline}[poster=first,%
    controls,
    begin={\begin{tikzpicture}[scale=1]%
        %% \useasboundingbox[draw=black] (0,0) rectangle (\dist,\dist);
    },%
    end={\end{tikzpicture}}%
  ]%
  {\iFrameRate}%
  %% \filldraw (0,0) circle (.4pt);
  %% \newframe %% step 0
  %% \filldraw (0,0) circle (.4pt);

  \multiframe{\iFrameCount}{iCurrFrame=0+1}{%
    \message{\iCurrFrame}
    \pgfmathsetmacro{\pct}{(\iCurrFrame)/(\iFrameCount - 1)};
    \pgfmathsetmacro{\xdelta}{(\pct > .5)? (1-\pct)*\xdist : (\pct)*\xdist}
    \pgfmathsetmacro{\ydelta}{\pct*\dist}
    \pgfmathsetmacro{\yda}{(\pct > .5)? \dist : (2*\pct)*\dist}
    \pgfmathsetmacro{\ydb}{(\pct > .5)? (\pct - .5)*2*\dist : 0}

      %% \draw[help lines] (-4,-2) grid (4,4);

      %%%%%%%%%%%%%%%%%%%%%%%%%%%%%%%%%%%%%%%%%%%%%%%%%%%%%%%%%%%%%%%%
      %% row 1

      %%  figure 1a
      \draw (-2.5,-2) node {easing: linear};
      \draw (-4,-1) node (a1a) {};
      \draw (-1,-1) node (b1a) {};

      \fill[gray] (a1a.center) .. controls (-2.5,-2) .. (b1a.center)
          .. controls (-2.5,-2+\ydelta) .. cycle;
      \draw (a1a.center) .. controls (-2.5,-2+\ydelta) .. (b1a.center);

      \draw (a1a.center) .. controls (-2.5,0) .. (b1a.center);
      \draw (a1a.center) .. controls (-2.5,-2) .. (b1a.center);

      \fill (a1a) circle[radius=1pt] node[left]{a};
      \fill (b1a) circle[radius=1pt] node[right]{b};

      %%%%%%%%%%%%%%%%%%%%%%%%%%%%%%%%%%%%%%%%%%%%%%%%%%%%%%%%%%%%%%%%
      %%  figure 1b
      \draw (2.5,-2) node {easing: quintic};
      \draw (1,-1) node (a1b) {};
      \draw (4,-1) node (b1b) {};

      \pgfmathsetmacro\ya{easeInQuintic(\iCurrFrame,\rFromY,\rDeltaY,(\iFrameCount-1))};
      \fill[gray] (a1b.center) .. controls (2.5,-2) .. (b1b.center)
          .. controls (2.5,-2+\ya) .. cycle;
      \draw (a1b.center) .. controls (2.5,-2+\ya) .. (b1b.center);

      \draw (a1b.center) .. controls (2.5,0) .. (b1b.center);
      \draw (a1b.center) .. controls (2.5,-2) .. (b1b.center);

      \fill (a1b) circle[radius=1pt] node[left]{a};
      \fill (b1b) circle[radius=1pt] node[right]{b};

      %%%%%%%%%%%%%%%%%%%%%%%%%%%%%%%%%%%%%%%%%%%%%%%%%%%%%%%%%%%%%%%%
      %%  row 2
      %%  2a
      \draw (-2.5,.75) node {easing: back in};
      \draw (-4,2) node (a2a) {};
      \draw (-1,2) node (b2a) {};
      \pgfmathsetmacro\ya{easeInBack(\iCurrFrame,\rFromY,\rDeltaY,(\iFrameCount-1))};
      \fill[gray] (a2a.center) .. controls (-2.5,1) .. (b2a.center)
          .. controls (-2.5,1+\ya) .. cycle;
      \draw (a2a.center) .. controls (-2.5,1+\ya) .. (b2a.center);

      \draw (a2a.center) .. controls (-2.5,3) .. (b2a.center);
      \draw (a2a.center) .. controls (-2.5,1) .. (b2a.center);

      \fill (a2a.center) circle[radius=1pt] node[left]{a};
      \fill (b2a.center) circle[radius=1pt] node[right]{b};

      %%  figure 2b
      \draw (2.5,.75) node {easing: linear};
      \draw (1,2) node (a2b) {};
      \draw (4,2) node (b2b) {};

      \fill[gray] (a2b.center) .. controls (2.5,1+\yda) and (2.5,1+\ydb) .. (b2b.center)
          .. controls (2.5,1) .. cycle; %
      \draw (a2b.center) .. controls (2.5,1+\yda) and (2.5,1+\ydb) .. (b2b.center);
      \draw (a2b.center) .. controls (2.5,3) .. (b2b.center);
      \draw (a2b.center) .. controls (2.5,1) .. (b2b.center);

      \fill (a2b.center) circle[radius=1pt] node[left]{a};
      \fill (b2b.center) circle[radius=1pt] node[right]{b};

  }
\end{animateinline}

\end{document}
