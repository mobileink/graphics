\documentclass{article} % say
\usepackage{tikz}
\pgfrealjobname{lineplotsA}

\begin{document}

\def\xmax{1}
\def\ymax{1}
\def\xmargin{.2}
\def\ymargin{.2}

\beginpgfgraphicnamed{galinsky-coefficients}
\begin{tikzpicture}[domain=-\xmax:\xmax,
scale=6
  ]
  \useasboundingbox[draw=black] (-\xmax-\xmargin,-\ymax-\ymargin) rectangle (\xmax+3*\xmargin,\ymax+\ymargin);
  %% \draw[very thin,color=gray] (-0.1,-0.1) grid (100,100);
  \draw (-\xmax/2,0) -- (\xmax/2,0) node[right] {}; %% x axis
  \draw (0,-\ymax/2) -- (0,\ymax/2) node[above] {}; %% y axis
  \draw[color=red] plot (\x,.12*\x) ;
  \def\xpt{\xmax}
  %% \filldraw[] (\xpt,.12*\xpt) circle (2pt) node (sex) {};
  \draw[color=green!50!black] plot (\x,.25*\x) node[pos=.8] (ownpysc) {};
  \draw[color=blue] plot (\x,.11*\x) node[pos=.7] (otherpsyc) {};

  %% \draw (\xmax/2,\ymax/4) node {Sex freq} -- (sex) ;
\end{tikzpicture}
\endpgfgraphicnamed

\end{document}
