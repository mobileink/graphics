\documentclass{article}

\usepackage{fontspec,xltxtra,xunicode}
\usepackage{fontspec}
\defaultfontfeatures{Scale=MatchLowercase}

%% \setmainfont[Mapping=tex-text]{Times New Roman}
%% \setromanfont[Mapping=tex-text]{Times New Roman}
%% \setsansfont[Mapping=tex-text]{Arial}

\setmainfont[Mapping=tex-text]{TeX Gyre Pagella}
%% \setromanfont[Mapping=tex-text]{TeX Gyre Pagella}
%% \setsansfont[Mapping=tex-text]{TeX Gyre Heros}

\usepackage{tikz}
\usepackage{pgfplots}
%% \pgfplotsset{compat=1.9}

\usetikzlibrary{arrows,shapes,patterns,backgrounds,spy}
\usepackage{pgffor}

\usepackage{animate}

\usepackage{arrayjobx}
\usepackage{multido}

\usepackage{layouts}

%%%% macros %%%%


%%%%%%%%%%%%%%%%%%%%%%%%%%%%%%%%%%%%%%%%%%%%%%%%%%%%%%%%%%%%%%%%
\begin{document}

\newarray\Actors \newarray\Dates \newarray\Sexes

\readarray{Actors}{Louise Brooks&Marlene Dietrich&Clark Gable} \readarray{Dates}{1906--1985&1902--1992&1901--1960} \readarray{Sexes}{2&2&1}
\begin{description} \item[\Actors(1)] : \Dates(1) \item[\Actors(2)] : \Dates(2) \item[\Actors(3)] : \Dates(3)
\end{description}

\newcommand{\NumberActors}{3}
\begin{description} \multido{\iActor=1+1}{\NumberActors}{%
\item[\Actors(\iActor)] : \Dates(\iActor)} \end{description}

\newlength\point
\newlength\dist

\def\FrameCount{60}

\global\pgfplotsarraynewempty\Pts
\global\pgfplotsarraynewempty\Cts

\multido{\i=1+1}{\FrameCount}{%
  \pgfplotsarraypushback\i\to\Cts;
}
       \pgfplotsarrayforeach\Cts\as\c{
         \message{\c}
         %% \draw (\theta r:\d) circle (2pt);
       }

       \pgfmathparse{2*pi};
       \pgfmathsetmacro\twopi{\pgfmathresult}
       \pgfmathparse{random(0,2*pi)};
       \pgfmathsetmacro\theta{\pgfmathresult}

       %% \pgfmathsetmacro\radius{\pgfmathresult}
       %% \multido{\i=1+1}{\iCurrFrame}{%
       %%   \Actors(\i)={\pgfmathresult pt};
       %% }

       %% \pgfmathparse{rnd}
       \pgfmathsetmacro{\radius}{rnd}
       \message{\radius}
       %% \Actors(1)={\radius}
       %% \message{\Actors(1)}
       \pgfmathsetlength{\dist}{10*\radius}
       \message{\the\dist}
       %% Actors(\iCurrFrame)={\dist}



\begin{animateinline}[poster=first]{10}%
   \multiframe{\FrameCount}{iCurrFrame=1+1}{%
     \message{\iCurrFrame}
     %% \pgfmathsetmacro\x{\pgfmathparse{rnd}\pgfmathresult}
     \begin{tikzpicture}
       \useasboundingbox (-1,-1) rectangle (2,2);

       \draw (0,0) circle (1);
       %% \draw (0,0) -- (3cm,3cm);
%% \message{\pgfmathresult}
%%        %% \Pts(\iCurrFrame)={.7};
%%        \Pts(1)={1};
%%  \message{\Pts(1)}

%%        \foreach \i in {1,..\iCurrFrame} {
%% \message{\i}

         %% \filldraw (\pts(\iCurrFrame).3,0) circle (.1);
       %% }

%% see http://tex.stackexchange.com/questions/83979/programming-with-pgf-arrays-how-to-create-an-array

       %% \pgfplotsarrayforeach\Pts\as\d{
       %%   \message{\the\d}
       %%   \draw (\theta r:\d) circle (2pt);
       %% }

       \pgfplotsarrayforeach\Cts\as\c{
         \message{\c}
         %% \draw (\theta r:\d) circle (2pt);
       }

       %% \multido{\i=1+1}{\iCurrFrame}{%
       %%   %% \setlength\point{\pgfmathresult pt}
       %%   %% \setlength\point{\Actors(\i)}
       %%   \message{\the\point}
       %%   %% \draw (0,0) circle (2pt);
       %% }

     \end{tikzpicture}
       %% \multido{\i=1+1}{\iCurrFrame}{%
       %%   %% \setlength\point{\i pt}
       %%   \i
       %%   \the\point
       %% }
   }
\end{animateinline}

\end{document}
