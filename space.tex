\documentclass[reqno,12pt]{tufte-handout}
%% \documentclass[reqno,12pt]{tufte-book}
%% \usepackage{trace}
%% \documentclass[reqno,12pt]{article}

\usepackage{appendix}

%% \usepackage{csquotes}

\usepackage{amsmath}
\usepackage{setspace}
\numberwithin{equation}{subsection}
\usepackage{amsfonts}

%% broken (doesn't work with tufte-handout):
\usepackage{zed-csp}
%% broken:
%% \usepackage{ltcadiz-fam}

\usepackage{fontspec,xltxtra,xunicode}
\usepackage{fontspec}
\defaultfontfeatures{Scale=MatchLowercase}

%% \defaultfontfeatures{Scale=MatchLowercase}
%% \setmainfont[Mapping=tex-text]{Times New Roman}
%% \setsansfont[Mapping=tex-text]{Arial}
%% \setmonofont{Courier}

\setmainfont[Ligatures=TeX]{TeX Gyre Pagella}
\setromanfont[Ligatures=TeX]{TeX Gyre Pagella}
\setsansfont[Ligatures=TeX]{TeX Gyre Heros}
\setmonofont{Courier}

%% \setmainfont[Mapping=tex-text]{Minion Pro}
%% \setromanfont[Mapping=tex-text]{Minion Pro}
%% \setsansfont[Mapping=tex-text]{TeX Gyre Heros}

%% Bugfix: see https://code.google.com/p/tufte-latex/issues/detail?id=64
% Set up the spacing using fontspec features
\renewcommand\allcapsspacing[1]{{\addfontfeature{LetterSpace=15}#1}}
\renewcommand\smallcapsspacing[1]{{\addfontfeature{LetterSpace=10}#1}}

\usepackage{epigraph}
\setlength{\epigraphwidth}{.8\textwidth}

%% general symbols - degree, etc.
\usepackage{gensymb}

\usepackage [english]{babel}
\usepackage [autostyle, english = american]{csquotes}

\usepackage{mflogo}

\numberwithin{equation}{subsection}

%% nice double-stroke fonts
\usepackage{dsfont}

% Small sections of multiple columns
\usepackage{multicol}

% Provides paragraphs of dummy text
\usepackage{lipsum}

% The following package makes prettier tables.  We're all about the bling!
\usepackage{booktabs}

% The units package provides nice, non-stacked fractions and better spacing
% for units.
\usepackage{units}



%\usepackage{geometry}                % See geometry.pdf to learn the layout options. There are lots.
%\geometry{letterpaper}                   % ... or a4paper or a5paper or ... 

\usepackage{xfrac}

\usepackage{hyperref}
\hypersetup{
  bookmarks=true,         % show bookmarks bar?
  bookmarksdepth=3,
  unicode=true,          % non-Latin characters in Acrobat’s bookmarks
  pdftoolbar=true,        % show Acrobat’s toolbar?
  pdfmenubar=true,        % show Acrobat’s menu?
  pdffitwindow=false,     % window fit to page when opened
  pdfstartview={FitH},    % fits the width of the page to the window
  pdftitle={The Poetics of Coordinate Space},    % title
  pdfauthor={G. A. Reynolds},     % author
  pdfsubject={Coordinate Space},   % subject of the document
  pdfcreator={G. A. Reynolds},   % creator of the document
  pdfproducer={Producer}, % producer of the document
  pdfkeywords={Coordinate Space} {Graphics}
  pdfnewwindow=true,      % links in new window
  colorlinks=true,       % false: boxed links; true: colored links
  linkcolor=blue,          % color of internal links
  citecolor=blue,        % color of links to bibliography
  filecolor=magenta,      % color of file links
  urlcolor=cyan           % color of external links
}

%% \usepackage[
%% bibstyle=numeric,
%% citestyle=authoryear,
%% hyperref,
%% bibencoding=utf8,
%% backref=true,
%% backend=biber]{biblatex}

%% http://tex.stackexchange.com/questions/66778/citation-alias-with-multibib-and-natbib
%% \makeatletter
%% \def\@mb@citenamelist{cite,citep,citet,citealp,citealt,citepalias,citetalias}
%% \makeatother

%% http://stackoverflow.com/questions/2496599/how-do-i-cite-the-title-of-an-article-in-latex
\defcitealias{z-iso-13568}{ISO 13568:2002 Information technology -- Z formal specification notation --
  Syntax, type system and semantics}

\usepackage{tikz}
\usetikzlibrary{shapes.geometric,decorations.text,decorations.pathreplacing}
\usepackage{pgfplots}
\pgfplotsset{height=7cm,compat=1.9}

\usepackage{tkz-euclide}
\usetkzobj{all}

\title{The Poetics of Coordinate Space}
\author{G. A. Reynolds}
%\date{}                                           % Activate to display a given date or no date

%%%%%%%%%%%%%%%%
%% tufte-latex customizations

\def\chpcolor{blue!45}
\def\chpcolortxt{blue!60}
%% \def\sectionfont{\sffamily\LARGE}

\setcounter{secnumdepth}{2}

\makeatletter
%Section:
\def\@sectionstrut{\vrule\@width\z@\@height12.5\p@}
\def\@makesectionhead#1{%
  {%\par\vspace{20pt}%
    \parindent -10pt\raggedleft%% \sectionfont
    %% \colorbox{\chpcolor}{%
    %%   \parbox[t]{90pt}{\color{white}\@sectionstrut\@depth4.5\p@\hfill
    %%     \ifnum\c@secnumdepth>\z@\thesection\fi}%
    %% }%
    \vspace{10pt}%
    \begin{minipage}[t]{\textwidth}%{\dimexpr\textwidth-90pt-2\fboxsep\relax}
      \@sectionstrut\hspace{-15pt}\textbf{\textit{\Large #1}}
    \end{minipage}\par
    \vspace{5pt}%
  }
}
%% \def\@makesectionhead#1{%
%%   {\par\vspace{20pt}%
%%    \parindent 0pt\raggedleft\sectionfont
%%    \colorbox{\chpcolor}{%
%%      \parbox[t]{90pt}{\color{white}\@sectionstrut\@depth4.5\p@\hfill
%%        \ifnum\c@secnumdepth>\z@\thesection\fi}%
%%    }%
%%    \begin{minipage}[t]{\dimexpr\textwidth-90pt-2\fboxsep\relax}
%%    \color{\chpcolortxt}\@sectionstrut\hspace{5pt}\textbf{#1}
%%    \end{minipage}\par
%%    \vspace{10pt}%
%%   }
%% }
\def\section{\@afterindentfalse\secdef\@section\@ssection}
\def\@section[#1]#2{%
  \ifnum\c@secnumdepth>\m@ne
  \refstepcounter{section}%
  \addcontentsline{toc}{section}{\protect\numberline{\thesection}#1}%
  \else
  \phantomsection
  \addcontentsline{toc}{section}{#1}%
  \fi
  \sectionmark{#1}%
  \if@twocolumn
  \@topnewpage[\@makesectionhead{#2}]%
  \else
  \@makesectionhead{#2}\@afterheading
  \fi
}
\def\@ssection#1{%
  \if@twocolumn
  \@topnewpage[\@makesectionhead{#1}]%
  \else
  \@makesectionhead{#1}\@afterheading
  \fi
}
\makeatother

%%%%%%%%%%%%%%%%
%% macros

\newtheorem{theorem}{Theorem}
%\newtheorem{cor}{Corollary}
%\newtheorem{lem}{Lemma}
%\theoremstyle{remark}
\newtheorem{remark}{Remark}

\newcommand\cspace{coordinate space}
\newcommand\Cspace{Coordinate space}
\newcommand\CSpace{Coordinate Space}

\newcommand\dspace{design space}
\newcommand\Dspace{Design space}
\newcommand\DSpace{Design Space}

\newcommand\Omg{\(\Omega\)}
\newcommand\sccs{standard cartesian coordinate space}
\newcommand\origin{\((0,0)\)}
\newcommand\ab{\((a,b)\)}
\newcommand\xy{\((x,y)\)}

\newcommand\N{\(\mathds{N}\)}
\newcommand\R{\(\mathds{R}\)}
\newcommand\RR{\(\mathds{R}\times\mathds{R}\)}
\newcommand\Rtwo{\(\mathds{R}^2\)}
\newcommand\Z{\(\mathds{Z}\)}

%%%%%%%%%%%%%%%%%%%%%%%%%%%%%%%%%%%%%%%%%%%%%%%%%%%%%%%%%%%%%%%%
\begin{document}
%% \ifx\traceon\undefined \tracingall \else \traceon \fi

\maketitle

\begin{abstract}
  \begin{center}
  \textit{Abstract}
  \end{center}
  The foundation of any graphics language or system is
  its \cspace{}.  \Cspace{}s organize space.  Despite my title--a play
  on G. Bachelard's famous \textit{The Poetics of Space}--this is a
  paper on the design and specification of a technical language
  of \cspace{}s.
\end{abstract}

\tableofcontents

\begin{remark}
  To cover: coordinate spaces, axes, transform functions, location
  expression (single points), space transformation expression (all
  points); size, bounding box, mapping to device/viewport space.

  Expressivity:  transform functions v. expressions
\end{remark}

\newpage
%%%%%%%%%%%%%%%%%%%%%%%%%%%%%%%%
\section{Introduction}
\label{sect:intro}

\begin{remark}
  What this is about.  Two things: vector graphics, and language.  The
  former motivates analysis and description of \cspace{}s; the latter
  motivates use of formal notations (lambda calculus and Z) to do so.
\end{remark}

%%%%%%%%
\subsection{Basic Concepts}

Coordinate space:  unit, axes, etc.  Basic cartesian space.

Expressivity: specification of points in a space v. transformation of
spaces (all points).  No fundamental distinction; a function that can
be used for one can be used for the other.  It's a question of how one
designs the language, what sort of expressiveness it has.

%%%%%%%%
\subsection{Vector Graphics Language}
\label{subs:reqts}

\begin{remark}
  Expressivity is what counts.  Just about everybody does cs and
  device independece - a conspicous exception being HTML5 Canvas.
\end{remark}


Conditions of adequacy for a graphics language:

\begin{description}
\item [Expressivity] ability to express any geometry
\item [Usability] expressions in the language must be understandable
  by both machines and humans.
\item [Simplicity] as simple as possible, no simpler
\item [Beauty] aesthetics is arguably the unacknowledged driver of
  all design
\item [Coordinate space independence] i.e. support for transforms
\item [Device Independence]
\item [etc]
\end{description}

This paper focuses on coordinate spaces and their relation to the
figures that are inscribed in them.  The goal is to specify a semantic
metalanguage for such spaces and figures that satisfies the criteria
of adequacy outlined above.

\begin{remark}
  Is ``semantic metalanguage'' really the right term?  We're using Z
  to describe a semantic domain, not another language.  We're using it
  as a language, not a metalanguage.
\end{remark}

%%%%%%%%
\subsection{Language and Metalanguage}
\label{subs:langmetalang}

\begin{remark}
  We will use Z to define a semantic metalanguage, which can then be
  used to define (fragments of) concrete languages (DSLs).
\end{remark}

%%%%%%%%
\subsection{Math and Logic}
\label{subs:mathlogic}

%%%%%%%%
\subsection{Sets and Functions}
\label{subs:functions}

\begin{remark}
  Overview.  Motivate use of lambda calculus \& Z.

  Two ways of thinking.  Elements v. relations, things v. structures,
  set theory v. category theory.
\end{remark}

Identity function: expression of the form \ab are interpreted in terms
of functions: (Id,Id)

\begin{remark}
  Functional thinking - from set theory (elements) to category theory
  (arrows).  Or categorial logic.  Everything a function, even
  numerical constants; functions as relations.
\end{remark}

Expressivity and function application - \(f(x)\) as \textit{denoting}
a value but \textit{expressing} a relation between source and target
values.  A lambda abstraction denotes a function qua set of ordered
pairs, but expresses a function qua set of relations from source
element to target elements.

\begin{remark}
  Evaluation strategies and expressivity of lambda calc.
\end{remark}

\begin{remark}
  Operators?  An op is just a well-known function, so well-known that
  convention has settled on a special symbol for it, e.g. \(\_+\_\)
  instead of \(add(\_,\_)\).
\end{remark}

\begin{remark}
  Continuous funcs applied to interval: use domain restriction op of
  Z.  Express by giving lambda expression plus end points of range.
\end{remark}

\begin{remark}
  Compare: \(f(x) = ax + b\) and TikZ's line op ``-\-''.  The domain
  of the former is \R{}; the domain of the latter is
  \((\real\cross\real)\cross(\real\cross\real)\).  So TikZ's func must
  construct the func and the endpoints, which we can express in Z.
\end{remark}

Function polymorphism: accepts different arg types (and numbers?)

Poylymorphic types

%%%%%%%%%%%%%%%%%%%%%%%%%%%%%%%%
\part{Technical Preliminaries}
\label{sect:prelims}

\begin{remark}
  TODO: Some kind of overview here.
\end{remark}

%%%%%%%%%%%%%%%%%%%%%%%%%%%%%%%%
\section{Expressivity}
\label{sect:expressivity}

\begin{remark}
  Use/mention distinction: ``(a,b)'' denotes (a,b)

  Shades of Tarski, ``Snow is white'' means snow is white.  Of course
  it is critical to make the distinction between expressions in a
  language and their meanings - between mention and use.
\end{remark}

Having defined a collection of types, we need a means of expressing
values of those types.  For this we need a language--a
\textit{Domain-Specific Language} or DSL--and we can design our DSL in
whatever way we please to suit our expressive purposes.\sidenote{As a
  set of syntactic conventions, a language may be textual or visual.}

Of course, we also needed a language in order to express our type
definitions; we used Z, which has the virtue of formal semantics.

Definitions in Z describe the \textit{semantic domain}
(i.e. mathematical structure) of our intended DSL.  Since Z is based
on standard (ZF, Zermelo-Frankel) set theory, Z itself is sufficiently
expressive to describe anything describeable in set-theoretic terms,
which is all we need in order to ``do graphics''.

This means that we can furnish our DSL -- whatever form it takes --
with a formal semantics.  All we need to do is map expressions in our
language to the Z expressions we used to define the elements of our
semantic domain.  For example, we will use Z to define \(POINT\) as a
pair of real numbers; in our DSL we might prefer the symbol \(Pt\);
all we need to do is specify a mapping from our DSL to our semantic
language. Something like \(Pt\rightarrow POINT\), for
example.\sidenote{In practice we would also need to indicate which
  terms in such an expression come from our DSL and which from our
  semantic language.  But such hairy details of actually specifying a
  DSL are beyond the scope of this paper.}

Expressiveness is a property of languages, not semantic domains.  The
critical point here is that the formal definitions of our semantic
language specification define \textit{types}.  They impose no
constraint on \textit{expression} in our DSLs.

This issue goes beyond mere renaming of constant symbols.
\textit{Any} expression, no matter how complex, that evaluates to a
value $x$ may be used to express that value in complex expressions.

%% In particular, recalling that a function application expression is a
%% kind of name for the value it denotes, we see that our DSL may give us
%% many different ways to refer to the same semantic value.

Consider the Z expression ``\((1,2)\)''.  This expression denotes a
mathematical object, the ordered pairing of the first two integers.  Z
follows convention in adopting parenthetical notation ``(,)'' for such
expressions.  But Z also includes standard arithmetic operators so we
have an unlimited number of ways of expressing this value:
\(``(1,1+1)'', ``(2-1,\frac{6}{3})'', ``(|\sqrt{1}|,|\sqrt{4}|)''\)
are distinct expressions all denoting the same value, namely
\((1,2)\).\sidenote{The use/mention distinction is critical.
  Quotation marks are the notational convention for expressing
  mentions; unquoted expressions are used, quoted ones are mentioned.
  Compare:

  \vspace{4pt}
         {\centering
           ``Snow is white'' means snow is white.}

         \vspace{4pt}
         Another means of marking the distinction is to use ``semantic
         brackets'' like \(\lbag\ \rbag\).  Then ``1+1'' (quoted) mentions
         (refers to) the expression inside the quotes, and \(\lbag 1+1\rbag\)
         denotes the value of the expression (i.e. the second integer), so we can say

         \vspace{4pt}
                {\centering
                  ``\((1+1)\)'' denotes \(\lbag 2\rbag\)}

}

Most important, we can use function application expressions to denote
values.  For example, the expression \(\lambda x.2x\) denotes the
function that doubles its argument.  \textit{Applying} a function to a
value--or rather, \textit{expressing} its application to a value by
writing down the appropriate bit of notation--denotes the value of the
function for that argument: \(\lambda x.2x(1) = 2\).  So we can
express \((1,2)\) by writing \((1,\lambda x.2x(1))\).

The significance of this will become clear when we discuss evaluation
strategies in section \ref{sect:eval}.  The gist of the story is that
compound expressions like \(1+1\) and \(\lambda x.2x(1)\) denote what
they denote, \textit{whether we actually compute those values or not}.
In the language of the Lambda Calculus, evaluation of such expressions
is called \textit{reduction}, since it involves transforming them to
the simplest possible expression having the same
meaning.\sidenote{``Reduction'' is effectively a synonym for
  ``computation''.}  The critical point is that the meaning of such
expressions is independent of their reduction.  This means that, as we
go about evaluating expressions in our DSL, we are not compelled to
immediately reduce (or fully evaluate, completely compute, etc.) each
expression as we encounter it.  We can just keep track of the
expressions themselves, rather than their (computed) values, knowing
as we do that the meanings of expressions are immutable.  We need not
reduce them until we need them in order to continue with the computation.

%%%%%%%%
\section{Immutability}
\label{sect:immutability}

Virtually all vector graphics languages use mutable data structures.
For example, path construction operators usually update a ``current
point'' datum.

\begin{remark}
  Explain how immutable data structures do not sacrifice expressivity.
  Maybe they improve it.

  One way: paths/figures as sequences rather than sets of points; that
  means they respond to a ``last'' function.  So path constructors
  applied to a path/figure invokes ``last'' to obtain the analog of
  ``current point''.  Except there's no temporal dimension to it, so
  no notion of ``current'' point.
\end{remark}

\begin{remark}
  the more general pt:  expressivity, with and without mutable data.
\end{remark}

%%%%%%%%%%%%%%%%%%%%%%%%%%%%%%%%
\section{Z Notation}
\label{sect:z}

\begin{remark}
  The end result of this paper will be a Z toolkit for graphics, which
  in principle should be sufficient for the formal specification of
  any graphics language.
\end{remark}

Z (pronounced ``Zed'') is a notation for formal
specification.\sidenote{% The official definition of Z is
  \citetalias{z-iso-13568}.  It is available for free download.

See also
\citet{spivey_z_1992} (free download),
\citet{jacky_way_1997},
\citet{woodcock_using_1996},
\citet{potter_introduction_1996}
}%
It is based on
standard mathematical and logical notation, but unlike the informal
conventions governing such notation in schools and among scholars
and professionals, Z is a formal notation, designed to enable its
users to express systems and languages in formal specifications with
rigorously defined semantics.  The Z definition document formally
defines the syntax, type system, and semantics of the notation.

The Z notation is clear, simple, concise, and expressive.  It allows
us to express formal definitions with great precision and clarity,
in an idiom that is easily recognizable and intuitive.  In this
paper, we will extend Z to describe the structure and properties
of \cspace{}s and geometry figures (graphics).

%% Since the
%% definition of Z specifies a formal notation (language) with an
%% explicit formal semantics, expressions written in that notation,
%% like ``\((0,0)\)'' or ``\RR'', have a precise and explicitly defined
%% mathematical meaning.

\begin{description}
\item [binding] finite function from names to values
\item [carrier set] of a type: set of all values in the type
\item [schema] set of bindings
\item [signature] function from names to types; i.e. each schema has a signature
\end{description}

%%%%%%%%%%%%%%%%%%%%%%%%%%%%%%%%
\subsection{Types, Sets \& Values}
\label{sect:types}

\begin{remark}
  Types are not sets. An expression like ``\(a:A\)'' means that \(a\)
  is a value of type \(A\); it does \textit{not} mean that \(a\) is a
  \textit{member} of the set \(A\).  That's the job of the
  \textit{carrier set} of the type.  So if \(a:A\), then we cannot
  infer \(a\in A\), but we can infer \(a\in carrier(A)\).

  Type symbols e.g. \R do not denote sets of values; that's the job of
  the carrier set of the type.
\end{remark}

Types we need to define for our topic:

\begin{itemize}
\item Coordinates
\item Points
\item Spaces
\item Axes
\item Paths
\item Figures (?)
\item other?
\end{itemize}

%%%%%%%%
\subsection{Abbreviation definition}
\label{subs:abbrevs}

``An abbreviation definition introduces a new global constant. The identifier on
the left becomes a global constant; its value is given by the expression on the
right, and its type is the same as the type of the expression.'' (Spivey, p.50)\marginnote{TODO: bib entry for Spivey}

Examples:

\[COORD == \real \\ POINT == COORD\cross COORD \]

This says that a point is a pair of coordinates, and a coordinate is a
real number.  Here ``\(\cross\)'' is the cartesian cross-product
operator.

%%%%%%%%
\subsection{Axiomatic definition}
\label{subs:ztypedefns}

``An axiomatic description introduces one or more global variables, and optionally
specifies a constraint on their values.'' (Spivey, p.48)

For example, this defines a \(sq\) function that computes the square
of its argument:

\begin{axdef}
  sq : \nat\fun\nat
  \where
  \forall n : \nat @ sq(n) = n\times n
\end{axdef}

%%%%%%%%
\subsection{Schema definition}
\label{subs:zschemadefn}

A Z schema \textit{binds} names to values; it is analogous to a struct
in C or a record in other languages.

Generic vertical and horizontal forms:
\begin{marginfigure}[1in]
  \begin{schema}{S}
    D_1;\ldots ;D_m
    \where
    P_1;\ldots ;P_n
  \end{schema}
\end{marginfigure}
\begin{schema}{S}
  D_1;\ldots ;D_m
  \where
  P_1;\ldots ;P_n
\end{schema}

\noindent Equivalently: \(S\defs [D_1;\ldots ;D_m | P_1;\ldots ;P_n]\)
\begin{marginfigure}
  \(S\defs [D_1;\ldots ;D_m | P_1;\ldots ;P_n]\)
\end{marginfigure}

Example:

\begin{marginfigure}[24pt]
  \begin{schema}{Axis}
    origin : COORD \\
    offset : COORD \\
    incidence : RADIAN \\
    scale : COORD\fun COORD
    \where
    origin = (0,0) \\
    offset = scale(origin)
  \end{schema}
\end{marginfigure}
\begin{schema}{Axis}
  origin : \real\cross\real \\
  offset : \real\cross\real \\
  incidence : \real \\
  scale : \real\fun\real
  \where
  origin = (0,0) \\
  offset = scale(origin)
\end{schema}

Same thing in horizontal form:

\[Axis \defs [origin:COORD;offset:COORD;incidence:RADIAN;scale:COORD\fun COORD \\
  \quad\quad\quad | origin=(0,0); offset=scale(origin)]\]

%%%%%%%%
\subsection{Generic definition}
\label{subs:zfndefns}

General form of

\begin{description}
\item [generic definition]
  \begin{gendef}[X,Y]
    <type declarations>
    \where
    <definitions>
  \end{gendef}
\end{description}

Example:

\begin{gendef}[X,Y]
  first : X \cross Y \rightarrow X \\
  second : X \cross Y \rightarrow Y
  \where
  \forall p : X \cross Y @ first\ p = p.1 \\
  \forall p : X \cross Y @ second\ p = p.2
\end{gendef}
\begin{marginfigure}[-12pt]
  \begin{gendef}[X,Y]
    first : X \cross Y \rightarrow X \\
    second : X \cross Y \rightarrow Y
    \where
    \forall p : X \cross Y @ first\ p = p.1 \\
    \forall p : X \cross Y @ second\ p = p.2
  \end{gendef}
\end{marginfigure}

This is a ``generic definition'' of the cartesian pair projection
operators:
\[first(a,b) = a,\quad second(a,b) = b\]
\noindent The top part of the definition describes the types of these
functions; each is of type ``function from set of ordered pairs of
values to a single value.''  $X$ and $Y$ are used to denote generic
types; in other words, these projection functions are polymorphic,
they work on any pair of types.

The second part of the definition (below the bar) states a predicate
and a value expression.

%%%%%%%%%%%%%%%%%%%%%%%%%%%%%%%%
\section{Lambda Calculus}
\label{sect:lambda}

Terminology:
\begin{description}
\item [term]
\item [abstraction]
\item [application]
\item [reduction]
\item [redex]
\end{description}

``[A] \(\lambda\)-calculus term is either a variable, an application
or an abstraction.  A redex is an application whose left term is an
abstraction.  The result of contracting redex \((\lambda M) N\) is the
term \([M[x:=N]\).''\marginnote{\citet{lalement_computation_1993}}

  %%%%%%%%
  \subsection{Types}
  \label{subs:lambdatypes}

  \begin{remark}
    Untyped v. typed calculii.  Relevance to this project: Z expressions
    are all strongly typed so if we're going to use lambda calc it needs
    to be typed, at least implicitly.  But maybe we don't need to make
    this a subsection.
  \end{remark}

  %%%%%%%%
  \subsection{Lambda Abstraction}
  \label{subs:lambda}

  %%%%%%%%
  \subsection{Application}
  \label{subs:applic}

  %%%%%%%%
  \subsection{Reduction}
  \label{subs:reduction}

  Redex, etc.

  %%%%%%%%
  \subsection{Evaluation Strategy}
  \label{subs:evalstrat}

  \begin{description}
  \item [Eager]  save the results of immediate evaluation of expressions to the cs.
  \item [Lazy] a/k/a lazy evaluation; save the expressions to the cs, rather than their evaluations.
  \end{description}

  %%%%%%%%%%%%%%%%%%%%%%%%%%%%%%%%
  %%%%%%%%%%%%%%%%%%%%%%%%%%%%%%%%
  \part{Structure of Spaces}
  \label{sect:cspaces}

  \begin{abstract}
    Abstract:
  \end{abstract}

  %%%%%%%%%%%%%%%%%%%%%%%%%%%%%%%%
  \section{Basics}
  \label{sect:basic}

  \begin{remark}
    A tour of commonly used and obscure systems
  \end{remark}

  %%%%%%%%
  \subsection{Coordinates}
  \label{subs:coordinates}

  Defn:  Coord == \(\mathds{R}\)

  %%%%%%%%
  \subsection{Points}
  \label{subs:points}

  POINT == \(\mathds{R}\times\mathds{R}\)

  A pair of real numbers \ab is just a pair of real numbers; it has no
  intrinsic connection to any space, Euclidean, real world, or other.
  To get points, you have to invent \cspace{}s, which M. Descartes has
  already taken care of.  But a \cspace{} is more than just a set of
  pairs of real numbers; it has additional structure, such as an origin
  and axes.  We will delve into the structure of \cspace{}s in section
  \ref{sect:csstruct}; here it suffices to say: no \cspace{}, no points.

  It follows that when we \textit{define} a POINT as a pair of reals
  (type \RR), we're cheating a little.  A pair of reals in isolation is
  not a point; but any expression denoting such a pair (e.g. \ab),
  occuring as a subexpression in a cs expression\marginnote{FIXME:
    clarify}, does serve to denote a point in a \cspace{}.  Since our
  only (or at least main) use for pairs of reals is so we can use them
  to express coordinate points, we label the type of such pairs POINT,
  rather than, say REALPAIR.

  %%%%%%%%
  \subsection{Coordinate Space}
  \label{subs:cspace}

  Set of points; type: powerset of POINT.

  %%%%%%%%%%%%%%%%%%%%%%%%%%%%%%%%
  \section{Axes}
  \label{sect:axes}

  \begin{remark}
    Distinction between the coordinate space axes, and what gets printed
    as axes on a chart or plot, which may not have anything to do with
    the coordinate space.  Specifying a visual representation for plot
    axes requires a specialized set of functions or parameters.
  \end{remark}

  \begin{remark}
    Aspect ratio?  ``Octave has axis config options that set aspect
    ratio.  ``The following options control the aspect ratio of the
    axes.  "square" Force a square aspect ratio.  "equal" Force x
    distance to equal y-distance.  "normal" Restore default aspect
    ratio.''

    I guess this refers to visual aspect ratio.
  \end{remark}

  Every \cspace{} is associated with one or more axes.  Polarity is a
  property of axes, so it is properly included in their specification
  (see \ref{sect:axes}).  The order of axes, by contrast, is a
  property of the \cspace{} that contains (or uses) them; an axis has
  no way of knowing (so to speak) how it will be used by a \cspace; it
  doesn't even ``know'' how many other axes might be involved.  So it
  would be counterproductive to attach order to axes - might render
  unusable in other spaces.

  Axes are defined by:

  \begin{description}
  \item [name]
  \item [origin]
  \item [unit]
  \item [scale]
  \item [angle of incidence]
  \item [polarity]
  \end{description}

  %%%%%%%%
  \subsection{Name}
  \label{subs:axisname}

Every axis must have a name.

  %%%%%%%%
  \subsection{Origin}
  \label{subs:axisorigin}

  Only axes have origins; \cspace{}s have points at \((0,0)\), but such
  points only count as \textit{origin} points if an axis says so.

  %%%%%%%%
  \subsection{Unit}
  \label{subs:axisunit}


  %%%%%%%%
  \subsection{Polarity}
  \label{subs:polarity}

  \begin{remark}
    Polarity could be expressed in terms of the scaling function.  But
    it seems like a distinct concept, and so should be kept separate.
    Besides, if you're drawing natively into Q3yx you don't want to
    treat it as derivative from some other space, you want it to be
    primitive.
  \end{remark}

  Axes (but not \cspace{}s) have polarity.

  Direction of increasing values: default is rightward, treated as
  ``positive''.  Left/down axes have ``negative'' polarity (pos always
  means rightwards, neg always means leftwards).  Note we only need one
  polarity scale rather than right/left and up/down, since we have angle
  of incidence.  The cartesian y axis has right polarity with 90 deg
  incidence.  Ex.: neg, neg increases in the 3rd quad.

  %%%%%%%%
  \subsection{Incidence}
  \label{subs:incidence}

  (azimuth?) relative to x axis in absolute space.
  Should the cspace be responsible for specifying this in terms of
  angle between axes?  Maybe not, since more axes would mean
  specifying angles pairwise.  Easier to give an absolute angle.

  %%%%%%%%
  \subsection{Scale}
  \label{subs:axisscale}

\begin{remark}
  An axis scale is just a function of type \(\real\fun\real\).
\end{remark}

  Suppose we scale our x axis by \(\lambda r.2r\).  This maps each \(x\)
  value to \(2x\), so e.g. \(2\) ends up at \(4\).

  Question: what does this do to the representation of the axis?  There
  are two immediate possibilities: either the scaling function affects
  the representation, or it doesn't.

  If the scaling function scales the representation as well as the
  mathematical space, then \(2\) will be represented by a point at the \(4\)
  position on the rendered axis, which will be labeled ``8'' (\(2\cdot
  4\)).  Probably not what we want.  If it does not scale the
  representation, then \(2\) will end up in the same place (\(4)\), but
  the label will remain \(4\).  This is more like it; conceptually, the
  scaling function applies to the entire axis, shifting it (but not its
  representation) by a factor of 2.  So any figures drawn to the space
  will be shifted as well.\sidenote{FIXME: Actually the labeling will
    come from the root cspace, which will not be affected by axis
    scaling higher up the stack.}

  \newthought{Now suppose} we scale the y axis by \(\lambda y.2y\).
  This will shift all y values up by a factor of \(2\).  If we then plot
  the function \(\lambda x.x\), the slope of the curve will change; it
  will look like the plot of \(\lambda x.2x\).  In this case, the
  plotting function will actually be the composition of the desired
  function and the y scaling function: \((\lambda r.2r\circ\lambda r.r)
  = \lambda r.2r\)\marginnote{%
    \(\leftarrow\)Here we use \(r\) instead of \(y\)
    and \(x\) to emphasize that the functions do not depend on the
    argument names.
  }%
  : first evaluate the function to be plotted, then
  evaluate the y scaling function.

  Since any function may be used as an axis scaling function, this
  allows us to make the plot of one function ``look like'' the plot of
  any other function.  Figure \ref{fig:yscaling} shows how we can make a
  linear function like \(\lambda x.2x\) look like a quadratic curve by setting the y-axis
  scaling function to \(\lambda y.y^2\), yielding \(\lambda
  r.r^2\circ\lambda r.2r = \lambda r.(2r)^2\).

  \begin{marginfigure}[12pt]%
    \begin{tikzpicture}[scale=.4]
      \useasboundingbox (-2,-4) rectangle (2,4);
      \draw[very thin,color=gray] (-2,-4) grid (2,4);
      \node[draw,circle,inner sep=.5pt,fill] at (0,0) {};
      \draw[color=red,domain=-2:2] plot (\x,2*\x);
      %% \node[rotate=90] at (-2.5,0) {$\lambda y.y$};
      \node at (0,-4.5) {$y$-axis: $\lambda y.2y$};
    \end{tikzpicture}
    \hspace{8pt}
    \begin{tikzpicture}[scale=.4]
      \useasboundingbox (-2,-4) rectangle (2,4);
      \draw[very thin,color=gray] (-2,-4) grid (2,4);
      \node[draw,circle,inner sep=.5pt,fill] at (0,0) {};
      \draw[color=red,domain=-1:1] plot (\x,{(2*\x)^2});
      %% \node[rotate=90] at (-2.4,0) {$\lambda y.y^2$};
      \node at (0,-4.5) {$y$-axis: $\lambda y.y^2$};
    \end{tikzpicture}
    \vspace{12pt}
    \caption{lambda x.2x with different y-axis scaling functions.}
    \label{fig:yscaling}
  \end{marginfigure}%

  \newthought{The moral of the story} is that axes, like coordinate
  spaces, are mathematical entities, independent of visual
  representation.  The task of representing the axes of a coordinate
  space is orthogonal to the task of scaling them.  And since we will
  ultimately want a single, unified representation of the composition of
  our stack of coordinate spaces and the figures inscribed in them,
  specifying that representation will be associated with the root
  space.\sidenote{FIXME: elaborate this, or refer to a section that
    does elaborate on cspace composition and specification of axis
    representations.}

  %%%%%%%%
  \subsection{Examples}
  \label{subs:axiseg}


  Cartesian space has x and y axes, each with unit of 1, angle of
  incidence of 0 and 90 deg, respectively, and orientation of x
  increases rightwards, y upwards.

  Example:  parallel axes.  Set y-axis origin to 5, angle of incidence to zero.

  %%%%%%%%%%%%%%%%%%%%%%%%%%%%%%%%
  \section{\CSpace{}s}
  \label{sect:dspaces}

  \begin{remark}
    Take e.g. cartesian and polar as primitive spaces -
    interconvertable, maybe, but having fundamentally different meanings.

    How many primitive spaces are there?
  \end{remark}

  %%%%%%%%
  \subsection{Carrier Set}
  \label{subs:cscarrier}

  A \cspace{} is a structure, of which its set of points is just one
  component.  Hence the need to specify the \textit{carrier set} of
  a \cspace{}.\sidenote{Compare Z, where each type has a carrier set.
    We're abusing terminology here a bit: since a \cspace{} is a type,
    it has a (Z) carrier set, but that is not what we mean by carrier
    set here.  Maybe we should call it the root set or something
    similar.}

  %%%%%%%%
  \subsection{Axes}
  \label{subs:orientation}

  Every \cspace{} is associated with an ordered set of one or more
  axes.  The ordering of this set is critical, because that is what
  determines the interpretation of locative expressions like
  \((a,b)\).

  Axis order is reflected in writing convention: \((a,b)\) just means
  that \(first(a,b)=a\) maps to the first axis (by convention, the $x$
  axis), and \(second(a,b)=b\) maps to the second axis (by convention,
  the $y$ axis).

  \newthought{The orientation of a \cspace{}} determines where points
  \((a,b)\) are positioned relative to the origin.  This depends on the
  \textit{polarity} of the axes--their directions of increasing value--
  and their \textit{order} within the \cspace{}.  By convention, the
  standard cartesian \cspace{} has Q1\sidenote{%
    Q1:  Quadrants are numered counter-clockwise, starting with the upper right quadrant:

    \centering
    \begin{tikzpicture}[scale=1]
      \useasboundingbox (-1,-1) rectangle (1,1);
      \draw[<->] (-1,0) -- (1,0);
      \draw[<->] (0,-1) -- (0,1);
      \draw (.4,.4) node {Q1};
      \draw (-.4,.4) node {Q2};
      \draw (-.4,-.4) node {Q3};
      \draw (.4,-.4) node {Q4};
    \end{tikzpicture}
  } \((x,y)\) orientation: every point in the first quadrant is
  with first coordinates increasing to the right and second
  coordinates increasing upwards.  To locate a point \((a,b)\) in the
  space, find \(first(a,b)\) on the $x$-axis and \(second(a,b)\) on
  the \(y\)-axis.

  Contrast this with the way cartesian space might be represented in
  cultures with right-to-left writing systems.  For such cultures, it
  would be more ``natural'' -- that is, more consistent with the way the
  writing system organizes space -- to use Q2 \((x,y)\) orientation,
  retaining \((x,y)\) axis order, but reversing the polarity of the
  \(x\) axis.  In Q2, first coordinates%
  \sidenote{%
    \(first(a,b)=a,\ second(a,b)=b\)
  }% would increase to the left rather than the right, placing (+,+)
  %% (+,+)\sidenote{(+,+) \(\rightarrow\) positive in both coordinates}
  points in the second quadrant.  Since the order of the axes remains
  \((x,y)\), that this system is the mirror image of the standard
  cartesian space across the (vertical) y axis.

  \newthought{Changing the axis order} from \((x,y)\) to \((y,x)\) may
  be unorthodox, but it has its uses.  In particular, in typesetting we
  may want to make our \cspace{}s agree with the way writing systems
  organize space.  This means that for some text objects we might want
  to use a \((y,x)\) order.

  For example, both left-to-right and right-to-left writing systems use
  the lower quadrants to organize their writing systems.  Left to right
  systems like English use Q4 orientations: characters flow left to
  right, and lines break, then stack from top to bottom.  Right to left
  systems like Arabic and Hebrew, by contrast, use Q3 orientation: as in
  Q4 left-to-right systems, lines stack top to bottom, but character
  flow (x axis polarity) is reversed, so characters move right to left.

  In both cases, we might want to use a \cspace{} with \((x,y)\) order
  for character layout but one with \((y,x)\) order for line
  layout.\marginnote{TODO: graphic illustration}

  A number of writing systems influenced by the Chinese literary
  tradition have historically organized their texts using a different Q3
  orientation: characters in a line descend top to bottom, and lines in
  a paragraph march from right to left.  For positioning characters, we
  might want to use \((y,x)\) order, so that increasing first
  coordinates track character flow (and thus eye movement) from top to
  bottom.  On the other hand, for positioning lines we might want to use
  \((x,y)\) order, so that increasing first coordinates track line
  stacking leftwards.

  \newthought{The upshot of all this} is that \cspace{}s are somewhat
  more than mere geometry.  They are not just there to make
  transformations of figures possible; they represent ways of organizing
  space.  Organizing space is not a matter of mere geometry; it has
  genuine cultural significance.\sidenote{TODO: something about the
    history of perspective drawing as a way of organizing space and how
    it changed our ways of thinking.  Maybe something from Bachelard's
    Poetics of Space?}

  %%%%%%%%%%%%%%%%%%%%%%%%%%%%%%%%
  \section{\DSpace{}s}
  \label{sect:dspaces}

  \begin{remark}
    Structure, then functions
  \end{remark}

  %%%%%%%%
  \subsection{Design Points}
  \label{sect:dpts}

  \begin{remark}
    Locatives denote locations?  Mention v. use; ``\((a,b)\)'' v. \((a,b)\)
  \end{remark}

  \begin{remark}
    We use the term \textit{lambda point} to refer to points that have
    been ``added'' to \cspace{}s, in order to distinguish between points
    that are present in the \cspace{} by definition and points that are
    added (inserted, projected, etc.) ``into'' the space.  So if we draw
    a point at \origin{} into space \Omg{}, the result is a space that
    contains the same \origin{} point it had to start with, plus the
    lambda point we projected on to that ``native'' point.
  \end{remark}

  \begin{remark}
    Lambda pt as func application - lambda expr, plus arg.
  \end{remark}

  Suppose we start with \cspace{} \Omg{}, and we define a point \ab ``in''
  \Omg{}; something like

  \begin{equation}
    \label{eq:omgapplic}
    Let\ C_1 = \Omega[(a,b)]
  \end{equation}

  Then we transform \Omg{} into some other \cspace{}--call it \(\Gamma\),
  by writing something like

  \begin{equation}
    \label{eq:gammapplic}
    Let\ C_2 = \Gamma[C_1]
  \end{equation}

  The question we want to address now is: what happens when we do this?
  What do statements like these (and their component expressions)
  denote?  What precisely do expressions like \(\Omega[(a,b)]\) and
  \(\Gamma[C_1]\) mean? What do \(C_1\) and \(C_2\) ``contain''?  What
  happened to \ab in formula \ref{eq:gammapplic}?

  Let's start by asking what \Omg{} denotes before we ``apply'' it to a
  \hyperref[subs:pointexprs]{point expression}.  According to our definitions, a \cspace{} denotes
  the combination of:

  \begin{itemize}
  \item \(\mathds{R}\times\mathds{R}\)\marginnote{\(\mathds{R}\times\mathds{R}\)
    = the set of all ordered pairs of real numbers, i.e. Points.}
  \item A pair of axis definitions, each including a scaling function.
  \item A specification of axis order.
  \end{itemize}

  Specifically, let's assume that \Omg{} denotes the \sccs{}; then it
  has \(x\) and \(y\) axes, ordered \xy{}, that have

  \begin{itemize}
  \item Angles of incidence \(0\degree\) and \(90\degree\), respectively
  \item Default (positive) polarity
  \item Identity scaling functions (\(\lambda x.x\))
  \end{itemize}

  So far so good.  Now what does \(Let\ C_1=\Omega[(a,b)]\)
  mean?  Specifically, that is; what's the machinery?

  \begin{remark}
    Long story short: applying a \cspace{} to a point yields a pair of
    the original \cspace{} and a representation of the point.  Under
    eager evaluation, that representation is of the form
    \ab\marginnote{\ab: TODO explain why this should still be considered
      a lambda expression.}; under lazy evaluation, it is the
    (unevaluated) lamba form expressing applyication of the axis scaling
    functions to the coordinates of the point.

    In other words, \cspace{}s never change (immutability); they just
    carry the points they contain along with them as a list of (lambda)
    expressions; hence ``lambda points''.  Only on the last step of
    transforming the design space to device space are all the forms
    fully evaluated.
  \end{remark}

  %%%%%%%%
  \subsection{Ink}
  \label{subs:ink}

  a/k/a style or graphics state

  %%%%%%%%
  \subsection{\DSpace{} Application}
  \label{subs:dsapplic}

  A \dspace{} is formally defined as a function.  This allows us to
  express what traditional languages usually call
  \textit{transformations} as function applications.

  \begin{gendef}[X,Y]
    \_[\_] : \Delta\cross\Delta\fun\Delta \\
    \_[\_] : \Gamma\cross\Gamma\fun\Gamma \\
    \where
    
  \end{gendef}

  A \textit{\dspace{} application} maps one \dspace to another \dspace{}.

  %%%%%%%%%%%%%%%%%%%%%%%%%%%%%%%%%%%%%%%%%%%%%%%%%%%%%%%%%%%%%%%%
  %%%%%%%%%%%%%%%%%%%%%%%%%%%%%%%%%%%%%%%%%%%%%%%%%%%%%%%%%%%%%%%%
  \part{Operations on \DSpace{}s}

  %%%%%%%%%%%%%%%%%%%%%%%%%%%%%%%%
  \section{Transforms}
  \label{sect:transforms}

  \begin{remark}
    The interpretation of function application as expressing a
    relation, mentioned in the intro, is critical in treating
    transformations.  Or at least in working with intuitions about
    transformations, as reflected in common CS design patterns.
  \end{remark}

  %%%%%%%%
  \subsection{Coordinate Point Transforms}

  %%%%%%%%
  \subsection{\CSpace Transforms}

  %%%%%%%%
  \subsection{\DSpace Transforms}
  \label{subs:}

  \Cspace{} transform, plus transform of ``lambda points''.  If lambda points are stored as lambda expressions, this amounts to function composition.

  %%%%%%%%
  \subsection{Notes}

  An example will expose the distinction.  Most graphics languages have
  a "transform" function that allows the user to translate, scale,
  rotate, or skew figures.  The syntax usually makes it look like the
  figures are being transformed; e.g.

  let c = (0,0) circle (1)  ;; figure: unit circle at origin
  translate(c, (1,1))	      ;; move origin of circle to (1,1)

  But that is the wrong interpretation; what gets transformed is the
  entire coordinate space in terms of which the figure is directly
  expressed.  We need a vocabulary that allows us to express this
  explicitly.  In the example, we want a way to say explicitly that the
  transform operator applies to a CS, not to a figure.  For example (toy syntax):

  1)	  let origin(cs:$\Omega$) = origin(cs:Root) + (1,1)  ;; start by specifying translation of cs:$\Omega$
  2)	  let c = (0,0) circle (1) in cs:$\Omega$  ;; then specify figure relative to cs:$\Omega$
  3)	  result: unit circle located at (1,1) in cs:Root

  A more succinct (possible) notation, using [ ] to indicate a cs "context" :

  1)	  let cs:$\Omega$[(0,0)] = cs:Root[(0,0)] + (1,1)  ;; origin in $\Omega$ coords = (1,1) in Abs coords
  2)	  cs:$\Omega$[(0,0) circle 1]   ;; circle expressed in cs:$\Omega$ coordinates
  3)	  result: unit circle located at (1,1) in cs:Root

  This may get the idea across, but the problem is that the transform
  expression only specifies the translation of a single point; it does
  not explicitly say that the entire cs is transformed.  A
  transformation is a function, but 2) does not clearly express this.

  CS transformations are functions and should be expressed as such.

  We could associate transformations with the axes of a CS.  For
  example, for cs:Cartesian we might write:

  axis : {name="x", unit=1, transform=\(\lambda x.x\), incidence=0}
  axis : {name="y", unit=1, transform=\(\lambda y.y\), incidence=0}

  But where do the inputs to the transform functions come from, and what
  is the meaning of their outputs?  The basic idea of a transform is
  that it maps points in one CS to points in another CS.  But the
  definition of an axis is scoped its CS, which leaves open the question
  of where we find the xs and ys to which the transforms apply.

  The solution to this problem is to see that transformations are not
  associated with particular spaces.  They map one space to another,
  both of which are in themselves independent of the transformation.  So
  we do not want to include a transformation function as part of the
  definition of a CS.  It's tempting to do so (possibly because of the
  influence of Object-Oriented thinking), but incorrect; a
  transformation is just a function, and is not connected with any
  "class" or "object".  This will lead us below to consider an algebra
  of coordinate spaces.

  It follows that we have to define a default "scale" for each axis,
  just as we define a default unit of 1.  Axiomatically, as it were.

  In other words, it is not accurate to say that transforms "change"
  coordinate spaces.  Instead, they map all the coordinates from one
  space to another, leaving both spaces per se unchanged.  This is
  critical, since it allows us to express various figures, each in its
  own cs, and then map them to a single "parent" cs under different
  transforms.  If transforms had the effect of producing a "distorted"
  version of the source space, this would not be possible, or at least
  it would be much more awkward.

  So how should we express such transformations?  First, since they are
  functions, we can use lambda notation to express them.  E.g.

  {\setstretch{1.25}
    \begin{alignat}{1}
      & \lambda x.x \\
      & \lambda x.2x   ;; scale x axis by factor of 2 \\
      & \lambda x.ln x ;; logarithmic x axis
    \end{alignat}
  }

  Second, we should make explicit the fact that inputs are coordinates
  in one space, and outputs are coordinates in another:

  $$x_a = \lambda x_b.x_b  ;; where _a = cs:Root, _b = cs:\Omega$$

  In other words, we need a typed lambda calculus.

  Third, they are applied to cspaces, but not part of their definitions.
  So:

  \begin{equation}
    x:B["x"] = \lambda a::A["x"].2a
  \end{equation}


  where foo::Bar["baz"] means foo is a coordinate on the "baz" axis of space Bar.

  (This notation is awkward, but at this point we are interested in
  clarity and explicitness.)

  Whatever sort of notation we come up with, we still have the problem
  of how to express the idea that one space is to be mapped into another
  ("into" literally, points into space) /under/ a transformation
  function.  In other words, in addition to specifying the function
  (possibly as a lambda expression), we also need to express the domain
  and range.

  Maybe something like the following, which apes y=f(x) notation:

  cs:B.x = f(cs:A.x)

  But y=f(x) defines a function; it does not explicitly state dom and ran.

  To be really explicit we might try:

  $$f = \lambda x.2x$$

  $$dom(f) = cs:\Omega$$

  $$ran(f) = cs:Root$$

  This has the virtue of clarity.  But each figure in a composition has
  its own "home" cspace, which we call $\Omega$, so we may end up with lots
  of $\Omega$s.  This would leave us to define f for each pair of spaces,
  even if they use the same function.  E.g.


  $$f = \lambda x.2x$$
  dom(f) = cs:\Omega.x, ran(f) = cs:Root.x
  ... draw some stuff into cs:\Omega
  dom(f) = cs:Gamma.x, ran(f) = cs:Root.x
  ... draw some stuff into cs:Gamma

  Here we use two spaces to draw stuff, but map them both to the same
  space using the same transformation.

  Advantage of lambda notation: much easier to express complex
  transformations.  Although vector/matrix notation is pretty handy.

  With lambda we don't need predefined functions like "translate",
  "scale", etc.  (Same is true of vector/matrix notation.)

  With lambda we also get non-linear transformations, e.g. 
  \(f = \lambda x.x^2\)

  So a key design decision: how to express transformation functions.
  lambda or matrix?

  Another key design decision: how to express the dom and ran of
  transformation functions.

  [We seem to be approaching something like the standard notion of
    layers.  But with the notion of cspace, we don't need layers.  Or
    rather, layering is a separate idea, which corresponds to order of
    composition.  We can think of cs:$\Omega$ as the top of a stack of
    cspaces, i.e. last (first) in a series of transforms, but we can use
    the same cspace in multiple stacks.]

  ================

  Transform functions v. positioning functions

  Both kinds are total functions; the difference is in how they are
  used.  Transform functions are to be applied to entire coordinate
  systems, mapping every point.  Positioning functions are just like
  transforms, but are intended to be used on subsets of the space,
  esp. on single points in path expressions.

  Some of TikZ's coordinate systems effectively function as positioning functions.

  ================

  Scales

  Note that some transform functions, e.g. scaling, can be applied to
  anything, not just axes.  For example, color gradients, color maps
  (see pgfplots).  So cspaces and axes and ink (styles) are all
  distinct, but transform functions can apply across the board.


  %% ``In the Cartesian plane the point (x, y) can also be represented in polar coordinates as

  %% (x, y) = (r\cos\theta, r\sin\theta)\qquad(r, \theta) = \left(\sqrt{x^2+y^2}, \quad \arctan\frac{y}{x}\right).'' http://en.wikipedia.org/wiki/Complex_plane


  %%%%%%%%%%%%%%%%%%%%%%%%%%%%%%%%
  \part{Representation}
  \label{sect:representation}

  \begin{remark}
    Two topics:
    \begin{itemize}
    \item How to visually represent mathematical \cspace{}s
    \item How to use \cspace{}s (and their visual representations) to
      represent the real world (or possible worlds) (as with e.g. OpenGL coord. spaces).
    \end{itemize}
  \end{remark}

  %%%%%%%%
  \subsection{Point Expressions}
  \label{subs:pointexprs}

  \ab as expression - denotes a pair of reals (type \RR), not a point in
  a \cspace{}.  Note: \RR is a type expression, not a cs.  If we want to
  think of \RR as \Rtwo space, then it must be a (or: the one and only)
  absolute space, with axiomatically defined axes, etc.  A \ab
  expression outside of an cs context does not denote a coordinate
  point.  Inside a cs context, it denotes a pair in \RR (\Rtwo) which
  the cs scaling functions map to a point in the \Rtwo associated with
  the cs.  (CS a combo of \Rtwo, axes, etc.)

  So it is incorrect to think of a cs only in terms of \Rtwo.  It's a
  different mathematical structure.

  A pair of reals \ab only becomes a ``point'' when interpreted ``by'' a
  cs, mapping from pure \Rtwo to a cs structure - which we can think of
  as a mapping from pure number ``location''.

  %%%%%%%%
  \subsection{Space Expressions}
  \label{subs:spaceexprs}

  %%%%%%%%
  \subsection{Figure Expressions}
  \label{subs:figexprs}

  %%%%%%%%%%%%%%%%%%%%%%%%%%%%%%%%
  \section{Canvases}
  \label{sect:canvas}

  \begin{remark}
    A canvas is a space together with an empirical unit of measure,
    e.g. 1cm.  So unlike a \cspace{}, a canvas is not an abstract,
    purely mathematical object.\marginnote{TODO: how does TikZ handle
      the distinction?}

    Another perspective: a \cspace{} has a mathematical but not an
    empirical denotation;\sidenote{Except perhaps space itself} a canvas
    has both.  A canvas represents some physical surface, albeit
    abstractly, whereas a \cspace{} represents
    nothing.\marginnote{Caveat: we're talking about mathematical enties,
      not expressions in a languge.  Of course, a \cspace{}
      \textit{expressions} represents something, namely a \cspace{}; but
      the space itself, as a mathematical ``object'', does not
      represent.}
  \end{remark}

  In a \cspace{}, a point expression \ab{} denotes a \textit{position} in
  a structure; in a canvas, the same expression denotes a
  \textit{location} in (physical) space.\marginnote[-24pt]{Convention:
    positions are structural, locations spatial.}

  Numbers ``in'' a \cspace{} just \textit{are} structural positions;
  ``on'' a canvas, they represent spatial
  location.\marginnote[-12pt]{``Space'' and ``spatial'' are ambiguous
    here. etc.  We should probably refer to coordinate structures rather
    than \cspace{}s.}

  Specifying a point on a canvas is analogous to dabbing a bit of paint
  on a literal canvas.  Specifying a point in a \cspace{} is better
  thought of as analogous to memorizing how to get to a place in a city
  - five blocks west and two blocks north of the
  Courthouse.\marginnote{Recall that numbers can be thought of as
    methods or arcs, rather than objects.}

  %%%%%%%%%%%%%%%%%%%%%%%%%%%%%%%%
  \section{Connecting Content to Graphs}
  \label{sect:geomcontent}

  ================

  For browsers (rather, XML structures), treat the DOM as coordinate
  space?

  ================

  D3 first "selects" DOM nodes, then brings in the data.  That seems
  backwards.  First read the data, then project to the design space.

  ================

  The DOM+CSS model.  Boxes (some of them, anyway) implicitly define
  coordinate systems.  Adding an element - a paragraph, say - to the DOM
  does two things: it inserts a node into the (abstract) tree, and it
  maps the new node to a figure located in a coordinate space.  So to
  the hierarchy of the DOM corresponds a hierarchy of cspaces.
  Ultimately constrained by the display device.

  What about adding an svg circle (d3 examples)?  The svg cspace gets
  inserted into the cspace of the doc.  So there is an implicit
  transform mapping coords from the cs:SVG to cs:DOM. Or rather than
  cs:DOM, to whatever cspace is associated with the DOM parent of the
  svg node.  Call it the "host" cspace?  In any case, this is all
  implicit.

  A DOM node always has an associated piece of geometry.  Compare TikZ
  nodes, which are named bits of geometry.

  Note that you cannot transform a cspace you are attaching to the DOM;
  attaching always means mapping origin to origin. (?)  You can
  transform subspaces within the cs:SVG, but you cannot use any
  transform other than the identity transform for mapping cs:SVG to
  cs:Host.

  Weeell, not really; you can use CSS positioning and offsetting.  So
  you can specify a transform, but not directly as a transform.  That's
  one of the design shortcomings of CSS.

  Plus, you rarely know where a host node cspace is itself going to end
  up after composition.

  In any case, the goal pursued here is investigation of DOM as a kind
  of coordinate space.  Or more generally (covering XML as well as
  HTML), of tree structured data and their relation of graphic
  coordinate spaces.


  ================

  Kinds of "content coordinate spaces".  Sequence, Relation, Tree, Graph
  (net).  

  Plaintext is a sequential structure, so display of plaintext editor
  maps sequence to one cspace.

  Postscript has some hierarchy - pages, for example.  But basically the
  graphical figs on a postscript page do not map to a tree structure.
  You can reference variables, data structures, etc. within PS, but
  that's all internal.  There is no content data structure that stands
  separate from the language, as is the case with XML, where there is a
  data structure and API used by client languages.

  PDF?  Another page description language; I don't recall much of a
  content model standing separate from the graphic language, but it's
  been quite a while since I looked.

  In all these cases, when you add a figure to a space, you say where it
  is to be placed.  You might express that in terms of offsets or
  something, but you always refer to some known point of reference.  You
  don't just say "offset (x,y) from the upleft corner of parent
  figure, wherever that happens to end up" as is the case with HTML+CSS.

  Hmm.  Better review my PS/PDF.  As in most graphics languages, when
  you position something at (x,y) in PS/PDF, the implicit assumption is
  that the cspace involved may be transformed later.  So all positioning
  is relative until demonstrated otherwise.

  So what are we looking for here?  Uniformity of expression?  We want
  to use the same positioning language for both web and print targets.
  In particular, we want to completely hide the DOM, since a DOM API
  would be meaningless for e.g. PDF.  The D3 strategy of "selecting" DOM
  nodes and then binding data makes no sense if there is no DOM.

  A graphics language just specifies graphicky stuff, not content stuff.
  D3 and similar muddies the distinction by mingling content access
  vocab and graphics vocab.

  One approach: treat DOM nodes (and selectors) as names of coordinates
  in cspace.  Compare named nodes in TikZ/PGF.  Then the semantics would
  be purely geometric.  At least for accessing; what about
  adding/appending?  Appending a DOM node corresponds to specifying
  another point in cspace.  Think path v. ink.  Appending an empty div
  node is just like adding a point object to the path collection.  The
  difference is that the cs coords of the empty div are unresolved; it's
  like saying "add a point object, but I'll tell you the coordinates
  later".  In other words, like an unevaluated lambda(?), or a "virtual"
  point.  Another diff: DOM objects inherently ordered; add two empty
  divs and they will retain that order.  Add two virtual points and they
  are unordered.  Or: the only order in a cspace is that defined by the
  geometry of the space.  The temporal order in which stuff gets added
  is discarded.

  Or: the order of an XML tree is just abstract order, A precedes B,
  without any concrete definition of what "precedes" means (i.e. it does
  not mean greater than, less than, etc.)

  %%%%%%%%%%%%%%%%%%%%%%%%%%%%%%%%
  \section{Caveats}
  \label{sect:caveats}

  Beware of people who contrast ``logical'' and ... something else.  For
  example, the Unicode folks contrast ``logical order'' and ``visual
  order''
  (\url{http://userguide.icu-project.org/transforms/bidi#TOC-Logical-Order-versus-Visual-Order}).  Some people refer to ``logical dimensions'' and ``display dimension'' in HTML5.

%%%%%%%%%%%%%%%%%%%%%%%%%%%%%%%%%%%%%%%%%%%%%%%%%%%%%%%%%%%%%%%%
\part{Notes}

  %%%%%%%%%%%%%%%%%%%%%%%%%%%%%%%%
  \section{Algebra of Coordinate Spaces}
  \label{sect:csalgebra}


  In addition to origin and axes, coordinate systems always have an
  algebra, which specifies operations on coordinate points.  This allows
  us to distinguish between e.g. cartesian and complex cspaces.

  Cartesian:

  \begin{itemize}
  \item +  =  scale unit of source arithmetically; e.g. cs:A[(0,0)]+(1,1)
  \item *  =  scale unit of source geometrically: (x,y)*(a,b) = (ax,by) e.g. cs:A[(1,1)]*(2,2) 
  \item ?  =  rotation
  \item ?  =  skew (shear)
  \item ?  =  reflection
  \item ?  =  glide reflection (http://en.wikipedia.org/wiki/Transformation\_(function))
  \end{itemize}

  NB: "scale" usually means scale multiplicatively, but it applies
  equally to arithmetic transformation.  We can talk of arithmetic,
  geometric, harmonic, etc. scaling.

  Complex:

  * = (x+yi)*(a+bi) = (xa - yb) + (xb + ya)i

  (x,y)*(a,b)   = ((xa-yb), (xb+ya))

  e.g. (1,0)(0,1) = ((0-0), (1+0)) = (0,1) = i
  ================

  %%%%%%%%%%%%%%%%%%%%%%%%%%%%%%%%
  \section{Primitive Coordinate Spaces}
  \label{sect:primitives}

  \begin{remark}
    Only one truly primitive, axiomatic space: cartesian?  All the
    others can be transformed into plain cartesian space.  But for
    practical reasons we will include some predefined spaces that will
    count as primitive in our language, even if they are not
    mathematically primitive.  Primitive expressivity, as it were.
  \end{remark}

  Affine:  "In mathematics, an affine coordinate system is a coordinate system on an affine space where each coordinate is an affine map to the number line. In other words, it is an injective affine map from an affine space A to the coordinate space Kn, where K is the field of scalars, for example, the real numbers R.  The most important case of affine coordinates in Euclidean spaces is real-valued Cartesian coordinate system. Orthogonal affine coordinate systems are rectangular, and others are referred to as oblique." (http://en.wikipedia.org/wiki/Affine\_coordinates)

  Homogenous:  "In mathematics, homogeneous coordinates or projective coordinates, introduced by August Ferdinand Mobius in his 1827 work Der barycentrische Calcul,[1][2] are a system of coordinates used in projective geometry, as Cartesian coordinates are used in Euclidean geometry. They have the advantage that the coordinates of points, including points at infinity, can be represented using finite coordinates. Formulas involving homogeneous coordinates are often simpler and more symmetric than their Cartesian counterpart." (http://en.wikipedia.org/wiki/Homogeneous\_coordinates)

  \begin{itemize}
  \item Cartesian  (logarithmic a special case?)
  \item Polar
  \item Complex
  \item Cylindrical
  \item Spherical
  \item Log-polar
  \end{itemize}

  ================


  Primitive cspaces always same, alterations come via transforms.
  Untransformed, the default is just plain Cartesian space, unit=1,
  scale is identity func, incidence 0 and 90, orientation standard.

  But what about polar space?  We can transform a cartesian space into a
  polar space, but we cannot draw directly into a polar space using
  cartesian coordinates.  So we need polar space as a primitive space.
  Two axes: angular (length along circumference) and radial (length of
  vector).

  %%%%%%%%%%%%%%%%%%%%%%%%%%%%%%%%
  \section{Misc}
  \label{sect:misc}

  Drawing a figure always involves two specifications, one explicit, the
  other (usually) implicit.  The explicit specification describes the
  figure: size, shape, location\sidenote{Design space includes a style
    element involving color, line thickness, etc. but we can ignore that
    for present purposes.}.  The implicit specification describes the
  coordinate space.  The explicit specification (of a figure) is always
  relative to a coordinate space.

  CS (Coordinate Space) specification:  origin (always 0) + axes


  Example: unit circle centered at the origin.  The usual way to specify
  this is something like

  (0,0 circle(1)  (using tikz syntax; any other would do as well)

  This is the explicit specification; implicit is the specification of the CS.

  \begin{remark}
    The idea is that expressions like \((a,b)\) find a first home in the
    Root space, which is standard, absolute, Cartesian space.  You
    cannot ``touch'' Root space; its fixed.  But you can ``insert''
    things into Root design space, and you can apply transforms to it to
    yield other spaces.
  \end{remark}

  But there are always two coordinate spaces: the one in which
  coordinates are directly expressed (as above), which we'll call the
  Root space, and the "\Omg" space to which the Root space is mapped.
  In other words, specifying a figure and a space is not enough; the
  space itself must be located in a "parent" space, which we're calling
  the \Omg space.  Maybe Alpha space would be better.\marginnote{Since
    Alpha is beginning and Omega is end; we begin by specifying points
    in absolute space, map spaces to spaces, and end up in device space
    (=\Omg space).}

  [Root space is axiomatic: x and y axes with unit = 1, incidence 0 and
    90, the usual orientation.  Root space is also absolute.]

  IOW, the minimal transform is the \(Ident\) transform, which maps the
  Root space directly to \Omg space with no changes.  Expressions in
  which a transform is not explicitly specified always implicitly use
  the \(Ident\) transform.

  \begin{remark}
    All spaces are immutable; all we can do is express mappings between spaces.
  \end{remark}

  [NB this is all device independent.  Mapping root space to device
    space is a separate issue.]

  [NB third kind of space, between device independent (abstract) root
    space and device space: device-independent space with empirical units
    of measure.  E.g. x and y axes with unit = 1cm (instead of just 1).]

  \clearpage
  %%%%%%%%%%%%%%%%%%%%%%%%%%%%%%%%%%%%%%%%%%%%%%%%%%%%%%%%%%%%%%%%
  \part{Appendices}
  %% \appendix
  %% \appendixpage
  \begin{appendices}
    %%%%%%%%%%%%%%%%%%%%%%%%%%%%%%%%
    \section{Formal Definitions}
    \label{sect:formadefs}

    %%%%%%%%
    \subsection{Coordinate}
    \label{subs:coorddefn}

    \[COORD == \real\]

    %%%%%%%%
    \subsection{Angle}
    \label{subs:angledefn}

    \[RADIAN == \real\]

    \[DEGREE == \real\]

    \begin{axdef}
      degToRad : \real\fun\real \\
      radToDeg : \real\fun\real
      \where
      \forall x : \real @ degToRad(x) = x\cdot\frac{\pi}{180} \\
              {\small\strut} \\
              \forall x : \real @ radToDeg(x) = x\cdot\frac{180}{\pi}
    \end{axdef}

    %%%%%%%%
    \subsection{Point}
    \label{subs:pointdefn}

    \[POINT == COORD\cross COORD \]

    %%%%%%%%
    \subsection{Identity Function}
    \label{subs:idfndefn}

    \begin{gendef}[X]
      ident : X\fun X
      \where
      \forall x : X @ ident(x) = x
    \end{gendef}

    %%%%%%%%
    \subsection{Axis}
    \label{subs:axisdefn}

    An \textit{axis} is ...\marginnote{Do we need to explicitly list the carrier set (\R)?}

    \begin{schema}{Axis}
      origin : COORD \\
      offset : COORD \\
      incidence : RADIAN \\
      scale : COORD\fun COORD
      \where
      origin = (0,0) \\
      offset = scale(origin)
    \end{schema}

    %%%%%%%%
    \subsection{\CSpace}
    \label{subs:csdefn}

    A \textit{\cspace} combines \RR{} and a sequence of axes.\marginnote{Do we need to explicitly list the carrier set (\Rtwo)?}

    %%%%%%%%
    \subsection{\DSpace}
    \label{subs:dspacedefn}

    A \textit{\dspace} is a \cspace{} plus a set of figures.

    %%%%%%%%
    \subsection{Curve}
    \label{subs:curvedefn}

    A curve is a set of points, that is, a graph.

    %%%%%%%%
    \subsection{Path}
    \label{subs:pathdefn}

    A path is a function; it denotes a curve.  Techically, a path denotes
    its graph.

    %%%%%%%%%%%%%%%%%%%%%%%%%%%%%%%%
    \section{Previous Work}
    \label{sect:prevwork}

    \epigraph{Immature poets imitate; mature poets steal}{TS Eliot}


    %%%%%%%%
    \subsection{\TeX, \MF, and \MP}

    %%%%%%%%
    \subsection{TikZ/PGF}
    \label{subsec:tikz}

    TikZ has by far the most sophisticated concept of coordinate space.

    Canvas =  x,y with explicit "dimension" (empirical UOM)

    XYZ    =  x,y,z as "factors" multiplying unit vectors of length 1cm)

    Canvas Polar = radius expressed in canvas system, i.e. with UOM

    XYZ Polar  = radius and angle interpreted in as in xy csystem, i.e. factors (of unit vectors of 1cm) not dimensions

    XY Polar (=XYZ Polar)

    Barycentric  -  "In geometry, the barycentric coordinate system is a coordinate system in which the location of a point of a simplex (a triangle, tetrahedron, etc.) is specified as the center of mass, or barycenter, of masses placed at its vertices. Coordinates also extend outside the simplex, where one or more coordinates become negative."   (http://en.wikipedia.org/wiki/Barycentric\_coordinate\_system)

    Node (=DOM?)  The cs:node is parasitic the more primitive empirical
    systems.  It allows you to refer to a previously specified point by
    specifying its name and an "anchor" or predefined location relative to
    the node's position.  The anchor name effectively denotes a function
    that computes a point, given the node' base point.

    Tangent Like cs:node, cs:tangent supports specification of points by
    refering to previously specified points.  A tangent function is
    implicitly applied to the points refered to - actually a point and a
    figure - and computes the tangent from the point to the figure.

    Custom -- the job of a custom cs is to compute x and y in the canvas
    coordinate system.  So defining a custom coordinate system boils down
    to specifying a transformation function.

    TikZ's XYZ system behaves like an abstract Euclidean system.  But it
    does have default unit vectors that map it to canvas space.

    \begin{remark}
      Compare Asymptote.  TikZ systems that allow unitless coordinate
      specifications (e.g. XYZ) use an implicit empirical UOM (default:
      1cm).  Any \cspace{} must eventually be mapped to a canvas.
    \end{remark}

    Thus all TikZ cspaces are empirical; some seem to provide means of
    expressing coordinates rather than establishing a distinct system;
    e.g. node and tangent systems.  They don't (at least not all of them)
    establish a primitive interpretation of (x,y).  For example, the
    tangent cs does not provide a meaning for any (x,y); instead, it
    allows you to specify a (tangent) point by reference to an already
    specified pair of a shape and a point.

    In other words, cs:node and cs:tangent are derived systems, the others
    primitive.  They are essentially transform functions using for
    positioning.  They do not need units of measure since that is already
    implicit in the points they reference.  Dunno about barycentric, which
    I don't really grok yet.

    Basically all the cspaces amount to positioning functions, which is
    why they can be mixed in single path expressions.

    Correction: they are NOT transform functions, in that they do not
    transform the entire csystem; instead they are positioning functions,
    designed to transform subsets of the space.

    Furthermore, TikZ's system allows for use of multiple coord systems in
    one expression.  So you can write (0,0) -- (polar cs:
    cs:angle=90,radius=1cm), using first the xyz (Id func) system then the
    polar system (a positioning function) to draw a line from the origin
    to (0,1), unit=cm.

    Note that the xyz cs seems to be the default for (x,y) syntax, where x
    and y are dimensionless.

    %%%%%%%%
    \subsection{Postscript}

    Figure specification as executable program.

    User space, device space

    Units

    %%%%%%%%
    \subsection{PDF}

    Ref:  PDF32000-1:2008 (\url{http://www.adobe.com/devnet/pdf/pdf\_reference.html}

    ``canvas''

    ``[A] PDF content stream is not a program to be interpreted; rather, it is a static description of a sequence of graphics objects.'' p.110

    Five ``graphics objects'': path, text, external, inline image, shading

    Add: clipping path

    ``A shading object describes a geometric shape whose colour is an arbitrary function of position within the shape. (A shading can also be treated as a colour when painting other graphics objects; it is not considered to be a separate graphics object in that case.)'' (p. 111)


    8.3 Coordinate Systems (p. 114)

    ``Coordinate systems define the canvas on which all painting occurs.'' (p.114)

    ``A coordinate space is determined by the following properties with respect to the current page:
    • The location of the origin
    • The orientation of the x and y axes
    • The lengths of the units along each axis'' p.114

    Device space

    User space
    ``Coordinates in user space (as in any other coordinate space) may be specified as either integers or real numbers, and the unit size in default user space does not constrain positions to any arbitrary grid. The resolution of coordinates in user space is not related in any way to the resolution of pixels in device space.'' p.115

    Transforms: translation, scaling, rotation, skewing


    ``The transformation from user space to device space is defined by the current transformation matrix (CTM), an element of the PDF graphics state...'' p.116

    Other coordinate spaces: text space (w/text matrix), glyph space
    (w/font matrix), image space (w/out transforms), form space (matrix),
    pattern space (matrix), 3D stuff

    See figure 12, p. 117

    %%%%%%%%
    \subsection{CSS}

    \url{http://www.w3.org/TR/CSS2/box.html}

    \href{http://www.w3.org/TR/CSS2/visuren.html}{Visual Formatting Model}

    \href{http://www.w3.org/TR/CSS2/visuren.html#containing-block}{Containing
      Block} ``In CSS 2.1, many box positions and sizes are calculated
    with respect to the edges of a rectangular box called a containing
    block. In general, generated boxes act as containing blocks for
    descendant boxes; we say that a box "establishes" the containing block
    for its descendants. The phrase "a box's containing block" means "the
    containing block in which the box lives," not the one it
    generates....Each box is given a position with respect to its
    containing block, but it is not confined by this containing block; it
    may overflow.''

    So the ``containing block'' is both geometric and structure.  CSS
    generally maps geometric structure to content structure directly.

    \href{http://www.w3.org/TR/CSS2/visuren.html#normal-flow}{9.4 Normal Flow}

    CSS \href{http://www.w3.org/TR/CSS2/media.html}{media types}; \href{http://www.w3.org/TR/css-print/}{CSS Print Profile}

    \href{http://www.w3.org/TR/css3-page/}{CSS Paged Media Module Level 3}

    \href{http://www.w3.org/TR/css3-page/#page-model}{The Page Model}

    %%%%%%%%
    \subsection{SVG}

    \url{http://www.w3.org/TR/SVG11/coords.html}

    Viewport space, user spaces

    %%%%%%%%
    \subsection{HTML5 Canvas}
    \label{subs:html5}

    ``The canvas element provides scripts with a resolution-dependent bitmap canvas, which can be used for rendering graphs, game graphics, art, or other visual images on the fly.'' \url{http://www.w3.org/TR/html5/scripting-1.html#the-canvas-element}

    \url{http://diveintohtml5.info/canvas.html}

    \href{https://developer.apple.com/library/safari/documentation/AudioVideo/Conceptual/HTML-canvas-guide/Introduction/Introduction.html#//apple_ref/doc/uid/TP40010542-CH1-SW1}{Safari HTML5 Canvas Guide}

    ``The canvas is a two-dimensional grid. The coordinate (0, 0) is at the upper-left corner of the canvas. Along the X-axis, values increase towards the right edge of the canvas. Along the Y-axis, values increase towards the bottom edge of the canvas.'' (\url{http://diveintohtml5.info/canvas.html})

    Each canvas element has a ``drawing context''.

    ``One of the most exciting features of HTML5 is the <canvas> element that can be used to draw vector graphics...'' \url{http://www.1stwebdesigner.com/design/creative-ways-use-html5-canvas/}


    \href{http://msdn.microsoft.com/en-us/hh534406.aspx}{The Developer’s Guide to HTML5 Canvas}

    \href{http://msdn.microsoft.com/en-us/library/ff975057.aspx}{CanvasRenderingContext2D object} - methods and props

    ``To draw on a canvas, you need to reference the context of the canvas.  The context gives you access to the 2D properties and methods that allow you to draw and manipulate images on a canvas element.'' \url{http://msdn.microsoft.com/en-us/hh534406.aspx}

    ``everything in the canvas is a pixel.'' http://msdn.microsoft.com/en-us/hh534406.aspx

    ``The canvas 2D API is an object that allows you to draw and
    manipulate images and graphics on a canvas element.  To reference the context of the canvas, you call getContext, which is a method on the canvas element.  It has one parameter, which currently is 2d.  Here’s the snippet of code for referencing the context.

    ``Each canvas has its own context, so if you’re page contains multiple canvas elements; you must have a reference to each individual context that you want to work with.'' \url{Canvas 2D
 API}  Correction: not an ``object'', a mathematical structure.

    \href{http://blogs.msdn.com/b/ie/archive/2011/04/22/thoughts-on-when-to-use-canvas-and-svg.aspx}{Thoughts on when to use Canvas and SVG}

    %%%%%%%%
    \subsection{OpenGL}
    \label{subs:opengl}


    \href{http://www.matrix44.net/cms/notes/opengl-3d-graphics/coordinate-systems-in-opengl}{Coordinate Systems in OpenGL}

    ``There are multiple coordinate Systems involved in 3D Graphics:

    Object Space
    World Space (aka Model Space)
    Camera Space (aka Eye Space or View Space)
    Screen Space (aka Clip Space)''

    \url{http://www.cse.ohio-state.edu/~parent/classes/581/Lectures/4.2DviewingHandout.pdf}

    %%%%%%%%
    \subsection{OS X/iOS Drawing}

    \href{https://developer.apple.com/library/mac/documentation/graphicsimaging/conceptual/drawingwithquartz2d/dq\_overview/dq\_overview.html}{Quartz 2D}

    \href{https://developer.apple.com/library/mac/documentation/graphicsimaging/conceptual/drawingwithquartz2d/dq\_overview/dq\_overview.html#//apple\_ref/doc/uid/TP30001066-CH202-CJBBAEEC}{Quartz 2D Coordinate Systems}

    Cocoa (OS X):  \url{https://developer.apple.com/library/mac/documentation/cocoa/conceptual/CocoaDrawingGuide/Transforms/Transforms.html#//apple\_ref/doc/uid/TP40003290-CH204-BCIDJJBI}

    iOS:  ??

    %%%%%%%%
    \subsection{Android Drawing}

    \href{http://developer.android.com/guide/topics/graphics/2d-graphics.html}{Canvas and Drawables}

    Transform operators are in the API ref, but the dev guides don't say much about it.

    %%%%%%%%
    \subsection{Direct3D, Direct2D}
    \label{subs:direct3d}


    \href{http://msdn.microsoft.com/en-us/library/dd370990%28VS.85%29.aspx}{Direct2D}

      \href{http://msdn.microsoft.com/en-us/library/dd756655(v=vs.85).aspx#the_direct2d_coordinate_space}{The Direct2D Coordinate Space}

      \href{http://msdn.microsoft.com/en-us/library/windows/desktop/ff476345(v=vs.85).aspx}{Programming Guide for Direct3D}

      \href{http://msdn.microsoft.com/en-us/library/windows/desktop/bb324489(v=vs.85).aspx}{3D coordinate systems and geometry}

      %%%%%%%%
      \subsection{R}
      \label{subs:r}

      Each ``plotting region'' has its own coordinate space.

      %%%%%%%%
      \subsection{Octave}
      \label{subs:octave}

      Graphic objects: figure, axes, line, text, patch, surface, text and image.

      %%%%%%%%%%%%%%%%%%%%%%%%%%%%%%%%
      \section{Bibliography}
      \label{sect:bib}

      %% \addcontentsline{toc}{chapter}{Bibliography}
      \bibliography{spaces.bib% don't omit the %!
        %% ../bib/logic.bib%
        %% ,../bib/math.bib%
        %% ,../bib/stats.bib%
      }
      \bibliographystyle{plainnat}
      %% \printbibliography[heading=none]
  \end{appendices}

\end{document}
